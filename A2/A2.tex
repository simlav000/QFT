\documentclass[12pt]{article}

% Packages
\usepackage[margin=1in]{geometry}
\usepackage{fancyhdr}
\usepackage{titlesec}
\usepackage{amsmath, amssymb, bm}
\usepackage{cancel}

\usepackage{mathtools}
\newcommand{\w}{\omega}
\newcommand{\phip}{\phi^{'}}
\newcommand{\phipp}{\phi^{''}}
\newcommand{\delmu}{\partial_{\mu}}
\newcommand{\delMu}{\partial^{\mu}}
\newcommand{\wk}{\omega_{\mathbf{k}}}
\newcommand{\wkp}{\omega_{\mathbf{k'}}}
\DeclarePairedDelimiter\bra{\langle}{\rvert}
\DeclarePairedDelimiter\ket{\lvert}{\rangle}
\DeclarePairedDelimiterX\braket[2]{\langle}{\rangle}{#1\,\delimsize\vert\,\mathopen{}#2}
\usepackage{braket}

% Header & Footer
\pagestyle{fancy}
\fancyhf{}
\fancyhead[L]{Name:  \underline{Simon Lavoie}}
\fancyhead[C]{Class: \underline{PHYS 610}}
\fancyhead[R]{Student ID: \underline{261 051 325}}
\fancyfoot[C]{\thepage}

% Title format for neatness
\titleformat{\section}{\large\bfseries}{\thesection.}{0.5em}{}
\titleformat{\subsection}{\normalsize\bfseries}{\thesubsection)}{0.5em}{}

% Document
\begin{document}

\begin{center}
    \Large \textbf{Assignment \#2}
\end{center}
\vspace{1cm}

\section*{Question 1}
To obtain the equation of motion, we start with the action

\begin{equation*}
    S = \frac{1}{2} \int d^4x \left[ \partial_\mu \phi(x) \, \partial^\mu
    \phi(x) - \phi^2(x) \, \exp\left( \int d^4y \, \Big( F_1(x-y)\,
\phi(y) + F_2(x-y)\, \phi^3(y) \Big) \right) \right]
\end{equation*}

For convenience, we'll label the integral in the exponent as $I(x)$ and we
compute the variation $\delta S$ of this action. Namely

\begin{equation*}
    \delta S = \int d^4x \left[ \delmu\phi\delMu(\delta\phi)
    -\phi\delta\phi e^{I(x)} -\frac{1}{2}\phi^2 \delta(e^{I(x)}) \right]
\end{equation*}

Let's focus on $\delta(e^{I(x)})$

\begin{align}
    \delta(e^{I(x)}) &= e^{I(x)} \delta I(x) \\
                     &= e^{I(x)} \int d^4y [F_1(x - y)\delta\phi(y) + F_2(x - y)
                     3\phi^2(y)\delta\phi(y)] \\
                     &= e^{I(x)} \int d^4y [F_1(x-y) + 3F_2(x-y)\phi^2(y)] \delta\phi(y) \\
\end{align}

Substituting this into our variation and letting $K(x, y) = F_1(x - y) + 3F_2(x - y)\phi^2(y)$

\begin{align*}
    \delta S = \int d^4x \left[ \delmu\phi\delMu(\delta\phi)
    -\phi\delta\phi e^{I(x)} -\frac{1}{2}\phi^2 e^{I(x)} \int d^4y K(x,y) \delta\phi(y) \right]
\end{align*}

We notice the first term can be computed via integration by parts

\begin{align}
\int d^4x \, \partial_\mu \phi(x)\, \partial^\mu \delta\phi(x)
&= \int d^4x \, \partial_\mu\!\big( \partial^\mu \phi(x)\,\delta\phi(x)\big)
   - \int d^4x \, (\partial_\mu \partial^\mu \phi(x))\,\delta\phi(x) \\
&= -\int d^4x \, \Box \phi(x)\,\delta\phi(x)
   + \int d^4x \, \partial_\mu\!\big( \partial^\mu \phi(x)\,\delta\phi(x)\big).
\end{align}

Where $\Box := \delmu\delMu$ and we recognize the second integral as the boundary
term. Putting this back into our variation:

\begin{align*}
\delta S
&= \int d^4x\;\bigg[-\Box\phi(x)\,\delta\phi(x)
   + \partial_\mu\!\big(\partial^\mu\phi(x)\,\delta\phi(x)\big)\bigg]
   -\int d^4x\;\phi(x)e^{I(x)}\delta\phi(x) \\
&\qquad -\tfrac{1}{2}\int d^4x\;\phi^2(x)e^{I(x)}
   \int d^4y\;K(x,y)\,\delta\phi(y),
\end{align*}

The last term is a double integral. Using Fubini's theorem we swap integrals so
that $\delta\phi(y)$ appears outside:
\[
\begin{split}
- \tfrac{1}{2}\int d^4x\;\phi^2(x)e^{I(x)}\int d^4y\;K(x,y)\,\delta\phi(y)
&= -\tfrac{1}{2}\int d^4y\;\delta\phi(y)\int d^4x\;\phi^2(x)e^{I(x)}K(x,y).
\end{split}
\]

$y$ here is a dummy variable, so we may rename it $x$ so that
all terms are written with the same outer variable. To avoid a clash with the
inner $x$, also rename that inner integration variable to $z$. This is purely
notational and changes nothing:

\[
- \tfrac{1}{2}\int d^4y\;\delta\phi(y)\int d^4x\;\phi^2(x)e^{I(x)}K(x,y)
= - \tfrac{1}{2}\int d^4x\;\delta\phi(x)\int d^4z\;\phi^2(z)e^{I(z)}K(z,x).
\]

Thus, our variation, once again

\[
\begin{split}
\delta S
&= \int d^4x\;\delta\phi(x)\Big[-\Box\phi(x)-\phi(x)e^{I(x)}
   -\tfrac{1}{2}\int d^4z\;\phi^2(z)e^{I(z)}K(z,x)\Big] \\
&\qquad + \int d^4x\;\partial_\mu\!\big(\partial^\mu\phi(x)\,\delta\phi(x)\big).
\end{split}
\]

Discarding the boundary term (assume $\delta\phi\to0$ at infinity) and
requiring $\delta S=0$ for arbitrary $\delta\phi(x)$ yields the EOM
\[
\boxed{%
\Box\phi(x)
+\phi(x)\,e^{I(x)}
+\tfrac{1}{2}\int d^4z\;\phi^2(z)\,e^{I(z)}[F_1(z-x)+3F_2(z-x)\,\phi^2(x)]
=0}
\qquad
\]
As desired.

Now, if we make the substitution $\phi(y) \mapsto \phi(y) + \lambda \Box' \phi(y)$, then $I(x)$ is modified:

\begin{align*}
I(x) &= \int d^4y \; \Big( F_1(x-y)\big(\phi(y) + \lambda \Box' \phi(y)\big) 
+ F_2(x-y) \big(\phi(y) + \lambda \Box' \phi(y)\big)^3 \Big), \\
\delta I(x) &= \int d^4y \; \Big[ F_1(x-y)\big(\delta\phi(y) + \lambda \Box' \delta\phi(y)\big) \\
&\qquad + 3 F_2(x-y) \big(\phi(y) + \lambda \Box' \phi(y)\big)^2 \big(\delta\phi(y) + \lambda \Box' \delta\phi(y)\big) \Big].
\end{align*}

Define $\theta(y) = \phi(y) + \lambda \Box' \phi(y)$ so that

\[
K(x,y) = F_1(x-y) + 3 F_2(x-y) \, \theta^2(y),
\]

and we may write

\[
\delta I(x) = \int d^4y \; \Big[ K(x,y) \, \delta\phi(y) + \lambda K(x,y) \, \Box' \delta\phi(y) \Big].
\]

The second term can be integrated by parts:

\begin{align*}
\lambda \int d^4y \; K(x,y) \, \Box' \delta\phi(y) &= \lambda \int d^4y \; \partial'_\mu \big( K(x,y) \, \partial'^\mu \delta\phi(y) \big) 
- \lambda \int d^4y \; (\partial'_\mu K(x,y)) \, \partial'^\mu \delta\phi(y), \\
- \lambda \int d^4y \; (\partial'_\mu K) \, \partial'^\mu \delta\phi &= - \lambda \int d^4y \; \partial'_\mu \big( (\partial'^\mu K) \, \delta\phi \big) 
+ \lambda \int d^4y \; (\Box' K) \, \delta\phi.
\end{align*}

Combining the total derivative terms gives the boundary contribution:

\[
\lambda \int d^4y \; \partial'_\mu \Big( K \, \partial'^\mu \delta\phi - (\partial'^\mu K) \, \delta\phi \Big),
\]

while the remaining bulk term is

\[
\lambda \int d^4y \; (\Box' K) \, \delta\phi.
\]

Thus we may write

\begin{align}
\delta I(x) &= \int d^4y \; \big[ K(x,y) - \lambda \Box'_y K(x,y) \big] \, \delta\phi(y) \nonumber \\
&\qquad + \lambda \int d^4y \; \partial'_\mu \Big( K(x,y) \, \partial'^\mu \delta\phi(y) - (\partial'^\mu K(x,y)) \, \delta\phi(y) \Big).
\end{align}

The first term contributes to the equation of motion, while the second term vanishes if $\delta\phi \to 0$ at infinity. Therefore, the variation of the action is

\begin{align}
\delta S &= \int d^4x \; \Big[ - \Box \phi(x) - \phi(x) \, e^{I(x)} \Big] \delta\phi(x) \nonumber\\
&\quad - \frac{1}{2} \int d^4x \int d^4z \; \phi^2(x) \, e^{I(x)} \big[ K(x,z) - \lambda \Box_z K(x,z) \big] \, \delta\phi(z),
\end{align}

where we have renamed the inner integration variable to $z$ for clarity. Swapping the order of integration in the second term, we obtain

\begin{align}
\delta S &= \int d^4z \; \delta\phi(z) \Bigg[ - \Box \phi(z) - \phi(z) \, e^{I(z)} 
- \frac{1}{2} \int d^4x \; \phi^2(x) \, e^{I(x)} \, \big[ K(x,z) - \lambda \Box_x K(x,z) \big] \Bigg].
\end{align}

Requiring $\delta S = 0$ for arbitrary $\delta\phi(z)$ yields the modified equation of motion:

\begin{equation}
\boxed{
\Box \phi(x) + \phi(x) \, e^{I(x)} + \frac{1}{2} \int d^4z \; \phi^2(z) \, e^{I(z)} \, \big[ K(z,x) - \lambda \Box_z K(z,x) \big] = 0,
}
\end{equation}

with

\[
K(z,x) = F_1(z-x) + 3 F_2(z-x) \, \theta^2(x), \qquad \theta(x) = \phi(x) + \lambda \Box \phi(x).
\]

Under the standard variational falloff $\delta\phi,\;\partial'\delta\phi\to0$ at infinity and for kernels $K(x,y)$ that do not grow at large $|y|$, the primed-surface integral
\[
\int_{\partial\mathbb R^4_y} d\Sigma'_\mu(y)\,\big( K\partial'^\mu\delta\phi-(\partial'^\mu K)\delta\phi\big)=0,
\]
so the extra boundary term under this translation vanishes and thus the boundary
terms do not change.

\section*{Question 2}
\subsection*{a)}

We wish to verify equations (2.25) through (2.31) of Peskin and Schroeder. Let
us start with equations (2.25) and (2.26), namely

\begin{align}
    \phi(\mathbf{x}) &= \int \frac{d^3k}{(2\pi)^3} \,
        \frac{1}{\sqrt{2\omega_{\mathbf{k}}}} \left( a_{\mathbf{k}}
        e^{i\mathbf{k}\cdot \mathbf{x}} + a_{\mathbf{k}}^{\dagger}
        e^{-i\mathbf{k}\cdot \mathbf{x}} \right). \\
    \pi(\mathbf{x}) &= \int \frac{d^3k}{(2\pi)^3}
        \,(-i)\sqrt{\frac{\w_{\mathbf{k}}}{2}} \left(a_{\mathbf{k}}e^{i\mathbf{k
        \cdot x}} - a_{\mathbf{k}}^{\dagger}e^{-i\mathbf{k \cdot x}}\right)
\end{align}

Note: I've taken the liberty to express these in terms of $\mathbf{k}$ as opposed
to $\mathbf{p}$ to be consistent with the notation used in class. 

The first thing we wish to do is unite the two mutually conjugate exponential
factors in each formula into one by making use of a simple trick. Consider
the $a_{\mathbf{k}}^{\dagger}e^{-i\mathbf{k \cdot x}}$ terms in equations [11] 
and [12]. They are both of the form

\begin{align*}
    \int_{-\infty}^{+\infty} d^3k \; f(\w_{\mathbf{k}}) a_{\mathbf{k}}^{\dagger}e^{-i\mathbf{k \cdot x}}
\end{align*}
Now, let us make the substitution $\mathbf{k} \mapsto -\mathbf{k}$. We obtain
\begin{align*}
    - \int_{+\infty}^{-\infty} d^3k \; f(\w_{-\mathbf{k}}) a_{\mathbf{-k}}^{\dagger}e^{i\mathbf{k
    \cdot x}}
\end{align*}
with the leading negative sign being due to the three factors $d^3k$. Flipping
the integral bounds back over, we obtain
\begin{align*}
    \int_{-\infty}^{+\infty} d^3k \; f(\w_{-\mathbf{k}}) a_{\mathbf{-k}}^{\dagger}e^{i\mathbf{k \cdot x}}
\end{align*}
Considering $\w_{\mathbf{k}} = \sqrt{m^2 + \mathbf{k}^2}$, we see that, in
particular, $\w_{-\mathbf{k}} = \w_{\mathbf{k}}$. These two pieces taken 
together then allow us to express (11) and (12) as

\begin{align}
    \boxed{\phi(\mathbf{x}) = \int \frac{d^3k}{(2\pi)^3} \,
        \frac{1}{\sqrt{2\omega_{\mathbf{k}}}}
        \left( a_{\mathbf{k}} + a_{-\mathbf{k}}^{\dagger} \right)
        e^{i\mathbf{k}\cdot \mathbf{x}}} \\
    \boxed{\pi(\mathbf{x}) = \int \frac{d^3k}{(2\pi)^3} \,
        (-i)\sqrt{\frac{\omega_{\mathbf{k}}}{2}}
        \left( a_{\mathbf{k}} - a_{-\mathbf{k}}^{\dagger} \right)
        e^{i\mathbf{k}\cdot \mathbf{x}} }
\end{align}

\noindent Next, we want to compute $\left[ \phi(\mathbf{x}), \pi{(\mathbf{x})} \right]$
using $\left[a_{\mathbf{k}}, a_{\mathbf{k}}^{\dagger} \right] =
(2\pi)^3\delta^{(3)}(\mathbf{k} - \mathbf{k}')$.

\[
\begin{aligned}
\big[\phi(\mathbf{x}),\pi(\mathbf{y})\big]
&= \int\frac{d^3k}{(2\pi)^3}\frac{d^3k'}{(2\pi)^3}
\frac{(-i)}{\sqrt{2\omega_{\mathbf{k}}}}\sqrt{\frac{\omega_{\mathbf{k}'}}{2}} \\
&\quad \times
\Big[ \big( a_{\mathbf{k}} e^{i\mathbf{k}\cdot\mathbf{x}} + a_{\mathbf{k}}^\dagger e^{-i\mathbf{k}\cdot\mathbf{x}}\big),
\big( a_{\mathbf{k}'} e^{i\mathbf{k}'\cdot\mathbf{y}} - a_{\mathbf{k}'}^\dagger e^{-i\mathbf{k}'\cdot\mathbf{y}}\big) \Big].
\end{aligned}
\]

\noindent Only the mixed commutators with one annihilation and one creation operator survive:
\[
\begin{aligned}
\big[\phi(\mathbf{x}),\pi(\mathbf{y})\big]
&= \int\frac{d^3k}{(2\pi)^3}\frac{d^3k'}{(2\pi)^3}\,\frac{(-i)}{2}
\Bigg( \sqrt{\frac{\omega_{\mathbf{k}'}}{\omega_{\mathbf{k}}}}
\big[ a_{\mathbf{k}},a_{\mathbf{k}'}^\dagger\big] e^{i\mathbf{k}\cdot\mathbf{x}} e^{-i\mathbf{k}'\cdot\mathbf{y}} \\
&\qquad\qquad\qquad\qquad
+ \sqrt{\frac{\omega_{\mathbf{k}}}{\omega_{\mathbf{k}'}}}
\big[ a_{\mathbf{k}}^\dagger,a_{\mathbf{k}'}\big] e^{-i\mathbf{k}\cdot\mathbf{x}} e^{i\mathbf{k}'\cdot\mathbf{y}}
\Bigg).
\end{aligned}
\]

\noindent Use $\big[a_{\mathbf{k}},a_{\mathbf{k}'}^\dagger\big]=(2\pi)^3\delta^{(3)}(\mathbf{k}-\mathbf{k}')$
and $\big[a_{\mathbf{k}}^\dagger,a_{\mathbf{k}'}\big]=- (2\pi)^3\delta^{(3)}(\mathbf{k}-\mathbf{k}')$ to get
\[
\begin{aligned}
\big[\phi(\mathbf{x}),\pi(\mathbf{y})\big]
&= \int\frac{d^3k}{(2\pi)^3}\frac{d^3k'}{(2\pi)^3}\,\frac{(-i)}{2}(2\pi)^3\delta^{(3)}(\mathbf{k}-\mathbf{k}')\\
&\quad\times\Bigg( \sqrt{\frac{\omega_{\mathbf{k}'}}{\omega_{\mathbf{k}}}}\,e^{i\mathbf{k}\cdot\mathbf{x}} e^{-i\mathbf{k}'\cdot\mathbf{y}}
+ \sqrt{\frac{\omega_{\mathbf{k}}}{\omega_{\mathbf{k}'}}}\,e^{-i\mathbf{k}\cdot\mathbf{x}} e^{i\mathbf{k}'\cdot\mathbf{y}}
\Bigg).
\end{aligned}
\]

\noindent Evaluating the $k'$ integral with the delta:
\[
\big[\phi(\mathbf{x}),\pi(\mathbf{y})\big]
= \int\frac{d^3k}{(2\pi)^3}\,\frac{(-i)}{2}
\Big( e^{i\mathbf{k}\cdot(\mathbf{x}-\mathbf{y})} + e^{-i\mathbf{k}\cdot(\mathbf{x}-\mathbf{y})} \Big).
\]

\noindent Since the integrand is even in $\mathbf{k}$, the two exponentials
give the same contribution. Combining them:
\[
\big[\phi(\mathbf{x}),\pi(\mathbf{y})\big]
= -i \int\frac{d^3k}{(2\pi)^3}\, e^{i\mathbf{k}\cdot(\mathbf{x}-\mathbf{y})}.
\]

\noindent Finally we recognize the integral to be the standard delta function
identity, obtaining the desired result:
\[
\boxed{\;[\phi(\mathbf{x}),\pi(\mathbf{y})]= i\,\delta^{(3)}(\mathbf{x}-\mathbf{y})\; }.
\]

Finally, we wish to express the Hamiltonian as an operator (in terms of the
creation and annihilation operators). For this, let us start with 

\[
H=\int d^3x\;\mathcal{H}(x)
=\frac{1}{2}\int
d^3x\;\big(\pi^2(\mathbf{x})+(\nabla\phi(\mathbf{x}))^2+m^2\phi^2(\mathbf{x})\big).
\]
Using (13) and (14) from above, we can express $\phi^2(\mathbf{x})$, 
$\pi^2(\mathbf{x})$, $\nabla\phi$ and, in particular $(\nabla\phi)^2$ as 

\begin{align*}
    \phi^2(\mathbf{x}) &= \int \frac{d^3kd^3k'}{(2\pi)^6} \,
    \frac{1}{2\sqrt{\omega_{\mathbf{k}}\omega_{\mathbf{k'}}}}
    \left( a_{\mathbf{k}} + a_{-\mathbf{k}}^{\dagger} \right) \left(
    a_{\mathbf{k'}} + a_{-\mathbf{k'}}^{\dagger} \right)
    e^{i\mathbf{k}\cdot \mathbf{x} + i\mathbf{k'}\cdot \mathbf{x}} \\
\pi^2(\mathbf{x}) &= - \int\frac{d^3kd^3k'}{(2\pi)^6}
\frac{\sqrt{\w_{\mathbf{k}}\w_{\mathbf{k'}}}}{2} \left( a_{\mathbf{k}} -
a_{-\mathbf{k}}^{\dagger} \right) \left( a_{\mathbf{k'}} -
a_{-\mathbf{k'}}^{\dagger} \right) e^{i\mathbf{k}\cdot \mathbf{x} +
i\mathbf{k'}\cdot \mathbf{x}} \\
\nabla\phi(\mathbf{x}) &=
\int\frac{d^3k}{(2\pi)^3}\;\frac{i\mathbf{k}}{\sqrt{2\omega_k}}
\left( a_{\mathbf{k}} + a_{-\mathbf{k}}^{\dagger} \right)e^{i\mathbf{k}\cdot\mathbf{x}},
\end{align*}

hence
\[
(\nabla\phi(\mathbf{x}))^2
= -\int\frac{d^3k\,d^3k'}{(2\pi)^6}\;
\frac{\mathbf{k}\!\cdot\!\mathbf{k}'}{2\sqrt{\omega_k\omega_{k'}}} \left(
a_{\mathbf{k}} + a_{-\mathbf{k}}^{\dagger} \right)\left( a_{\mathbf{k}'} +
a_{-\mathbf{k}'}^{\dagger} \right)
e^{i(\mathbf{k}+\mathbf{k}')\cdot\mathbf{x}}.
\]

Thus the Hamiltonian becomes
\begin{align*}
H &= \frac{1}{2} \int d^3x \int \frac{d^3k\,d^3k'}{(2\pi)^6} 
   \Bigg\{
      \frac{1}{2\sqrt{\omega_k \omega_{k'}}}
      \bigl(a_{\mathbf{k}} + a_{-\mathbf{k}}^{\dagger}\bigr)
      \bigl(a_{\mathbf{k'}} + a_{-\mathbf{k'}}^{\dagger}\bigr) \\
&\quad + \frac{\sqrt{\omega_k \omega_{k'}}}{2}
      \bigl(a_{\mathbf{k}} - a_{-\mathbf{k}}^{\dagger}\bigr)
      \bigl(a_{\mathbf{k'}} - a_{-\mathbf{k'}}^{\dagger}\bigr) \\
&\quad + \frac{\mathbf{k}\!\cdot\!\mathbf{k}'}{2\sqrt{\omega_k \omega_{k'}}}
      \bigl(a_{\mathbf{k}} + a_{-\mathbf{k}}^{\dagger}\bigr)
      \bigl(a_{\mathbf{k'}} + a_{-\mathbf{k'}}^{\dagger}\bigr)
   \Bigg\}
   e^{i(\mathbf{k}+\mathbf{k}')\cdot \mathbf{x}} .
\end{align*}


Perform the \(\mathbf{x}\)-integral to produce the delta, which collapses the
\(\mathbf{k}'\)-integral and turns each $\mathbf{k}' \mapsto -\mathbf{k}$.
Recall that $\w_{-\mathbf{k}} = \wk$
\[
\begin{aligned}
(2\pi)^3\delta^{(3)}(\mathbf{k}+\mathbf{k}') = \int d^3x\;
e^{i(\mathbf{k}+\mathbf{k}')\cdot\mathbf{x}}
\end{aligned}
\]

\begin{align*}
H &= \frac{1}{2}\int\frac{d^3k}{(2\pi)^3}\Bigg[
    \frac{m^2}{2\omega_k}\bigl(a_{\mathbf k}+a_{-\mathbf k}^\dagger\bigr)
    \bigl(a_{-\mathbf k}+a_{\mathbf k}^\dagger\bigr)\\
&\qquad\qquad\qquad\qquad
    +\frac{\omega_k}{2}\bigl(a_{\mathbf k}-a_{-\mathbf k}^\dagger\bigr)
    \bigl(a_{-\mathbf k}-a_{\mathbf k}^\dagger\bigr)\\
&\qquad\qquad\qquad\qquad
    +\frac{\mathbf k\!\cdot\!\mathbf k}{2\omega_k}\bigl(a_{\mathbf k}+a_{-\mathbf k}^\dagger\bigr)
    \bigl(a_{-\mathbf k}+a_{\mathbf k}^\dagger\bigr)
\Bigg] \\[6pt]
&= \frac{1}{2}\int\frac{d^3k}{(2\pi)^3}\Bigg[
    \frac{m^2+k^2}{2\omega_k}\bigl(a_{\mathbf k}+a_{-\mathbf k}^\dagger\bigr)
    \bigl(a_{-\mathbf k}+a_{\mathbf k}^\dagger\bigr)\\
&\qquad\qquad\qquad\qquad
    +\frac{\omega_k}{2}\bigl(a_{\mathbf k}-a_{-\mathbf k}^\dagger\bigr)
    \bigl(a_{-\mathbf k}-a_{\mathbf k}^\dagger\bigr)
\Bigg] \\[6pt]
&= \frac{1}{2}\int\frac{d^3k}{(2\pi)^3}\frac{\omega_k}{2}\Big[
    \bigl(a_{\mathbf k}+a_{-\mathbf k}^\dagger\bigr)\bigl(a_{-\mathbf k}+a_{\mathbf k}^\dagger\bigr)
    +\bigl(a_{\mathbf k}-a_{-\mathbf k}^\dagger\bigr)\bigl(a_{-\mathbf k}-a_{\mathbf k}^\dagger\bigr)
\Big] \\[6pt]
&= \int\frac{d^3k}{(2\pi)^3}\,\omega_k\!\left(a_{\mathbf k}^\dagger a_{\mathbf k} + \tfrac12\right),
\end{align*}

Expanding the inside and collecting like-terms
\begin{align*}
    H &= \frac{1}{2} \int \frac{d^3k}{(2\pi)^3} \frac{\wk}{2}
    \left[ a_{\mathbf{k}}a_{\mathbf{k}}^{\dagger} +
    a_{\mathbf{-k}}^{\dagger}a_{\mathbf{-k}} \right]
\end{align*}
By doing the same trick as we had done above, where we consider the 
$a_{\mathbf{-k}}^{\dagger}a_{\mathbf{-k}}$ terms in their own integral, 
substitute $-\mathbf{k} \mapsto \mathbf{k}$, and using the negative sign 
acquired from $d^3k$ to correct the reversed integral bounds from doing this
substitution, we realize that the above is equivalent to

\begin{align*}
    H &= \frac{1}{2} \int \frac{d^3k}{(2\pi)^3} \frac{\wk}{2}
    \left(a_{\mathbf{k}}a_{\mathbf{k}}^{\dagger} + a_{\mathbf{k}}^{\dagger}a_{\mathbf{k}} \right)
\end{align*}
Using the definition of the commutator, we arrive at our desired result.
\[
\boxed{%
H=\int\frac{d^3k}{(2\pi)^3}\,\omega_k\!\left(a_{\mathbf{k}}^\dagger a_{\mathbf{k}}
+\tfrac{1}{2}[a_{\mathbf{k}},a_{\mathbf{k}}^\dagger]\right).}
\]

%TODO: Check factors of 2

\subsection*{b)}
We consider a real scalar field $\phi(x,t) \equiv \phi(x_1,x_2,t)$ defined on a
two--dimensional torus. The action is given in the usual form
\begin{equation}
    S = \int d^3x \, \mathcal{L}, \qquad 
    \mathcal{L} = \tfrac{1}{2}\left( \partial_\mu \phi \, \partial^\mu \phi 
    - m^2 \phi^2 \right),
\end{equation}
where $\mu = 0,1,2$, with $x^0 = t$, and $m$ is the mass of the scalar field.

The Euler--Lagrange equations for a field are
\begin{equation}
    \partial_\mu \left( \frac{\partial \mathcal{L}}
    {\partial (\partial_\mu \phi)} \right) 
    - \frac{\partial \mathcal{L}}{\partial \phi} = 0.
\end{equation}

We compute the relevant derivatives:
\begin{align}
    \frac{\partial \mathcal{L}}{\partial(\partial_\mu \phi)} 
        &= \partial^\mu \phi, \\
    \frac{\partial \mathcal{L}}{\partial \phi} 
        &= - m^2 \phi.
\end{align}

Hence the equation of motion becomes
\begin{equation}
    \partial_\mu \partial^\mu \phi + m^2 \phi = 0,
\end{equation}
or equivalently,
\begin{equation}
    (\Box + m^2)\phi = 0,
\end{equation}
which is the Klein--Gordon equation in $2+1$ dimensions.


Locally, the form of the equation of motion is \emph{identical} to the flat
Minkowski case. The distinction arises globally due to the topology of the
torus: the spatial coordinates $x_1, x_2$ are compact with periodic
boudary conditions.

\subsection*{c)}
If the torus radii are $L_1$ and $L_2$, then the only allowable wavevectors 
$\mathbf{k}$ are quantized as
\begin{equation}
   k_i = \frac{2\pi n_i}{L_i}, \qquad n_i \in \mathbb{Z}.
\end{equation}
where $i \in {1, 2}$. Thus we have a countably infinite set of wavevectors.
This already differs from the Minkowski space case where we had an uncountable
set of wavevectors. 

Consider the Klein--Gordon equation on the torus $T^2$:
\begin{equation}
    (\partial_t^2 - \nabla^2 + m^2)\,\phi(\mathbf{x},t) = 0.
\end{equation}
Assuming we can solve via separation of variables
\begin{equation}
    \phi(\mathbf{x},t) = f_{\mathbf{n}}(t)\,u_{\mathbf{n}}(\mathbf{x}),
\end{equation}
where the spatial mode functions are
\begin{equation}
    u_{\mathbf{n}}(\mathbf{x})
    = \frac{1}{\sqrt{V}} e^{i \mathbf{k}_{\mathbf{n}} \cdot \mathbf{x}},
    \qquad
    \mathbf{k}_{\mathbf{n}}
    = \left( \frac{2\pi n_1}{L_1}, \frac{2\pi n_2}{L_2} \right),
    \quad n_1,n_2 \in \mathbb{Z},
\end{equation}
with $V=L_1L_2$ the spatial volume.

Here, we adapt the form from the class but include a normalization factor of
$\frac{1}{\sqrt{V}}$. This factor comes about due to the requirement that our
solutions to the Klein-Gordan equation are orthonormal on $T^2$:
\[
\int_{T^2} d^2x\, u_{\mathbf n}(\mathbf x)^* u_{\mathbf m}(\mathbf x)
= \frac{1}{V}\int_{T^2} d^2x\, e^{i(\mathbf k_{\mathbf m}-\mathbf k_{\mathbf n})\cdot\mathbf x}
= \delta_{\mathbf n,\mathbf m}.
\]
Going forward, we know these modes obey
\begin{equation}
    -\nabla^2 u_{\mathbf{n}}(\mathbf{x})
    = |\mathbf{k}_{\mathbf{n}}|^2 u_{\mathbf{n}}(\mathbf{x}),
\end{equation}
Inserting the ansatz into the Klein--Gordon equation gives
\begin{equation}
    \bigl( \ddot f_{\mathbf{n}}(t)
    + \big(|\mathbf{k}_{\mathbf{n}}|^2 + m^2 \big) f_{\mathbf{n}}(t) \bigr)\,
    u_{\mathbf{n}}(\mathbf{x}) = 0.
\end{equation}
Because $u_{\mathbf{n}}(\mathbf{x})$ is nonzero, the time-dependent bit vanishes
\begin{equation}
    \ddot f_{\mathbf{n}}(t) + \omega_{\mathbf{n}}^2 f_{\mathbf{n}}(t) = 0,
    \qquad
    \omega_{\mathbf{n}}^2 = |\mathbf{k}_{\mathbf{n}}|^2 + m^2.
\end{equation}
This is precisely the harmonic oscillator equation for each mode. As we have
seen in class. This time, instead of a continuum of momenta, the field admits a
discrete mode
expansion,
\begin{equation}
    \phi(\mathbf{x},t) 
    = \sum_{n_1,n_2 \in \mathbb{Z}} 
    \frac{1}{\sqrt{2\omega_{\mathbf{n}} V}}
    \left( a_{\mathbf{n}} e^{i \mathbf{k}_{\mathbf{n}} \cdot \mathbf{x}}
    + a_{\mathbf{n}}^\dagger e^{- i \mathbf{k}_{\mathbf{n}} \cdot \mathbf{x}} \right),
\end{equation}
where
\begin{equation}
    \mathbf{k}_{\mathbf{n}} 
    = \left(\tfrac{2\pi n_1}{L_1}, \tfrac{2\pi n_2}{L_2}\right), 
    \qquad \omega_{\mathbf{n}} = \sqrt{|\mathbf{k}_{\mathbf{n}}|^2 + m^2}, 
    \qquad V = L_1 L_2.
\end{equation}
In summary, the toroidal topology does not alter the local Klein-Gordon
equation, but it enforces periodic boundary conditions, leading to a discrete
spectrum of allowed momentum modes, in contrast to the continuum of flat
Minkowski space. This discrete spectrum is still infinite, but \emph{less} 
infinite than the flat 2 + 1 dimentional universe :)

\subsection*{d)}
Other than a normalization factor and the fact that we must sum over a 
discrete set of wavevetors as opposed to integrate over a continuous
space of them, the operators look identical.

\subsection*{e)}
if the circle has radius \(r\), then
\[
x \sim x + 2\pi r,\qquad L\equiv 2\pi r
\]
is the circumference. The Klein--Gordon equation in \(1+1\) dimensions is
\[
(\partial_t^2-\partial_x^2 + m^2)\,\phi(x,t)=0.
\]

Periodic boundary conditions force the spatial eigenfunctions of \(-\partial_x^2\)
to be plane waves with discrete momenta. The allowed wavevectors are
\[
k_n=\frac{2\pi n}{L}=\frac{n}{r},\qquad n\in\mathbb Z,
\]
and the corresponding normalized spatial modes are
\[
u_n(x)=\frac{1}{\sqrt{L}}e^{i k_n x},
\]
which satisfy
\[
\int_0^L dx\, u_n(x)^* u_m(x)=\delta_{n,m},\qquad
\frac{1}{L}\sum_{n\in\mathbb Z} e^{i k_n(x-y)}=\delta_{S^1}(x-y).
\]
Seek \(\phi(x,t)=f_n(t)\,u_n(x)\). Using \(-\partial_x^2 u_n = k_n^2 u_n\) and
substituting into the KG equation gives for each \(n\)
\[
\ddot f_n(t) + \omega_n^2 f_n(t)=0,\qquad
\omega_n\equiv\sqrt{k_n^2+m^2}.
\]
Hence the general real classical solution may be written as

\[
\boxed{%
\phi(x)=\frac{1}{\sqrt{L}}\sum_{n\in\mathbb Z}\frac{1}{\sqrt{2\omega_n}}
\Big( a_n e^{i k_n x} + a_n^\dagger e^{-i k_n x} \Big),
}
\]

\section*{Question 3}
\subsection*{a)}

We consider a single real degree of freedom \(\phi(t)\) depending only on time.
The action is
\[
S=\int dt\,\mathcal{L}(t),\qquad
\mathcal{L}(t)=\tfrac12\big(\dot\phi(t)^2 - m^2\phi(t)^2\big),
\]
Now, the Euler--Lagrange equation
\[
\frac{d}{dt}\!\left(\frac{\partial\mathcal{L}}{\partial\dot\phi}\right)
- \frac{\partial\mathcal{L}}{\partial\phi}=0
\]
yields
\[
\ddot\phi(t) + m^2\phi(t)=0.
\]
Thus the field behaves as a single simple harmonic oscillator with
frequency \(\omega=m\).

\subsection*{b)}

The canonical momentum is
\[
\pi(t)=\frac{\partial\mathcal{L}}{\partial\dot\phi}=\dot\phi(t).
\]
In general $d+1$ dimensions, the Hamiltonian reads
\[
H = \int d^dx \,\left[\tfrac{1}{2}\pi^2 + \tfrac{1}{2}(\nabla\phi)^2 + \tfrac{1}{2} m^2 \phi^2\right].
\]
Since we are in $0+1$ dimensions, there is no spatial integral and no gradient term, so the Hamiltonian reduces to
\[
H = \tfrac{1}{2}\pi^2 + \tfrac{1}{2} m^2 \phi^2.
\]
To quantize, we promote $\phi,\pi$ to operators and impose the equal-time canonical commutator
\[
[\phi,\pi]=i \qquad (\hbar=1).
\]
Define annihilation and creation operators by
\[
a=\sqrt{\frac{m}{2}}\,\phi + i\frac{1}{\sqrt{2m}}\,\pi,\qquad
a^\dagger=\sqrt{\frac{m}{2}}\,\phi - i\frac{1}{\sqrt{2m}}\,\pi,
\]
which satisfy $[a,a^\dagger]=1$. Adding the two equations
\begin{align*}
a + a^\dagger 
&= \sqrt{\tfrac{m}{2}}\,\phi + i \tfrac{1}{\sqrt{2m}}\,\pi 
+ \sqrt{\tfrac{m}{2}}\,\phi - i \tfrac{1}{\sqrt{2m}}\,\pi \\
&= 2\sqrt{\tfrac{m}{2}}\,\phi.
\end{align*}
Hence
\[
\phi = \frac{1}{\sqrt{2m}} (a + a^\dagger).
\]
Subtracting the two equations
\begin{align*}
a - a^\dagger 
&= \sqrt{\tfrac{m}{2}}\,\phi + i \tfrac{1}{\sqrt{2m}}\,\pi 
- \Big( \sqrt{\tfrac{m}{2}}\,\phi - i \tfrac{1}{\sqrt{2m}}\,\pi \Big) \\
&= 2i \tfrac{1}{\sqrt{2m}}\,\pi.
\end{align*}
Thus
\[
\pi = -i \sqrt{\tfrac{m}{2}}\,(a - a^\dagger).
\]
The Hamiltonian is
\[
H = \tfrac{1}{2}\big(\pi^2 + m^2\phi^2\big).
\]
Now, we can write the Hamiltonian in terms of the $a$ and $a^{\dagger}$
operators by substituting the expressions for \(\phi\) and \(\pi\). First
compute each square:

\begin{align*}
\pi^2
&= \big(-i\sqrt{\tfrac{m}{2}}\big)^2 (a - a^\dagger)^2
= -\tfrac{m}{2}\,(a - a^\dagger)^2, \\[6pt]
m^2\phi^2
&= m^2 \cdot \tfrac{1}{2m}\,(a + a^\dagger)^2
= \tfrac{m}{2}\,(a + a^\dagger)^2.
\end{align*}

Thus
\[
\pi^2 + m^2\phi^2 = \tfrac{m}{2}\Big[-(a - a^\dagger)^2 + (a + a^\dagger)^2\Big].
\]

Expand the squares:
\begin{align*}
(a + a^\dagger)^2 &= a^2 + a a^\dagger + a^\dagger a + a^{\dagger 2},\\
(a - a^\dagger)^2 &= a^2 - a a^\dagger - a^\dagger a + a^{\dagger 2}.
\end{align*}

Hence
\begin{align*}
-(a - a^\dagger)^2 + (a + a^\dagger)^2
&= -\big(a^2 - a a^\dagger - a^\dagger a + a^{\dagger 2}\big)
   + \big(a^2 + a a^\dagger + a^\dagger a + a^{\dagger 2}\big)\\[4pt]
&= 2\,\big(a a^\dagger + a^\dagger a\big).
\end{align*}

So
\[
\pi^2 + m^2\phi^2 = \tfrac{m}{2}\cdot 2\big(a a^\dagger + a^\dagger a\big) = m\big(a a^\dagger + a^\dagger a\big).
\]

Therefore the Hamiltonian becomes
\[
H = \tfrac{1}{2}\big(\pi^2 + m^2\phi^2\big)
  = \tfrac{1}{2} m\big(a a^\dagger + a^\dagger a\big).
\]

Using the commutator \(a a^\dagger = a^\dagger a + [a,a^\dagger] = a^\dagger a + 1\), we obtain
\begin{align*}
H &= \tfrac{m}{2}\big(2 a^\dagger a + 1\big) \\
  &= m\Big(a^\dagger a + \tfrac{1}{2}\Big).
\end{align*}

\[
\boxed{\,H = m\Big(a^\dagger a + \tfrac{1}{2}\Big)\,}
\]
Thus the energy spectrum is the familiar
\[
E_n = m\Big(n+\tfrac{1}{2}\Big), \qquad n=0,1,2,\dots
\]
with the lowest energy (vacuum energy) being
\[
\boxed{E_0 = \tfrac{1}{2}m.}
\]
This is in sharp contrast to the infinite series expression for energy in our
toroidal and circular cases where we must add the energy of each allowable
wavevector.

\section*{Question 4}
So far, all of our examples have involved a mass $m$. 
Let us now take the massless limit $m \to 0$ and compare the three universes 
we have studied: toroidal ($T^2$), circular ($S^1$), and the $0+1$ dimensional 
quantum mechanics case.

\noindent In the toroidal case, the mode frequencies were
\[
\omega_{\mathbf n} = \sqrt{|\mathbf k_{\mathbf n}|^2 + m^2}.
\]
In the limit $m \to 0$, this reduces to
\[
\omega_{\mathbf n} = |\mathbf k_{\mathbf n}|,
\]
so the quantized field is still an infinite sum of harmonic oscillators, one for each
$\mathbf n \in \mathbb Z^2$. However, notice that the $\mathbf n=(0,0)$ mode has 
$\omega_{\mathbf 0}=0$. Unlike all others, it does not oscillate and instead
behaves as a free particle degree of freedom. This is a significant
qualitative difference from the massive case: the field decomposition is an
infinite tower of oscillators plus a free zero mode.

\noindent For the circular case, the mode frequencies were
\[
\omega_{n} = \sqrt{\left(\tfrac{n}{r}\right)^2 + m^2}.
\]
Taking $m \to 0$ gives
\[
\omega_{n} = \frac{|n|}{r}.
\]
Thus, as in the toroidal case, we have an infinite tower of harmonic oscillators
with discrete mode labels $n \in \mathbb Z$, but now the zero mode $n=0$ has 
$\omega_0 = 0$ and again becomes a free particle.

Now, for the zero spatial dimensions ($0+1$)
Here, there was only a single oscillator with frequency $\omega = m$. 
Sending $m \to 0$ eliminates the restoring force entirely:
\[
H = \tfrac{1}{2}\pi^2, \qquad [\phi,\pi]=i.
\]
This is the Hamiltonian of a free particle, not an oscillator. 
The spectrum is continuous, not discrete, and there is no notion of 
a lowest vacuum energy.

\section*{Question 5}
Starting with plugging the proposed $e^{-imt}\tilde{\phi}$ into the Klein-Gordon equation:

\begin{align*}
    (\delmu\delMu + m^2) (e^{-imt}\tilde{\phi}) &= 0 \\
    (\partial_t^2 - \nabla^2 + m^2)(e^{-imt}\tilde{\phi}) &= 0 \\
    (-m^2e^{-imt}\tilde{\phi} + -2ime^{-imt}\partial_t\tilde{\phi} + e^{-imt}\partial_t^2\tilde{\phi} - e^{-imt}\nabla^2\tilde{\phi} + m^2e^{-imt}\tilde{\phi}) &= 0 \\
    (-2im\partial_t\tilde{\phi} + \partial_t^2\tilde{\phi} - \nabla^2\tilde{\phi}) &= 0 
\end{align*}
\begin{equation}
    \boxed{\ddot{\tilde{\phi}} - 2im\ddot{\tilde{\phi}}  -\nabla^2\tilde{\phi} = 0}
\end{equation}

As desired. Next, consider a plane-wave ansatz for the field $\tilde{\phi} = e^{-i(Et - px)}$ with $\omega = m + E$. Plugging this into the dispersion relation $\omega^2 = m^2 + p^2$:
\begin{align*}
    (m + E)^2 &= m^2 + p^2 \\
    2mE + E^2 &= p^2
\end{align*}
In the nonrelativistic limit, $|\mathbf{p}| \ll m$ so $E \ll m$, then we can ignore the $E^2$ term. Thus $2mE \approx p^2 \rightarrow E \approx \frac{p^2}{2m}$ which, for the above, implies
\begin{equation}
    \boxed{i\frac{\partial\tilde{\phi}(x,t)}{\partial t} = -\frac{1}{2m}\nabla^2\tilde{\phi}(x,t)}
\end{equation}
Now, to justify whether the exclusion of the $(2\omega_k)^{-1/2}$ factor is consistent, let us recall that in the relevant limit
$$ \omega_k = \sqrt{m^2 +k^2}$$
Expanding this for small $k$:
\begin{align*}
    \omega_k &= m\sqrt{1 + k^2/m^2} \\
    &= m\left(1 + \frac{1}{2}\frac{k^2}{m^2} - \frac{1}{8}\frac{k^4}{m^4} + ...\right) \\
    &\approx m
\end{align*}
and so $w_k$ is dominated by $m$ in this limit. Therefore,
\begin{equation}
    \boxed{\phi(x) = \int \frac{d^3k}{(2\pi)^3}a_ke^{-ikx}}
\end{equation}
is consistent. Next, to verify that that
\begin{equation*}
    X = \int d^3x \;\textbf{x} \phi^{\dagger}(x)\phi(x)
\end{equation*}
is a sensible position operator, let us impose the standard commutation relations
\begin{equation*}
    \left[\phi(x), \phi^{\dagger}(y)\right] = \delta^3(x-y) \qquad \left[a_k, a_{k'}^{\dagger} \right] =(2\pi)^3\delta^3(k - k')
\end{equation*}
and use the operator on some state $\ket{y} = \phi^{\dagger}(y)\ket{0}$ as a sanity check. Hence,
\begin{align*}
    X\ket{y} &= \int d^3x \; x\phi^{\dagger}(x)\phi(x)\phi^{\dagger}(y)\ket{0} \\
    &= \int d^3x \; x \phi^{\dagger}(x)\left(\phi^{\dagger}(y)\phi(x) + [\phi(x), \phi^{\dagger}(y)]\right) \ket{0} \\
\end{align*}
Noticing that $\phi^{\dagger}(y)\phi(x)\ket{0} = 0$ because $a_k \in \phi(x)$, this simplifies to
\begin{align*}
    X\ket{y} &= \int d^3x \; x\phi^{\dagger}(x)\delta^3(x-y)\ket{0}\\
    X\ket{y} &= y\phi^{\dagger}(y)\ket{0}\\
\end{align*}
\begin{equation}
    \boxed{X\ket{y} = y\ket{y}}
\end{equation}
and thus $X$ is indeed a valid position operator. Now, defining a state $\ket{\psi} := \int d^3x \; \psi(x) \ket{x}$, let us observe how the position operator acts on $\ket{\psi}$:
\begin{align*}
    X\ket{\psi} &= \int d^3y \; y\phi^{\dagger}(y)\phi(y) \; \ket{\psi} \\
    &= \int d^3y \; y\phi^{\dagger}(y)\phi(y) \; \int d^3x \; \psi(x) \ket{x} \\
    &= \int d^3y \; y\phi^{\dagger}(y)\phi(y) \; \int d^3x \; \psi(x) \phi^{\dagger}(x)\ket{0} \\
    &= \int d^3y \, d^3x \; y\psi(x)\phi^{\dagger}(y)\underbrace{\phi(y)\phi^{\dagger}(x)}_{\cancel{\phi^{\dagger}(x)\phi(y)} + [\phi(y)\phi^{\dagger}(x)]}\ket{0} \\
    &= \int d^3y \, d^3x \; y\psi(x)\phi^{\dagger}(y)\delta^3(x-y)\ket{0}\\
    &= \int d^3x \; x\psi(x) \phi^{\dagger}(x)\ket{0}\\
\end{align*}
\begin{equation}
    \boxed{X\ket{\psi} = \int d^3x \; x\psi(x) \ket{x}}
\end{equation}
For the momentum operator, making the suggested substitutions, we have
\begin{equation*}
    P\ket{k} = \int \frac{d^3k}{(2\pi)^3} \;k a_ka^{\dagger}_k \ket{k}
\end{equation*}
Where $\ket{k}$ is similarly defined to be $\ket{k} = a^{\dagger}_k \ket{0}$. Letting $P$ act on our previous $\ket{\psi}$:

\begin{align*}
    P\ket{\psi} &= \int \frac{d^3k}{(2\pi)^3} \; k a_ka^{\dagger}_k \ket{\psi} \\
    &= \int \frac{d^3k}{(2\pi)^3} \;k a_ka^{\dagger}_k \int d^3x \; \psi(x) \underbrace{\phi^{\dagger}(x)\ket{0}}_{\int \frac{d^3q}{(2\pi)^3}e^{-iq\cdot k}\ket{q}}\\
    &= \int \frac{d^3k \, d^3x \, d^3q}{(2\pi)^6} ka^{\dagger}_k a_k \psi(x) e^{-iq \cdot x} a^{\dagger}_q \ket{0} \\
    &= \int \frac{d^3k \, d^3x \, d^3q}{(2\pi)^3} \psi(x) ka^{\dagger}_k \delta^3(k - q)e^{-iq \cdot x} \ket{0} \\
    &= \int \frac{d^3k \, d^3x}{(2\pi)^3} \psi(x) ka^{\dagger}_k e^{-ik \cdot x} \ket{0} \\
    &= \int \frac{d^3k \, d^3x}{(2\pi)^3} \psi(x) k e^{-ik \cdot x} \ket{k} \\
\end{align*}
Now express the momentum eigenstate \( \ket{k} \) in terms of position eigenstates:

\begin{align*}
\ket{k} &= \int d^3y \, e^{i k \cdot y} \ket{y}
\end{align*}

Substitute into the previous expression:

\begin{align*}
P\ket{\psi} &= \int \frac{d^3k \, d^3x \, d^3y}{(2\pi)^3} \, \psi(x) \, k \, e^{-i k \cdot x} \, e^{i k \cdot y} \ket{y} \\
&= \int d^3x \, d^3y \, \psi(x) \left[ \int \frac{d^3k}{(2\pi)^3} \, k \, e^{i k \cdot (y - x)} \right] \ket{y}
\end{align*}

The integral over \( k \) is a known Fourier identity:

\begin{align*}
\int \frac{d^3k}{(2\pi)^3} \, k \, e^{i k \cdot (y - x)} &= -i \nabla_y \delta^3(y - x)\\
\rightarrow P\ket{\psi} &= -i \int d^3x \, d^3y \, \psi(x) \, \nabla_y \delta^3(y - x) \ket{y}
\end{align*}
Now integrate by parts to move the derivative onto \( \psi(x) \):
\begin{align*}
P\ket{\psi} &= i \int d^3x \, \nabla \psi(x) \ket{x}
\end{align*}
And so finally:
\begin{align}
\boxed{P\ket{\psi} = \int d^3x \, (-i \nabla \psi(x)) \ket{x}}
\end{align}

Now, we want to compute the commutator \( [X_i, P_j] \) by acting on a general state \( \ket{\psi} \). That is,

\begin{align*}
[X_i, P_j] \ket{\psi} = X_i P_j \ket{\psi} - P_j X_i \ket{\psi}
\label{eq:commutator_definition}
\end{align*}
Let us evaluate each term separately. We start by using the result:
\[
P_j \ket{\psi} = \int d^3x \, (-i \partial_j \psi(x)) \ket{x}
\]
Then:
\begin{align*}
X_i P_j \ket{\psi} &= X_i \int d^3x \, (-i \partial_j \psi(x)) \ket{x} \nonumber \\
&= \int d^3x \, (-i \partial_j \psi(x)) X_i \ket{x} \nonumber \\
&= \int d^3x \, (-i \partial_j \psi(x)) x_i \ket{x}
\label{eq:first_term}
\end{align*}
Then, for the remaining term, we use:
\[
X_i \ket{\psi} = \int d^3x \, x_i \psi(x) \ket{x}
\]

Then:
\begin{align*}
P_j X_i \ket{\psi} &= P_j \int d^3x \, x_i \psi(x) \ket{x} \nonumber \\
&= \int d^3x \, x_i \psi(x) P_j \ket{x} \nonumber \\
&= \int d^3x \, x_i \psi(x) (-i \partial_j \delta^3(x - y)) \ket{x} \nonumber \\
&= \int d^3x \, \left( -i \partial_j (x_i \psi(x)) \right) \ket{x} \nonumber \\
&= \int d^3x \, \left( -i \delta_{ij} \psi(x) - i x_i \partial_j \psi(x) \right) \ket{x}
\label{eq:second_term}
\end{align*}
Putting it together:
\begin{align*}
[X_i, P_j] \ket{\psi} &= \int d^3x \left[ -i \partial_j \psi(x) x_i + i \delta_{ij} \psi(x) + i x_i \partial_j \psi(x) \right] \ket{x} \\
&= \int d^3x \, i \delta_{ij} \psi(x) \ket{x} \\
\end{align*}

\begin{equation}
    \boxed{[X_i, P_j] = i \delta_{ij}}
\end{equation}

Finally, to derive the Schrödinger equation in terms of $\psi(x)$, we begin by recalling that the nonrelativistic Hamiltonian is given by:
\begin{equation*}
    H = \frac{\mathbf{P}^2}{2m}
\end{equation*}
We act this operator on the state $\ket{\psi}$:
\begin{align*}
    H\ket{\psi} &= \frac{1}{2m} \mathbf{P}^2 \ket{\psi} \\
    &= \frac{1}{2m} \mathbf{P} \left( \int d^3x \, (-i \nabla \psi(x)) \ket{x} \right) \\
    &= \frac{1}{2m} \int d^3x \, (-i \nabla \psi(x)) \, \mathbf{P} \ket{x}
\end{align*}
But since $P_j \ket{x} = -i \partial_j \ket{x}$, then
\begin{align*}
    \mathbf{P} \ket{x} = -i \nabla \ket{x}
\end{align*}
So continuing:
\begin{align*}
    H\ket{\psi} &= \frac{1}{2m} \int d^3x \, (-i \nabla \psi(x)) (-i \nabla \ket{x}) \\
    &= \frac{1}{2m} \int d^3x \, (-i)(-i) (\nabla \psi(x)) \cdot (\nabla \ket{x}) \\
    &= \frac{1}{2m} \int d^3x \, \nabla \psi(x) \cdot \nabla \ket{x}
\end{align*}
Now apply integration by parts, or recognize the result directly:
\begin{align*}
    H\ket{\psi} &= \int d^3x \, \left( -\frac{1}{2m} \nabla^2 \psi(x) \right) \ket{x}
\end{align*}
Thus, using the definition $H\ket{\psi} = i \partial_t \ket{\psi}$:
\begin{align*}
    i \partial_t \ket{\psi} &= \int d^3x \, \left( -\frac{1}{2m} \nabla^2 \psi(x) \right) \ket{x}
\end{align*}
Matching both sides in the position basis:
\begin{equation*}
    \boxed{i \frac{\partial \psi(x,t)}{\partial t} = -\frac{1}{2m} \nabla^2 \psi(x,t)}
\end{equation*}

\end{document}

