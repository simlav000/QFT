\documentclass[12pt]{article}

% Packages
\usepackage[margin=1in]{geometry}
\usepackage{fancyhdr}
\usepackage{titlesec}
\usepackage{amsmath, amssymb, bm}

\usepackage{mathtools}
\newcommand{\w}{\omega}
\newcommand{\phip}{\phi^{'}}
\newcommand{\phipp}{\phi^{''}}
\DeclarePairedDelimiter\bra{\langle}{\rvert}
\DeclarePairedDelimiter\ket{\lvert}{\rangle}
\DeclarePairedDelimiterX\braket[2]{\langle}{\rangle}{#1\,\delimsize\vert\,\mathopen{}#2}
\usepackage{braket}

% Header & Footer
\pagestyle{fancy}
\fancyhf{}
\fancyhead[L]{Name:  \underline{Simon Lavoie}}
\fancyhead[C]{Class: \underline{PHYS 610}}
\fancyhead[R]{Student ID: \underline{261 051 325}}
\fancyfoot[C]{\thepage}

% Title format for neatness
\titleformat{\section}{\large\bfseries}{\thesection.}{0.5em}{}
\titleformat{\subsection}{\normalsize\bfseries}{\thesubsection)}{0.5em}{}

% Document
\begin{document}

\begin{center}
    \Large \textbf{Assignment \#1}
\end{center}
\vspace{1cm}

\section*{Question 1}
\subsection*{a)}
We will start with the four-momentum of the original electron, denoted
$p^{\mu}_i = (E_i, \bm{p}_i)$. Using the mostly-minus metric 
$\eta^{\mu\nu} = (+, -, -, -)$, we have that 
$p_i^2 = E_i^2 - |\bm{p}_i|^2 = m_e^2$.
Meanwhile, we define the four- momenta of the electron in the final state and
the emitted photon as $p^{\mu}_f = (E_f, \bm{p}_f)$ and $q^{\mu} = (\omega,
\bm{q})$ respectively. For the photon, we made use of $E = \hbar \omega$,
where, in natural units, $\hbar = c = 1$. Now let us try to conserve both
energy and momentum in this reaction in one move via four-momentum
conservation.
$$ p_i^{\mu} = p_f^{\mu} + q^{\mu} $$
Now let us square this equation
\begin{align}
    p^2_i = m_e^2 &= (p_f^{\mu} + q^{\mu})^2 \\
         &= p_f^2 + 2(p_f \cdot q) + q^2 \\
         &= m_e^2 + 2(p_f \cdot q)
\end{align}
And cancelling the squared electron mass term on both sides we obtain 
$p_f \cdot q = 0$, where $p_f \cdot q = E_f\omega - \bm{p_f}\cdot\bm{q}$, therefore
$E_f\omega = \bm{p_f}\cdot\bm{q}$. Notice, however, that 
$$\bm{p_f}\cdot\bm{q} \leq |\bm{p_f}||\bm{q}| = |\bm{p_f}|\omega$$
$$E_f\omega \leq |\bm{p_f}|\omega$$
$$E_f \leq |\bm{p_f}|$$
$$E_f^2 \leq |\bm{p_f}|^2$$
But, by the invariant, we require $E_f^2 - |\bm{p}|^2 = m_e^2 > 0$ and thus 
$$E_f^2 > |\bm{p_f}|^2$$
Which is clearly a contradiction. Thus, we cannot have both momentum and energy
conservation in this instance.

\subsection*{b)}
If the photon were to pick up a nonzero mass $\mu > 0$, then the electric
potential gains an exponential term, having a Yukawa character $V(r) =
-\frac{kq^2}{r}e^{-\mu r}$, where $k = \frac{1}{4\pi\epsilon_0}$. If we expand
this exponential via $$ e^{-\mu r} = 1 - \mu r + \frac{\mu^2 r^2}{2} + ...$$
we see that at the leading order
$$ V(r) = -\frac{kq^2}{r} + kq^2\mu $$
Now, the Schrodinger equation for the system becomes (substituting the electric
charge $e$ in place of $q$)
$$ \frac{\nabla^2}{2m}\psi - \frac{ke^2}{r}\psi + kq^2\mu\psi = E\psi $$
$$ \rightarrow  \frac{\nabla^2}{2m}\psi - \frac{ke^2}{r}\psi = (E - ke^2\mu)\psi$$ 
Which is equivalent to a system with original potential $V(r) = -\frac{kq^2}{r}$
but energy $E - \frac{e^2\mu}{4\pi\epsilon_0}$. Thus hydrogen would have a lower
ionization energy.

\subsection*{c)}
A particle of mass $m$ decaying from rest must emit the electron-positron pair
back-to-back to conserve momentum, meaning $p_+ = p_- = p$, and since both
electrons and positrons have equal mass, it must impoart its rest energy $E_i =
mc^2$ equally among them. Thus $$E_+ = E_- = \frac{1}{2}E_i$$
Meanwhile, the total energy of the two products is given by
$$ E_f = E_+ + E_- = 2\sqrt{m_ec^4 + p^2c^2} $$
By energy conservation
$$mc^2 = 2\sqrt{m_e^2c^4 + p^2c^2}$$
Isolating for $p$
\begin{align}
    4p^2c^2 &= m^2c^4 - 4m_e^2c^4 \\
    p       &= c\sqrt{\frac{m^2}{4} - m_e^2}
\end{align}
Now we can relate the momentum to the particle velocity via $p = mv$. Then, the
magnetic force is $\bm{F} = q\bm{v} \times \bm{B}$ and since $\bm{B}$ is 
perpendicular to the plane of the decay, the magnitude is $F = evB$. Since this
force is centripetal, we may write $evB = mv^2/r$. Isolating for $r$ in terms
of known quantities
\begin{align}
    r &= \frac{mv}{eB} \\
      &= \frac{m}{eB}\frac{p}{m} \\
\end{align}
and substituting $p$ from (5):
\begin{equation}
    \boxed{r = \frac{c\sqrt{\frac{m^2}{4} - m_e^2}}{eB}}
\end{equation}
As desired.


\subsection*{d)}
The four-momenta of the particles are 
\begin{align}
    p_x^{\mu} &= \left(\tfrac{E_x}{c}, \bm{p_x}\right) \\
    p_y^{\mu} &= \left(\tfrac{E_y}{c}, 0\right) = \left(\tfrac{m_yc^2}{c}, 0\right) = (m_yc, 0) \\
    p_z^{\mu} &= \left(\tfrac{E_z}{c}, \bm{p_z}\right)
\end{align}

By conservation of four-momentum
\begin{align}
    p_x^{\mu} + p_y^{\mu} &= p_z^{\mu} \\
    (p_x^{\mu} + p_y^{\mu})^2 &= (p_z^{\mu})^2 \\
    m_x^2c^2 + 2\, p_y \cdot p_x + m_y^2c^2 &= m_z^2c^2 \\
    p_y \cdot p_x &= \frac{(m_z^2 - m_x^2 - m_y^2)c^2}{2}
\end{align}

Now, by explicit evaluation of the dot product (since $p_y^\mu = (m_y c,0)$):
\begin{align}
    p_y \cdot p_x &= \frac{m_ycE_x}{c} = m_y E_x
\end{align}

Thus
\begin{align}
    m_y E_x &= \frac{(m_z^2 - m_x^2 - m_y^2)c^2}{2}
\end{align}

By energy conservation,
\begin{align}
    E_z &= E_x + m_yc^2,
\end{align}
so
\begin{align}
    m_y(E_z - m_yc^2) &= \frac{(m_z^2 - m_x^2 - m_y^2)c^2}{2} \\
    E_z - m_yc^2 &= \frac{(m_z^2 - m_x^2 - m_y^2)c^2}{2m_y} \\ 
    E_z  &= \frac{(m_z^2 - m_x^2 - m_y^2)c^2}{2m_y} + m_yc^2 \\ 
    E_z  &= \frac{(m_z^2 - m_x^2 - m_y^2)c^2}{2m_y} + \frac{2m_y^2}{2m_y}c^2 \\ 
    E_z  &= \frac{(m_z^2 - m_x^2 + m_y^2)c^2}{2m_y}
\end{align}
As desired. Meanwhile, in the head-on collision, the initial momenta, which are 
equal and opposite in magnitude add to zero. This means particle z also has zero 
momentum and thus
$$E_z = m_zc^2$$


\section*{Question 2}
Between each pair of plates, we have an infinite tower of standing waves due to
the fact that the wavefunctions must vanish at the boundaries. This permits
only wavefunctions of the form $\psi_n(x) = \sin\left(\frac{n\pi x}{a}\right)$
where $a$ is $d$ in region AB and $L - d$ in region BC. This corresponds to
wavenumbers $k_n = \frac{n\pi}{a}$. Now, for a massless field such as this, 
we have $E_n = p_nc = \hbar k_n c$ and since $E_n = \hbar \omega_n$, then the
allowable angular frequencies are
$$ \omega_n = \frac{n\pi c}{a}$$

Since each mode of the field is a quantum harmonic oscillator, the vacuum
(zero-point) contribution of each mode is

$$E_n=\frac{1}{2}\hbar \omega_n = \frac{1}{2}\frac{\hbar n \pi c}{a}$$
Thus, taking the sum of these zero-point energies for each individual harmonic
oscillator, making sure to substitute the proper expressions for $a$ when 
evaluating the energies in regions AB and BC respectively, we obtain
\begin{align}
    E &= \sum_{n=1}^\infty \frac{1}{2}\frac{\hbar n \pi c}{d}
+ \sum_{m=1}^\infty \frac{1}{2}\frac{\hbar m \pi c}{L - d} \\
   E &= \frac{\pi c \hbar}{2} \left(\sum_{n=1}^\infty \frac{n}{d}
       + \sum_{m=1}^\infty \frac{m}{L-d}\right)
\end{align}
As desired.

Now, to perform the regularization, let us substitute 
$n \rightarrow n e^{-\frac{an\pi}{d}}$ and $m \rightarrow m
e^{-\frac{am\pi}{L-d}$ into the above result
$$ E = \frac{\pi c \hbar}{2} \left(\sum_{n=1}^\infty
\frac{ne^{-\frac{an\pi}{d}}}{d} + \sum_{m=1}^\infty
\frac{me^{-\frac{am\pi}{L-d}}}{L-d}\right) $$
Let us evaluate these sums one at a time, starting with
\begin{align}
    E_1&= \frac{\pi c \hbar}{2d} \sum_{n=1}^\infty n e^{-\frac{an\pi}{d}} \\
       &= -\frac{c \hbar}{2} \frac{\partial}{\partial a}
       \left(\sum_{n=1}^\infty e^{-\frac{an\pi}{d}}\right) \\
       &= -\frac{c \hbar}{2} \frac{\partial}{\partial a}
       \left(\frac{1}{1 - e^{-\frac{a\pi}{d}}}\right) \\
       &= -\frac{c \hbar}{2} \left(-\frac{\pi}{d}\right)
       \frac{e^{-\frac{a\pi}{d}}}{\left(1 - e^{-\frac{a\pi}{d}}\right)^2} \\
       &= \frac{c\hbar\pi}{2d} \frac{e^{\frac{a\pi}{d}}}
           {\left(e^{\frac{a\pi}{d}} - 1 \right)^2}
\end{align}

The equivalent calulation for the BC region gives
\begin{align &= \frac{\pi c \hbar}{2(L-d)}\sum_{m=1}^\infty me^{-am\pi/(L-d)} \\
        &= -\frac{c \hbar}{2}\frac{\partial}{\partial a}
    \sum_{m=1}^\infty e^{-am\pi/(L-d)} \\
        &= \frac{c \hbar \pi}{2(L-d)}
        \frac{e^{a\pi/(L-d)}}{\left(e^{a\pi/(L-d)} - 1\right)^2} \\
\end{align}
In both instances, we proceed by taylor expanding $\frac{e^x}{(e^x - 1)^2}$
in $x$
\begin{align}
    \frac{e^x}{(e^x -1)^2} &= \frac{\left(1 + x + \frac{x^2}{2!} +
    \frac{x^3}{3!} +...\right)}{\left(x + \frac{x^2}{2!} + \frac{x^3}{3!} + ...
\right)^2} \\
    &= \frac{1}{x^2}\frac{\left(1 + x + \frac{x^2}{2!} +\frac{x^3}{3!} +...\right)}
    {\left(1 + \frac{x}{2!} + \frac{x^2}{3!} + ... \right)^2}
\end{align}
We can use the binomial theorem $(1 + u)^{\alpha} = 1 + \alpha u +
\frac{\alpha(\alpha - 1)}{2}u^2 + ...$ to simplify the denominator.
Specifically, letting $u = \frac{x}{2} + \frac{x^2}{6}$ and $\alpha = -2$
\begin{align}
    \left(1 + \frac{x}{2!} + \frac{x^2}{3!} \right)^{-2} &= 
1 - x + -\frac{x^2}{3} + \frac{3x^2}{4} + \mathcal{O}(x^3) \\
    &\approx 1 - x + \frac{5x^2}{12}
\end{align}
And putting this result back into the above
\begin{align}
    \frac{e^x}{(e^x - 1)^2} &= \frac{1}{x^2}\left(1 + x + \frac{x^2}{2} +
        \frac{x^3}{6} + ...\right) \left(1 - x + \frac{5x^2}{12} \right) \\
    &= \frac{1}{x^2} \left(1 - \frac{1}{12}x^2 + \mathcal{O}(x^3) \right) \\
    &= \frac{1}{x^2} - \frac{1}{12} + \mathcal{O}(x^3)
\end{align}
Using this, and substituting $x = \frac{a\pi}{d}$ in the AB region instance
and $x = \frac{a\pi}{(L-d)}$ in the BC region instance, we get a total energy
of
\begin{align}
    E_1 + E_2 &= \frac{\pi c \hbar}{2d} \left(\frac{d^2}{a^2\pi^2} -
    \frac{1}{12} + \mathcal{O}(a^3)\right) + \frac{c\hbar\pi}{2(L-d)}
    \left(\frac{(L-d)^2}{a^2\pi^2} - \frac{1}{12} + \mathcal{O}(a^3) \right) \\
    &= \hbar c \left(\frac{d}{2\pi a^2} - \frac{\pi}{24d} + \frac{(L-d)}{2\pi
    a^2} - \frac{\pi}{24(L-d)} + \mathcal{O}(a^3) \right) \\
    &= \hbar c \left(\frac{L}{2\pi a^2} - \frac{\pi}{24d} - \frac{\pi}{24(L-d)}
    + \mathcal{O}(a^3) \right)
\end{align}
Now, we can finally evaluate the force by taking the partial derivative of this
expression with respect to $d$. Remarkably, the divergent first term vanishes 
here, meaning we have a finite force even in the $a \rightarrow 0$ limit!
As such
\begin{align}
    F = -\frac{\partial E}{\partial d} &= - \hbar c \left(\frac{\pi}{24d^2} 
        - \frac{\pi}{24(L - d)^2} + ...\right)
\end{align}
But in the $L \gg d$ limit, this second term also vanishes. Thus, we finally
obtain the desired result of
\begin{equation}
    \boxed{F \approx -\frac{\pi c \hbar}{24d^2}}
\end{equation}


\section*{Question 3}
We have that the amplitude of finding the particle propagating from $x$ to $y$
is $\mathcal{A} = \braket{y|e^{-iHT}|x}$. If we divide this distance between
$x$ and $y$ into $N$ points labelled $(x, x_1, x_2, ..., X_{N-1}, y)$ into 
equal intervals of time $\delta t = T/N$, then we can break up the exponential
in the inner product as
$$\mathcal{A} = \braket{y|e^{-iH\delta t}e^{-iH\delta t}...e^{-iH\delta t}|x}$$
Which contains exactly $N$ such exponential factors. We may insert an identity
operator $\mathbb{I} = \int dx \ket{x}\bra{x}$ between each such factor
\begin{align}
    \mathcal{A} &= \braket{y|e^{-iH\delta t}\mathbb{I}e^{-iH\delta
    t}\mathbb{I}...e^{-iH\delta t}|x} \\
&= \boxed{\left( \prod_{j=1}^N \int dx_j \right) \braket{y|e^{-iH\delta t}|x_{N-1}}
\braket{x_{N-1}|e^{-iH\delta t}|X_{N-2}}...\braket{x_1|e^{-iH\delta t}|x}}
\end{align}
As desired. Now, taking an individual factor $f = \braket{x_{j+1}|e^{-iH\delta t}|x_j}$
let us compute this by taking $H = \tfrac{\bm{p}^2}{2m}$ where $\bm{p}$ is an
operator. Plugging this Hamiltonian in our expression for the factor
$$ f = \braket{x_{j+1}|e^{-i\bm{p}^2\delta t/2m}|x_j}$$
Inserting the identity $\mathbb{I} = \int \frac{dp}{2\pi} \ket{p}\bra{p}$ in
the inner product and noticing that $e^{-i\bm{p}^2}\ket{p} = e^{-ip^2}\ket{p}$

\begin{align}
    &= \frac{1}{2\pi} \int dp \bra{x_{j+1}} e^{-ip^2\delta t/2m}\ket{p}\braket{p|x_j} \\
    &= \frac{1}{2\pi} \int dp  e^{-i\bm{p}^2\delta t/2m}\braket{x_{j+1}|p}\braket{p|x_j} \\
\end{align}
Now, note that $\braket{x_{j+1}|p} = e^{ipx_{j+1}}$ and 
$\braket{p|x_j} = \left(\braket{x_j|p}\right)^* = e^{-ipx_j}$ 
such that
\begin{align}
    f &= \frac{1}{2\pi} \int dp e^{-ip^2\delta t/2m} e^{ip(x_{j+1} - x_j)} \\
      &= \frac{1}{2\pi} \int dp e^{-ap^2 + bp}
\end{align}
where $a = \frac{i\delta t}{2m}$ and $b = i(x_{j+1} - x_j)$. Completing the
square, $-ap^2 + bp = -a\left(p - \frac{b}{2a}\right)^2 + \frac{b^2}{4a}$ such
that
$$ f = \frac{e^{b^2/4a}}{2\pi} \int dp e^{-a(p - b/2a)^2} $$
Which is a standard Gaussian integral which we evaluate over all space. Doing 
this and substituting our values for $a$ and $b$:
\begin{align}
    f &= \frac{e^{b^2/4a}}{2\pi} \sqrt{\frac{\pi}{a}}\\
      &= \boxed{\sqrt{\frac{-im}{2\pi\delta t}}\exp\left[im\frac{(x_{j+1} - x_j)^2}{2\delta
      t}\right]}
\end{align}
as desired, at least up to a factor of $2\pi$.
Now, putting the given result in for every such factor
\begin{align}
    \mathcal{A} &= \left(\frac{-2\pi i m}{\delta t}\right)^{N/2} \prod_{j=0}^N
    \int dx_j\exp\left[\frac{im}{2}\frac{(x_{j+1} - x_j)^2}{\delta t}\right] \\
    &= \left(\frac{-2\pi i m}{\delta t}\right)^{N/2} \int
    dxdx_{N-1}dx_{N-2}...dx_1dy\exp\left[\delta t\sum_{j=0}^N
        \frac{im}{2}\left(\frac{x_{j+1} - x_j}{\delta t}\right)^2\right] \\
\end{align}
Making the substitutions $\frac{x_{j+1} - x_j}{\delta t} \rightarrow \dot{x}$
and $\delta t \sum_{j=0}^{N} \rightarrow \int_0^T dt$ in the
$\delta t \rightarrow 0$ and $N \rightarrow \infty$ continuum limit:
$$ \mathcal{A} = \left(\frac{-2\pi i m}{\delta t}\right)^{N/2} \int
dxdx_{N-1}dx_{N-2}...dx_1dy\exp\left[i \int_0^T dt \frac{1}{2}
m\dot{x}^2\right]$$
And using $\left(\frac{-2\pi i m}{\delta t}\right)^{N/2} \prod_{j=0}^{N-1}
\int dx_j = \int \mathcal{D}x$ in this limit we obtain
\begin{equation}
    \boxed{\mathcal{A} = \int \mathcal{D}x e^{i\int_0^T dt
    \frac{1}{2}m\dot{x}^2}}
\end{equation}
As desired.



\section*{Question 4}
\subsection*{a)}
Starting with the action
\begin{align}
    S &= \int dx \left[\frac{1}{2}\dot{\phi}^2 - \frac{1}{2}
    (\partial_x\phi)^2 - \frac{\lambda}{4}\left(\phi^2 - \frac{m^2}{\lambda}
    \right)^2\right] = \int \mathcal{L} dx \\
      &= \int T - V dx\\
\end{align}
Such that $V[\phi] = \int dx \frac{1}{2}(\partial_x\phi)^2 + \frac{\lambda}{4}
\left(\phi^2 - \frac{m^2}{\lambda}\right)^2$. Now, the Euler-Lagrange equation
tells us 
$$\frac{\delta S}{\delta \phi} = \partial_{\mu}\left(\frac{\partial
\mathcal{L}}{\partial(\partial_{\mu}\phi)}\right) - \frac{\mathcal{L}}{\partial
\phi} = 0$$
We can break up the first term into its temporal part
$\dfrac{\partial\mathcal{L}}{\partial(\partial_0\phi)}=\partial_0\phi$
such that
$$\partial_0\big(\dfrac{\partial\mathcal{L}}{\partial(\partial_0\phi)}\big)=\partial_0^2\phi$$
and then its spatial part
$\dfrac{\partial\mathcal{L}}{\partial(\partial_x\phi)}=-\partial_x\phi$
such that
$$\partial_x\big(\dfrac{\partial\mathcal{L}}{\partial(\partial_x\phi)}\big)=-\partial_x^2\phi$$
and thus in sum we have $\partial_0^2\phi-\partial_x^2\phi$ from the first term.
Meanwhile, for the second term, we have 
\begin{align}
    \frac{\partial \mathcal{L}}{\partial \phi} &= -\frac{\lambda}{2} \left(\phi^2
    - \frac{m^2}{\lambda}\right)2\phi \\
    &= -\lambda\phi(\phi^2 - m^2/\lambda)
\end{align}
Therefore, the full EOM is
$$ \partial_0^2\phi - \partial_x^2\phi + \lambda\phi(\phi^2 - m^2/\lambda) = 0 $$
Now, for the trivial solution, the EOM reduces to
\begin{align}
    \lambda\phi_0(\phi_0^2 - m^2\lambda) &= 0 \\
    \boxed{\phi_0 = \pm \frac{m}{\sqrt{\lambda}}}
\end{align}
Then, for the space-dependent solution, our EOM becomes an ODE of the form
\begin{align}
    \phi^{''}(x) &= \lambda\phi(x)(\phi^2(x) - m^2/\lambda)\\
    \phi^{'}(x)\phi^{''}(x) &= \phi^{'}(x)\lambda\phi(x)(\phi^2(x) - m^2/\lambda) \\
\end{align}
We can write both sides of this equation in terms of a derivative like so
\begin{align}
    \frac{d}{dx}\left( \frac{1}{2}(\phi^{'}(x))^2 \right) &= \frac{dV}{d\phi}\phi^{'}(x) \\
                                                          &= \frac{dV}{d\phi}\frac{d\phi}{dx} \\
                                                          &= \frac{d}{dx}V(\phi(x))
\end{align}
And integrating w.r.t. $x$
$$ \frac{1}{2}(\phi^{'}(x))^2 - V(\phi) = C $$
We can determine the constant of integration $C$ via the argument that for a 
finite-energy solution, we require the field to approach the trivial vacuum
solutions $\phi_0 = \pm \frac{m}{\sqrt{\lambda}}$ as $x\to\pm\infty$. Through this, 
we have that $V(\pm \frac{m}{\sqrt{\lambda}}) =
\tfrac{\lambda}{4}((\pm\frac{m}{\sqrt{\lambda}})^2-(\pm\frac{m}{\sqrt{\lambda}})^2)^2 = 0$,
and similarly $\phi^{'}(x)\to 0$ because the slope is constant at infinity.
Therefore, $\frac{1}{2}(\phi^{'}(\infty))^2 - V(\phi(\infty)) = 0 - 0 = 0 = C$.
Thus, defining $\frac{m}{\sqrt{\lambda}} = \nu$ for simplicity we may continue
with
\begin{align}
    \frac{1}{2}(\phi^{'}(x))^2 &= V(\phi) = \frac{\lambda}{4}(\phi^2 - \nu^2)^2 \\
    \phi^{'}(x) &= \pm \sqrt{\frac{\lambda}{2}}(\phi^2 - \nu^2) \\
    \frac{d\phi}{dx} &= \pm \sqrt{\frac{\lambda}{2}}(\phi^2 - \nu^2) \\
\end{align}
Which is a separable ODE which evaluates to
\begin{align}
    \ln\left|\frac{\nu + \phi}{\nu - \phi}\right| &=
    2\nu\sqrt{\frac{\lambda}{2}}x + C_2 \\
    \ln\left|\frac{\nu + \phi}{\nu - \phi}\right| &=
    \sqrt{2}mx + C_2 \\
    \frac{\nu + \phi}{\nu - \phi} &= C e^{\sqrt{2}mx} \\
\end{align}
Solving the above for $\phi(x)$, we obtain
\begin{align}
    \phi(x) = \nu \frac{Ce^{\sqrt{2}mx} - 1}{Ce^{\sqrt{2}mx} + 1} \\
\end{align}
Then, writing $C = e^{-\sqrt{2}mx_0}$ for some other constant $x_0$, we
get a cleaner result, namely
\begin{align}
    \phi(x) = \nu \frac{e^{\sqrt{2}m(x - x_0)} - 1}{e^{\sqrt{2}m(x - x_0)} + 1} \\
\end{align}
Recognizing this expression as being similar to the hyperbolic tangent identity
$$ \tanh(z) = \frac{e^{2z} - 1}{e^{2z} + 1} $$
we finally obtain our non-trivial space-dependent background solutions
$$ \boxed{\phi_{\text{cl}}(x) = \pm \frac{m}{\sqrt{\lambda}}\tanh\left(\frac{m(x - x_0)}{\sqrt{2}}\right)}$$

\subsection*{b)}
Starting with the action
\begin{align}
    S &= \int \frac{d\w d^3q}{(2\pi)^4} dr \left[A(r) \phi(r, -\w)\phipp(r, \w)
    + B(r)\phip(r, -\w)\phip(r, \w) \\
      &+ C(r)\phi(r, -\w)\phip(r, \w)
  + D(r)\phi(r, -\w)\phi(r, \w) \right]
\end{align}
We begin by varying the action
\begin{align}
    \delta S = \int\!\frac{d\omega\,d^3q}{(2\pi)^4} dr\;\Big\{
A(r)\big[\delta\phi(r,-\omega)\,\phi''(r,\omega)+\phi(r,-\omega)\,\delta\phi''(r,\omega)\big]\\
+ B(r)\big[\delta\phi'(r,-\omega)\,\phi'(r,\omega)+\phi'(r,-\omega)\,\delta\phi'(r,\omega)\big]\\
+ C(r)\big[\delta\phi(r,-\omega)\,\phi'(r,\omega)+\phi(r,-\omega)\,\delta\phi'(r,\omega)\big]\\
+ D(r)\big[\delta\phi(r,-\omega)\,\phi(r,\omega)+\phi(r,-\omega)\,\delta\phi(r,\omega)\big]\Big\}.
\end{align}
Now, we can look at terms containing $\delta\phi(r, \w)$ and its derivatives
\begin{align}
    \int & dr\;\Big[\,A(r)\,\phi(r,-\omega)\,\delta\phi''(r,\omega)
+\big(B(r)\,\phi'(r,-\omega)+C(r)\,\phi(r,-\omega)\big)\,\delta\phi'(r,\omega)\\
&+ D(r)\,\phi(r,-\omega)\,\delta\phi(r,\omega)\Big].
\end{align}
Now, let's do integration by parts on the first term
\begin{align}
    \int dr A\phi(-\w)\delta\phipp &=
    \Big[A\phi(-\w)\delta\phip\Big]_{r_{in}}^{r_{out}} - \int dr
    (A\phi(-\w))'\delta\phip
\end{align}
Putting this back into the variation we obtain
\begin{align}
    \delta S|_{\delta\phi(\omega)} &= \int\!\frac{d\omega\,d^3q}{(2\pi)^4}\Big\{
\Big[A\phi(-\omega)\,\delta\phi'\Big]_{r_{\rm in}}^{r_{\rm out}}\\
                                   &+ \int dr\;\Big[-\big(A\phi(-\omega)\big)'\,\delta\phi' + (B\phi'(-\omega)+C\phi(-\omega))\,\delta\phi' 
+ D\phi(-\omega)\,\delta\phi\Big]\Big\}.
\end{align}
For convenience, let us define
$$M(r) = -\big(A\phi(-\omega)\big)' + B\phi'(-\omega) + C\phi(-\omega)$$
such that the integral over $r$ becomes
$$ \int dr\;M(r)\,\delta\phi' + \int dr\;D\phi(-\omega)\,\delta\phi $$
Let's use the same trick and integrate by parts again
$$ \int dr\;M(r)\,\delta\phi' = \Big[ M(r)\,\delta\phi\Big]_{r_{\rm
in}}^{r_{\rm out}} - \int dr\;M'(r)\,\delta\phi$$
Where
$$ M'(r) = -\big(A\phi(-\omega)\big)'' + \big(B\phi'(-\omega)\big)' + \big(C\phi(-\omega)\big)'$$
Such that the variation on the action now reads

\begin{align}
    \delta S|_{\delta\phi(\omega)} &= \int\!\frac{d\omega\,d^3q}{(2\pi)^4}\Big\{
\Big[A\phi(-\omega)\,\delta\phi'\Big]_{r_{\rm in}}^{r_{\rm out}}
+ \Big[\Big(-(A\phi(-\w))' + B\phi'(-\w) + C\phi(-\w)\Big)\delta\phi\Big]_{r_{\rm in}}^{r_{\rm out}} \\
                                   &+ \int dr \Big(-\big(A\phi(-\omega)\big)''
                                   + \big(B\phi'(-\omega)\big)' +
                               \big(C\phi(-\omega)\big)' + D\phi(-\w)\Big)\delta\phi
\end{align}
The EOM is given by setting $\delta S = 0$, which implies that, in the bulk 
integrand
$$ (A(r)\,\phi(r,-\omega))'' - (B(r)\,\phi'(r,-\omega))' -
(C(r)\,\phi(r,-\omega))' + D(r)\,\phi(r,-\omega)=0$$
Which we'll write simply as
$$(A\phi)''-(B\phi')'-(C\phi)'+D\phi=0.$$
Expanding these terms individually
\begin{align}
    (A\phi)'' &= A''\phi + 2A'\phi' + A\phi'',\\
    (B\phi')' &= B'\phi' + B\phi'',\\
    (C\phi)' &= C'\phi + C\phi'.
\end{align}
Substituting into the EOM
$$\big(A''\phi + 2A'\phi' + A\phi''\big)
- \big(B'\phi' + B\phi''\big)
- \big(C'\phi + C\phi'\big)
+ D\phi = 0.$$
Collecting terms
$$(A - B)\,\phi'' + (2A' - B' - C)\,\phi' + (A'' - C' + D)\,\phi = 0.$$
Finally, we obtain
$$\phi'' + \frac{2A' - B' - C}{A - B}\,\phi' + \frac{A'' - C' + D}{A - B}\,\phi = 0.$$
Meanwhile, going back to the boundary terms obtained via integration by parts
% 1) boundary piece as obtained after two integrations by parts
\begin{align}
\delta S_{\rm bdy}
&= \int\!\frac{d\omega\,d^3q}{(2\pi)^4}\;
\Big[\,A(r)\,\phi(r,-\omega)\,\delta\phi'(r,\omega)
+ \big(- (A(r)\phi(r,-\omega))' + B(r)\phi'(r,-\omega) \\
&+ C(r)\phi(r,-\omega)\big)\,\delta\phi(r,\omega)
\Big]_{r_{\rm in}}^{r_{\rm out}}.
\end{align}
we expand the derivative on the second term and collect like terms
% 2) expand and simplify the bracket
\begin{align}
-(A\phi)' + B\phi' + C\phi
&= -\big(A'\phi + A\phi'\big) + B\phi' + C\phi \\
&= (B - A)\,\phi' + (C - A')\,\phi .
\end{align}
such that the boundary variation obtains the form
% 3) simplified boundary variation
\begin{align}
\delta S_{\rm bdy}
&= \int\!\frac{d\omega\,d^3q}{(2\pi)^4}\;
\Big[\,A(r)\,\phi(r,-\omega)\,\delta\phi'(r,\omega)
+ \big((B(r)-A(r))\,\phi'(r,-\omega) \\
&+ (C(r)-A'(r))\,\phi(r,-\omega)\big)\,\delta\phi(r,\omega)
\Big]_{r_{\rm in}}^{r_{\rm out}}.
\end{align}
Unfortunately, from here, I am not sure how to go much further.






\end{document}
