\documentclass[12pt]{article}

% Packages
\usepackage[margin=1in]{geometry}
\usepackage{fancyhdr}
\usepackage{titlesec}
\usepackage{amsmath, amssymb, bm}
\usepackage{cancel}

\usepackage{mathtools}
\newcommand{\w}{\omega}
\newcommand{\phip}{\phi^{'}}
\newcommand{\phipp}{\phi^{''}}
\newcommand{\del}{\partial}
\newcommand{\delmu}{\partial_{\mu}}
\newcommand{\delMu}{\partial^{\mu}}
\newcommand{\delnu}{\partial_{\nu}}
\newcommand{\delNu}{\partial^{\nu}}
\newcommand{\wk}{\omega_{\mathbf{k}}}
\newcommand{\wkp}{\omega_{\mathbf{k'}}}
\newcommand{\+}{\dagger}
\DeclarePairedDelimiter\bra{\langle}{\rvert}
\DeclarePairedDelimiter\ket{\lvert}{\rangle}
\DeclarePairedDelimiterX\braket[2]{\langle}{\rangle}{#1\,\delimsize\vert\,\mathopen{}#2}
\usepackage{braket}

% Header & Footer
\pagestyle{fancy}
\fancyhf{}
\fancyhead[L]{Name:  \underline{Simon Lavoie}}
\fancyhead[C]{Class: \underline{PHYS 610}}
\fancyhead[R]{Student ID: \underline{261 051 325}}
\fancyfoot[C]{\thepage}

% Title format for neatness
\titleformat{\section}{\large\bfseries}{\thesection.}{0.5em}{}
\titleformat{\subsection}{\normalsize\bfseries}{\thesubsection)}{0.5em}{}

% Document
\begin{document}

\section*{Question 1: Pole structures of the two-point functions}
The two-point functions in momentum space are given by:
\begin{equation*}
    D_F(p) = \frac{i}{p^2 - m^2 + i\epsilon}
\end{equation*}
For a free massless scalar field theory, we set $m=0$:
\begin{align*}
    D_F(p) &= \frac{i}{(p^0)^2 - \mathbf{p}^2 + i\epsilon} \\
    &= \frac{i}{(p^0 - |\mathbf{p}| + i\epsilon)(p^0 + |\mathbf{p}| - i\epsilon)}
\end{align*}
thus we have simple poles at $p^0 = \pm |\mathbf{p}| \mp i\epsilon$. We can compute the
residues via the formula:
\begin{equation*}
    \text{Res}(f, z_0) = \lim_{z \to z_0} (z - z_0) f(z)
\end{equation*}
where $z_0$ are our two poles and $f(z)$ is our two-point function $D_F(p)$. We find:
\begin{align*}
    \text{Res}(D_F, |\mathbf{p}| - i\epsilon) &= \lim_{p^0 \to |\mathbf{p}| -
    i\epsilon} (p^0 - |\mathbf{p}| + i\epsilon) \frac{i}{(p^0 - |\mathbf{p}| + i\epsilon)(p^0 + |\mathbf{p}| - i\epsilon)}
 \\
    &= \lim_{p^0 \to |\mathbf{p}| - i\epsilon} \frac{i}{p^0 + |\mathbf{p}| -
    i\epsilon} = \frac{i}{2(|\mathbf{p}|  - i\epsilon)} \\
    %comment
    \text{Res}(D_F, -|\mathbf{p}| + i\epsilon) &= \lim_{p^0 \to -|\mathbf{p}| +
    i\epsilon} (p^0 + |\mathbf{p}| - i\epsilon) D_F(p) \\
    &= \lim_{p^0 \to -|\mathbf{p}| + i\epsilon} \frac{i}{p^0 - |\mathbf{p}| +
    i\epsilon} = -\frac{i}{2(|\mathbf{p}| + i\epsilon)}
\end{align*}
In the non-relativistic limit, we instead have:
\begin{align*}
    D_F(p) &= \frac{i}{(p^0)^2 - m^2 + i\epsilon} \\
    &= \frac{i}{(p^0 - m + i\epsilon)(p^0 + m - i\epsilon)}
\end{align*}
and so in an identical manner, we find simple poles at $p^0 = \pm m \mp i\epsilon$ with
residues:
\begin{align*}
    \text{Res}(D_F, m - i\epsilon) &= \lim_{p^0 \to m - i\epsilon} (p^0 - m + i\epsilon)
    D_F(p) \\
    &= \lim_{p^0 \to m - i\epsilon} \frac{i}{p^0 + m - i\epsilon} = \frac{i}{2(m - i\epsilon)} \\
    %comment
    \text{Res}(D_F, -m + i\epsilon) &= \lim_{p^0 \to -m + i\epsilon} (p^0 + m - i\epsilon)
    D_F(p) \\
    &= \lim_{p^0 \to -m + i\epsilon} \frac{i}{p^0 - m + i\epsilon} = -\frac{i}{2(m + i\epsilon)} \\
\end{align*}
Thus, the Feynmann propagator
\begin{align*}
    D_F(x-y) &= \int \frac{d^4p}{(2\pi)^4} \frac{i e^{-ip\cdot(x-y)}}{(p^0)^2 - m^2
    + i\epsilon} \\
    &= \int \frac{d^3\mathbf{p}}{(2\pi)^3} \int \frac{dp^0}{2\pi} \frac{i
    e^{-ip^0(x^0 - y^0) + i\mathbf{p}\cdot(\mathbf{x}-\mathbf{y})}}{(p^0)^2 - m^2
    + i\epsilon} \\
    &= \int \frac{d^3\mathbf{p}}{(2\pi)^3}e^{i\mathbf{p}\cdot (\mathbf{x} - \mathbf{y})} \int \frac{dp^0}{2\pi} \frac{i
    e^{-ip^0(x^0 - y^0)}}{(p^0)^2 - m^2 + i\epsilon} \\
    &= \delta^3(x - y) \int \frac{dp^0}{2\pi} \frac{i
    e^{-ip^0(x^0 - y^0)}}{(p^0  - m + i\epsilon)(p^0 + m - i\epsilon)}
\end{align*}
Now, let us define 
\begin{align*}
    I(t) &= \int \frac{dp^0}{2\pi}e^{-ip^0 t} \frac{i}{(p^0  - m +
    i\epsilon)(p^0 + m - i\epsilon)}  \\
    &= \int \frac{dp^0}{2\pi} e^{-ip^0 t} D_F(p)
\end{align*}
where $t = x^0 - y^0$. We can evaluate this integral using the residue theorem.
Notice that the  inclusion of the $e^{-ip^0t}$ has  the same poles as
$D_F(p)$, because it is analytic. We consider two cases: $t>0$ and $t<0$. For
$t>0$, we close the contour in the lower half-plane, which encloses the pole at
$p^0 = m - i\epsilon$. Thus, we find:
\begin{align*}
    I(t) &= \frac{-2\pi i}{2\pi} \cdot \text{Res}(D_F, m - i\epsilon) e^{-i(m - i\epsilon)t} \\
    &= -i \cdot \frac{i}{2(m - i\epsilon)} e^{-i(m - i\epsilon)t} \\
    &= \frac{1}{2(m - i\epsilon)} e^{-i(m - i\epsilon)t}
\end{align*}
Such that the lower half-plane result is
\begin{equation*}
   D_{F_l}(x-y) = \delta^3(x-y) \frac{1}{2(m - i\epsilon)} e^{-i(m - i\epsilon)(x^0 - y^0)}
\end{equation*}
For $t<0$, we close the contour in the upper half-plane, and we obtain  an equivalent 
result:
\begin{equation*}
    D_{F_u}(x-y) = \delta^3(x-y) \frac{1}{2(m + i\epsilon)} e^{i(m + i\epsilon)(x^0 - y^0)}
\end{equation*}
We can combine these two results into one expression using the Heaviside step function
$\theta(t)$:
\begin{equation*}
    D_F(x-y) = \delta^3(x-y) \left[ \theta(t) \frac{e^{-i(m  - i\epsilon)t}}{2(m - i\epsilon)} 
    +  \theta(-t) \frac{e^{i(m + i\epsilon)t}}{2(m + i\epsilon)} \right] \\
\end{equation*}
Taking the limit $\epsilon \to 0$, we find:
\begin{align*}
    D_F(x-y) &= \delta^3(x-y) \left[ \theta(t) \frac{e^{-imt}}{2m} +  \theta(-t)
    \frac{e^{imt}}{2m} \right] \\
    &= \delta^3(x-y) \frac{1}{2m} \left[ \theta(t) e^{-imt} +  \theta(-t) e^{imt}
    \right] \\
    &= \delta^3(x-y) \frac{1}{2m} e^{-im|t|} \\
\end{align*}
\begin{equation}
    \boxed{D_F(x-y) = \delta^3(x-y) \frac{1}{2m} e^{-im|x^0 - y^0|}}
\end{equation}

Now, let us consider the scalar field theory with action 
\begin{equation}
    S =  \int d^{D+1}x \left( \frac{1}{2}\delmu\phi\delMu\phi + |\lambda_2|\phi^2 + \lambda_3\phi^4 \right)
\end{equation}
We always begin by computing the minima of the potential, which  in this case is $|\lambda_2|\phi^2 + \lambda_3\phi^4$. We find:
\begin{align*}
    \frac{d}{d\phi} (|\lambda_2|\phi^2 + \lambda_3\phi^4) &= 2|\lambda_2|\phi + 4\lambda_3\phi^3 = 0 \\
    \phi(2|\lambda_2| + 4\lambda_3\phi^2) &= 0
\end{align*}
Thus, we have three stationary points: $\phi_c = 0$ and $\phi = \pm
\sqrt{\frac{-|\lambda_2|}{2\lambda_3}}$. The following plot shows that $\lambda_3 < 0$ is 
the only case where we have local minima at $\phi = \pm \sqrt{\frac{-|\lambda_2|}{2\lambda_3}}$:
\begin{center}
    %\includegraphics[width=0.8\textwidth]{fig.png}
\end{center}
In the case where the potential has $\lambda_2\phi^2$ instead of
$|\lambda_2|\phi^2$, then when $\lambda_2 < 0$ we have a single minimum at $\phi = 0$,
and when $\lambda_2 > 0$ we have local maxima at $\phi = 0$ and local minima at $\phi = \pm
\sqrt{\frac{-|\lambda_2|}{2\lambda_3}}$.

Now, we can expand the field around one of the minima $\phi_c = \pm
\sqrt{\frac{-|\lambda_2|}{2\lambda_3}}$ by defining a fluctuation field
$\phi = \varphi - \phi_c$. The potential becomes
\begin{equation}
    V(\varphi) = -|\lambda_2|\varphi^2 + \lambda_3 \varphi^4,
\end{equation}
and expanding around $\phi_c$ gives
\begin{align}
    V(\phi_c + \phi)
    &= V(\phi_c) + V'(\phi_c)\phi + \frac{1}{2}V''(\phi_c)\phi^2 + \mathcal{O}(\phi^3) \\
    &= \frac{|\lambda_2|^2}{4\lambda_3} + 2|\lambda_2|\phi^2 + \mathcal{O}(\phi^3),
\end{align}
where we used $V'(\phi_c) = 0$ and $V''(\phi_c) = 4|\lambda_2|$.

Thus, the potential near the minimum is approximately quadratic:
\begin{equation}
    V(\phi) \simeq \frac{|\lambda_2|^2}{4\lambda_3} + 2|\lambda_2|\phi^2.
\end{equation}

The constant term does not affect the dynamics, so the quadratic (free) part of the action becomes
\begin{equation}
    S[\phi] = \frac{1}{2}\int d^{D+1}x\,\left( \partial_\mu\phi\,\partial^\mu\phi - 4|\lambda_2|\,\phi^2 \right).
\end{equation}
By comparison with the canonical Klein–Gordon form
\[
    S = \frac{1}{2}\int d^{D+1}x\,\left( \partial_\mu\phi\,\partial^\mu\phi - m^2\phi^2 \right),
\]
we identify the new mass as
\begin{equation}
    m = 2\sqrt{|\lambda_2|}.
\end{equation}
This way, we can write the Feynmann propagator as 
\begin{equation*}
    D_F(p) = \frac{i}{p^2 - 4|\lambda_2| + i\epsilon}.
\end{equation*}
With simple poles at $p^0 = \pm \sqrt{\mathbf{p}^2 + 4|\lambda_2|} \mp i\epsilon$ and residues:
\begin{align*}
    \text{Res}(D_F, \sqrt{\mathbf{p}^2 + 4|\lambda_2|} - i\epsilon) &= \frac{i}{2(\sqrt{\mathbf{p}^2 + 4|\lambda_2|} - i\epsilon)} \\
    \text{Res}(D_F, -\sqrt{\mathbf{p}^2 + 4|\lambda_2|} + i\epsilon) &= -\frac{i}{2(\sqrt{\mathbf{p}^2 + 4|\lambda_2|} + i\epsilon)}
\end{align*}

To obtain the propagator in coordinate space, we perform the Fourier transform:
\begin{align*}
    D_F(x - y) &= \int \frac{d^{D+1}p}{(2\pi)^{D+1}} \,
    \frac{i\,e^{-ip\cdot(x - y)}}{p^2 - 4|\lambda_2| + i\epsilon}. \\
    D_F(x - y) &=
    \int \frac{d^D\mathbf{p}}{(2\pi)^D} e^{i\mathbf{p}\cdot(\mathbf{x} - \mathbf{y})}
    \int \frac{dp^0}{2\pi} \,
    \frac{i\,e^{-ip^0(x^0 - y^0)}}{(p^0)^2 - (\mathbf{p}^2 + 4|\lambda_2|) + i\epsilon}.
\end{align*}

The integrand has simple poles at
\[
    p^0 = \pm E_{\mathbf{p}} \mp i\epsilon, \qquad
    E_{\mathbf{p}} = \sqrt{\mathbf{p}^2 + 4|\lambda_2|}.
\]
Closing the contouts and using the residue theorem in the same manner as before, we find:
\begin{equation}
    \boxed{D_F(x - y)
    = \int \frac{d^D\mathbf{p}}{(2\pi)^D}
    \frac{e^{i\mathbf{p}\cdot(\mathbf{x} - \mathbf{y})}}{2E_{\mathbf{p}}}
    \left[
        \theta(x^0 - y^0) e^{-iE_{\mathbf{p}}(x^0 - y^0)}
        + \theta(y^0 - x^0) e^{+iE_{\mathbf{p}}(x^0 - y^0)}
    \right]}
\end{equation}

\section*{Question 2: Energy-momentum tensor in field theory}
For a free scalar field theory, we have the Lagrangian density
\begin{align*}
    \mathcal{L} &= \frac{1}{2}\delmu\phi\delMu\phi - \frac{1}{2}m^2\phi^2.\\
    &= \frac{1}{2}\eta^{\mu\nu}\delmu\phi\delnu\phi - \frac{1}{2}m^2\phi^2.\\
\end{align*}
Or, on some arbitrary curved spacetime with metric $g_{\mu\nu}$, we have
\begin{equation*}
    \mathcal{L} = \frac{1}{2}g^{\mu\nu}\delmu\phi\delnu\phi - \frac{1}{2}m^2\phi^2.
\end{equation*}
Running with this, we can compute
\begin{align*}
    -\frac{2}{\sqrt{-g}}\frac{\partial(\sqrt{-g}\mathcal{L})}{\partial g^{\mu\nu}}
    &= -\frac{2}{\sqrt{-g}} \left( \frac{\partial\sqrt{-g}}{\partial g^{\mu\nu}}\mathcal{L} + \sqrt{-g}\frac{\partial\mathcal{L}}{\partial g^{\mu\nu}} \right)\\
\end{align*}
Using the matrix identity $\ln \det A = \text{tr} \ln A$, and varying both sides, we find
\begin{align*}
    \delta(\ln \det A) &= \delta(\text{tr} \ln A) \\
    &= \text{tr}(\delta \ln A) \\
    &= \text{tr} (A^{-1}\delta A) \\
\end{align*}
Therefore, we have
\begin{equation*}
    \frac{\delta(\det A)}{\det A} = \text{tr}(A^{-1}\delta A).
\end{equation*}
Putting this in the language of metrics, we have
\begin{align*}
    \frac{\delta g}{g} &= g^{\mu\nu}\delta g_{\mu\nu} \\
    \delta g &= g g^{\mu\nu}\delta g_{\mu\nu} \\
\end{align*}
Therefore, we can compute
\begin{align*}
    \delta \sqrt{-g} &= -\frac{1}{2\sqrt{-g}}\delta g \\
    &= -\frac{1}{2\sqrt{-g}} g g^{\mu\nu}\delta g_{\mu\nu} \\
    &= -\frac{1}{2}\sqrt{-g} g^{\mu\nu}\delta g_{\mu\nu} \\
    \frac{\partial \sqrt{-g}}{\partial g^{\mu\nu}} &= -\frac{1}{2}\sqrt{-g} \; g_{\mu\nu} \\
\end{align*}
Putting this back into our expression, we find
\begin{align*}
    \frac{2}{\sqrt{-g}}\frac{\partial(\sqrt{-g}\mathcal{L})}{\partial g^{\mu\nu}}
    &= \frac{2}{\sqrt{-g}} \left( -\frac{1}{2}\sqrt{-g} \;
    g_{\mu\nu}\left(\frac{1}{2}\delmu\phi\delnu\phi - \frac{1}{2}m^2\phi^2\right)+
    \sqrt{-g}\left(\frac{1}{2}\delmu\phi\delnu\phi \right)\right)\\
    &= -g_{\mu\nu}\left(\frac{1}{2}\delmu\phi\delnu\phi - \frac{1}{2}m^2\phi^2\right) + \delmu\phi\delnu\phi \\
    &= \delmu\phi\delnu\phi - g_{\mu\nu}\mathcal{L}\\
\end{align*}
Or, swapping back to Minkowski space, we have
\begin{equation*}
    \boxed{T_{\mu\nu} = \delmu\phi\delnu\phi - \eta_{\mu\nu}\mathcal{L}.}
\end{equation*}
Just as was obtained using Noether's theorem.

To couple the free field theory Lagrangian to gravity, we replace the Minkowski
metric with a general metric $g_{\mu\nu}$, and we include a factor of
$\sqrt{-g}$ to ensure that the action is a scalar under general coordinate
transformations. Lastly, we would also have to make the derivatives covariant
instead of partial, but since $\phi$ is a scalar field, the covariant derivative
reduces to the partial derivative.

\section*{Question 3}
We can imagine $\phi$ and $\phi^{\dagger}$ as a sum of two real fields $\phi_1$ 
and $\phi_2$, such that $\phi = \phi_1 + i\phi_2$ and $\phi^{\dagger} = \phi_1 -
i\phi_2$. Plugging this presecription into our Lagrangian
\begin{align*}
   \mathcal{L} &= \frac{1}{2}\delmu\phi\delMu\phi^{\dagger} \\ 
               &= \frac{1}{2}\delmu(\phi_1 + i \phi_2)\delMu(\phi_1 - i\phi_2) \\
               &= \frac{1}{2}\delmu\phi_1\delMu\phi_1
               + \frac{1}{2}\delmu\phi_2\delMu\phi_2
               + \frac{i}{2}\delmu\phi_2\delMu\phi_1
               - \frac{i}{2}\delmu\phi_1\delMu\phi_2 \\
               &= \frac{1}{2}\delmu\phi_1\delMu\phi_1
               + \frac{1}{2}\delmu\phi_2\delMu\phi_2
               + \frac{i}{2}\delmu\delMu\phi_2\phi_1
               - \frac{i}{2}\delmu\delMu\phi_2\delMu\phi_1 \\
               &= \frac{1}{2}\delmu\phi_1\delMu\phi_1
               + \frac{1}{2}\delmu\phi_2\delMu\phi_2
\end{align*}
because two independent scalar fields $\phi_1$, $\phi_2$ commute. Thus, we can
see this Lagrangian describes two independent scalar field theories. Now,
let us quantize both $\phi_1$ and $\phi_2$ independently, giving them each
their own creation and annihilation operators, which we shall label $\alpha$ 
and $\beta$.
\begin{align*}
   \phi_1(x) &= \int \frac{d^3k}{(2\pi)^3} \frac{1}{\sqrt{2\omega_k}} (\alpha_k
   e^{-ik\cdot x} + \alpha_k^{\dagger}e^{ik\cdot x}) \\
   \phi_2(x) &= \int \frac{d^3k}{(2\pi)^3} \frac{1}{\sqrt{2\omega_k}} (\beta
   e^{-ik\cdot x} + \beta^{\dagger}e^{ik\cdot x}) \\
\end{align*}
Using these definitions, we write $\phi$ and $\phi^{\dagger}$
\begin{align*}
   \phi(x) &= \int \frac{d^3k}{(2\pi)^3} \frac{1}{\sqrt{2\omega_k}} \left(
   (\alpha_k + i\beta_k)e^{-ik\cdot x} + (\alpha^{\dagger}_k +
   i\beta^{\dagger}_k)e^{ik\cdot x} \right) \\
   \phi^{\dagger}(x) &= \int \frac{d^3k}{(2\pi)^3} \frac{1}{\sqrt{2\omega_k}}
   \left( (\alpha_k - i\beta_k)e^{ik\cdot x} + (\alpha^{\dagger}_k -
   i\beta^{\dagger}_k)e^{-ik\cdot x} \right) \\
\end{align*}
let us now define $a_k = \alpha_k + i\beta_k$ and $b_k = \alpha_k - i\beta_k$. Then,
immediately we see $a^{\dagger}_k = \alpha_k^{\dagger} - i\beta_k^{\dagger}$ and 
$b^{\dagger}_k = \alpha_k^{\dagger} + i \beta_k^{\dagger}$ such that

\begin{align*}
   \phi(x) &= \int \frac{d^3k}{(2\pi)^3} \frac{1}{\sqrt{2\omega_k}} \left(
   a_k e^{-ik\cdot x} + b_k^{\dagger}e^{ik\cdot x}
   \right) \\
   \phi^{\dagger}(x) &= \int \frac{d^3k}{(2\pi)^3} \frac{1}{\sqrt{2\omega_k}}
   \left( a^{\dagger}_k e^{ik\cdot x} + b_k e^{-ik\cdot x} \right) \\
\end{align*}
Now, let's compute $[\phi(x), \phi^{\dagger}(y)]$

\begin{align*}
   [\phi(x),\phi^\dagger(y)]
   &=\int \frac{d^3\mathbf{k}}{(2\pi)^3}
          \frac{d^3\mathbf{k}'}{(2\pi)^3}\,
          \frac{1}{\sqrt{2\omega_{\mathbf{k}}}
                   \sqrt{2\omega_{\mathbf{k}'}}}\,
   \Big[
     \big(a_{\mathbf{k}} e^{-ik\cdot x}
      + b_{\mathbf{k}}^\dagger e^{+ik\cdot x}\big),
     \big(a_{\mathbf{k}'}^\dagger e^{+ik'\cdot y}
      + b_{\mathbf{k}'} e^{-ik'\cdot y}\big)
   \Big] \\[6pt]
   &=\int \frac{d^3\mathbf{k}}{(2\pi)^3}
           \frac{d^3\mathbf{k}'}{(2\pi)^3}\,
           \frac{1}{\sqrt{2\omega_{\mathbf{k}}}
                    \sqrt{2\omega_{\mathbf{k}'}}}\,
   \Big(
     a_{\mathbf{k}} a_{\mathbf{k}'}^\dagger e^{-ik\cdot x} e^{+ik'\cdot y}
    +a_{\mathbf{k}} b_{\mathbf{k}'} e^{-ik\cdot x} e^{-ik'\cdot y} \\
   &\qquad\qquad
    +b_{\mathbf{k}}^\dagger a_{\mathbf{k}'}^\dagger e^{+ik\cdot x} e^{+ik'\cdot y}
    +b_{\mathbf{k}}^\dagger b_{\mathbf{k}'} e^{+ik\cdot x} e^{-ik'\cdot y} \\
   &\qquad\qquad
    -a_{\mathbf{k}'}^\dagger a_{\mathbf{k}} e^{+ik'\cdot y} e^{-ik\cdot x}
    -b_{\mathbf{k}'} a_{\mathbf{k}} e^{-ik'\cdot y} e^{-ik\cdot x} \\
   &\qquad\qquad
    -a_{\mathbf{k}'}^\dagger b_{\mathbf{k}}^\dagger e^{+ik'\cdot y} e^{+ik\cdot x}
    -b_{\mathbf{k}'} b_{\mathbf{k}}^\dagger e^{-ik'\cdot y} e^{+ik\cdot x}
   \Big) \\[6pt]
   &=\int \frac{d^3\mathbf{k}}{(2\pi)^3}
           \frac{d^3\mathbf{k}'}{(2\pi)^3}\,
           \frac{1}{\sqrt{2\omega_{\mathbf{k}}}
                    \sqrt{2\omega_{\mathbf{k}'}}}\,
   \Big(
    [a_{\mathbf{k}},a_{\mathbf{k}'}^\dagger]\,e^{-ik\cdot x}e^{+ik'\cdot y}
    +[a_{\mathbf{k}},b_{\mathbf{k}'}]\,e^{-ik\cdot x}e^{-ik'\cdot y} \\
   &\qquad\qquad
    +[b_{\mathbf{k}}^\dagger,a_{\mathbf{k}'}^\dagger]\,e^{+ik\cdot x}e^{+ik'\cdot y}
    +[b_{\mathbf{k}}^\dagger,b_{\mathbf{k}'}]\,e^{+ik\cdot x}e^{-ik'\cdot y}
   \Big) \\[6pt]
   &=\int \frac{d^3\mathbf{k}}{(2\pi)^3}
           \frac{d^3\mathbf{k}'}{(2\pi)^3}\,
           \frac{1}{\sqrt{2\omega_{\mathbf{k}}}
                    \sqrt{2\omega_{\mathbf{k}'}}}\,
   \Big(
    (2\pi)^3\delta^{(3)}(\mathbf{k}-\mathbf{k}')\,e^{-ik\cdot x}e^{+ik'\cdot y}
    -(2\pi)^3\delta^{(3)}(\mathbf{k}-\mathbf{k}')\,e^{+ik\cdot x}e^{-ik'\cdot y}
   \Big) \\[6pt]
   &=\int \frac{d^3\mathbf{k}}{(2\pi)^3}\,
         \frac{1}{2\omega_{\mathbf{k}}}\,
         \Big(e^{-ik\cdot(x-y)} - e^{+ik\cdot(x-y)}\Big). \\
   &= \int \frac{d^3k}{(2\pi)^3} e^{-i\mathbf{k}(\mathbf{x} - \mathbf{y})}
   \left(\frac{1}{2\omega_k} e^{-ik\cdot(x-y)}\bigg|_{k^0 = \omega_k}
   + \frac{1}{-2\omega_k}e^{-ik\cdot(x-y)}\bigg|_{k^0 = -\omega_k}\right) \\
   &= \int \frac{d^3k}{(2\pi)^3} \int \frac{dp^0}{2\pi i} \frac{-1}{p^2 - m^2}
   e^{-ik\cdot (x-y)} \\
   &= \int \frac{d^4k}{(2\pi)^4} \frac{i}{p^2 - m^2} e^{-ik\cdot(x-y)}
\end{align*}
which we immediately recognize as $D_R(x-y)$, the retarded propagator. It makes
sense that the commutator would yield the retarded propagator because the
commutator measures how much the quantum field $\phi$ at one point in spacetime
$x$ interferes with $\phi^{\dagger}$ at a later point $y$, and $D_R(x-y)$
describes what this causal effect is.

Next, to find the Hamiltonian, we will need to compute the conjugate momenta
\begin{align*}
   \pi(x) &= \frac{\partial \mathcal{L}}{\partial(\del_0 \phi)} = \tfrac12\del^0
   \phi^{\dagger} \\
   \pi^{\dagger}(x) &= \frac{\partial \mathcal{L}}{\partial(\del_0
   \phi^{\dagger})} = \tfrac12\del^0 \phi \\
\end{align*}
then, the Hamiltonian density is given by
\begin{align*}
    \mathcal{H} &= \pi \del_0 \phi + \pi^{\dagger} \del_0 \phi^{\dagger} -
    \mathcal{L} \\
    &= \del^0 \phi^{\dagger} \del_0 \phi + \del^0 \phi \del_0
    \phi^{\dagger} - \left( \delmu\phi^{\dagger}\delMu\phi \right) \\
    &= \del^0 \phi^{\dagger} \del_0 \phi + \del^0 \phi \del_0
    \phi^{\dagger} - \left( \del^0\phi^{\dagger}\del_0\phi -
    \nabla\phi^{\dagger}\cdot\nabla\phi \right) \\
    &= \nabla\phi^{\dagger}\cdot\nabla\phi + \del^0 \phi^{\dagger} \del_0 \phi \\
\end{align*}

Now, considering the massive case, we have the Lagrangian
\begin{equation*}
   \mathcal{L} = \delmu\phi^{\dagger}\delMu\phi - \tfrac12 m^2 \phi^{\dagger}\phi
\end{equation*}
If we consider a transformation 
\begin{equation*}
   \phi \;\to\; \phi' \,=\, e^{-i\theta}\phi, \qquad
   \phi^\dagger \;\to\; \phi'^{\dagger} \,=\, e^{+i\theta}\phi^\dagger .
\end{equation*}
then we can tell that if we were to plug this into the Lagrangian, we would
find no change since in both terms, the exponential factors cancel. Further,
expanding the rotated fields in terms of real components, we find
\begin{align*}
   \phi' \,=\, \phi_1'+i\phi_2'
   \;=\; e^{-i\theta}\,(\phi_1+i\phi_2)
   \;=\; (\cos\theta\,\phi_1+\sin\theta\,\phi_2)
   \;+\; i(\cos\theta\,\phi_2-\sin\theta\,\phi_1)
\end{align*}
so
\begin{align*}
   \begin{pmatrix}\phi_1'\\ \phi_2'\end{pmatrix}
   \;=\;
   \begin{pmatrix}
      \cos\theta & \ \ \sin\theta \\
      -\sin\theta & \ \ \cos\theta
   \end{pmatrix}
   \begin{pmatrix}\phi_1\\ \phi_2\end{pmatrix}.
\end{align*}
Thus the global phase $\phi\to e^{-i\theta}\phi$ acts as a rotation by angle $\theta$ in the $(\phi_1,\phi_2)$ plane.
Now, considering a small $\theta$, we can expand the exponentials to first order in $\theta$:
\begin{align*}
   e^{-i\theta} &\simeq 1 - i\theta, \\
   e^{+i\theta} &\simeq 1 + i\theta,
\end{align*}
so
\begin{align*}
    \phi' &= (1 - i\theta)\phi = \phi - i\theta\phi, \\
   \therefore \delta\phi &= -i\theta\,\phi \\
   \phi'^{\dagger} &= (1 + i\theta)\phi^{\dagger} = \phi^{\dagger} + i\theta\phi^{\dagger}, \\
   \therefore \delta\phi^\dagger &= +i\theta\,\phi^\dagger
\end{align*}
Now, we can find the conserved current using Noether's theorem:
\begin{align*}
   J^\mu
   &= \frac{\partial \mathcal{L}}{\partial(\del_\mu \phi)}\,\delta\phi
     \;+\; \frac{\partial \mathcal{L}}{\partial(\del_\mu \phi^\dagger)}\,\delta\phi^\dagger \\
   &= \big(\del^\mu\phi^\dagger\big)\,(-i\theta\,\phi)
     \;+\; \big(\del^\mu\phi\big)\,(+i\theta\,\phi^\dagger) \\
   &= \frac{i\theta}{2}\Big(\phi^\dagger \del^\mu \phi \;-\; (\del^\mu\phi^\dagger)\,\phi\Big).
\end{align*}
Then, the conserved charge $Q$ can be found by integrating the charge density
over space
\begin{align*}
    Q &= \int d^3x J^{0} = \int d^3x \; \frac{i\theta}{2}\Big(\phi^\dagger \del^0 \phi -
    \del^0\phi^\dagger\phi\Big). \\
      &= \int d^3x \frac{i\theta}{2} \Big(\phi^{\dagger}\pi^{\dagger} - \pi\phi \Big) \\
\end{align*}
Knowing that all of $\phi$, $\pi$ (and their daggered counterparts) are quantized,
then we can express the charge in terms of their mode expansions. For this we
can compute $\del^0\phi$ and $\del^0\phi^{\dagger}$. Using 
\begin{align*}
    \del^0 c_k e^{-ik\cdot x} &= c_k \del^0 e^{-ik^0x^0 + i\mathbf{k} \cdot \mathbf{x}} \\
                              &= -i k^0 c_k e^{-ik\cdot x} \\
                              &= -i \omega_k c_k e^{-ik\cdot x}
\end{align*}
Therefore
\begin{align*}
    \tfrac12\del^0\phi &= \pi^{\dagger}(x) = \tfrac12 \int \frac{d^3k}{(2\pi)^3}
    \frac{1}{\sqrt{2\omega_k}} \Big((-i\omega_k) a_k e^{-ik\cdot x} +
    (i\omega_k) b^{\dagger}_k e^{ik\cdot x} \Big) \\
                       &= \frac{-i}{2} \int \frac{d^3k}{(2\pi)^3}
                       \sqrt{\frac{\omega_k}{2}}\Big(a_ke^{-ik\cdot x}
                           - b^{\dagger}_ke^{ik\cdot x} \Big)
\end{align*}
and similarly
\begin{equation*}
    \pi(x) = \frac{i}{2} \int \frac{d^3k}{(2\pi)^3} \sqrt{\frac{\omega_k}{2}}
    \Big(a^{\dagger}_ke^{ik\cdot x} - b_ke^{-ik\cdot x} \Big)
\end{equation*}
And so the conserved charge is given by
\begin{align*}
\hat{Q}
  &= \frac{i}{4} \!\int\! \frac{d^3x\, d^3k\, d^3k'}{(2\pi)^6}
     \sqrt{\frac{\omega_{k'}}{\omega_k}}\,
     \Big[
       \big(a_{k'}^{\dagger} e^{ik'\cdot x} + b_{k'} e^{-ik'\cdot x}\big)
       (-i)\big(a_k e^{-ik\cdot x} - b_k^{\dagger} e^{ik\cdot x}\big)
\\[-0.5em]
  &\quad\quad\;\;
       -\big(a_{k'}^{\dagger} e^{ik'\cdot x} - b_{k'} e^{-ik'\cdot x}\big)
       (i)\big(a_k e^{-ik\cdot x} + b_k^{\dagger} e^{ik\cdot x}\big)
     \Big] \\
  &= \frac{1}{4} \!\int\! \frac{d^3x\, d^3k\, d^3k'}{(2\pi)^6}
     \sqrt{\frac{\omega_{k'}}{\omega_k}}\,
     \Big[
     \big(a^{\dagger}_{k'}a_ke^{i(k' - k)\cdot x} -
         a^{\dagger}_{k'}b^{\dagger}_{k} e^{i(k' + k)\cdot x} + b_{k'}a_k
         e^{i(k + k')\cdot x} - b_{k'}b^{\dagger}_k e^{-i(k - k')\cdot x} \big)
\\[-0.5em]
  &\quad\quad\;\;
       + \big(a^{\dagger}_{k'}a_ke^{-i(k' - k)\cdot x} +
         a^{\dagger}_{k'}b^{\dagger}_{k} e^{-i(k' + k)\cdot x} - b_{k'}a_k
         e^{i(k' + k)\cdot x} - b_{k'}b^{\dagger}_k e^{-i(k - k')\cdot x} \big)
     \Big]
\end{align*}

We now carry out the spatial integral. Each exponential gives either 
$e^{i(\mathbf{k}'-\mathbf{k})\cdot\mathbf{x}}$ or 
$e^{i(\mathbf{k}'+\mathbf{k})\cdot\mathbf{x}}$, so

\[
\int d^3x\, e^{i(\mathbf{k}'-\mathbf{k})\cdot\mathbf{x}} = (2\pi)^3\delta^{(3)}(\mathbf{k}'-\mathbf{k}),
\qquad
\int d^3x\, e^{i(\mathbf{k}'+\mathbf{k})\cdot\mathbf{x}} = (2\pi)^3\delta^{(3)}(\mathbf{k}'+\mathbf{k}).
\]
\begin{align*}
\hat{Q}
  &= \frac{1}{4} \!\int\! \frac{d^3x\, d^3k\, d^3k'}{(2\pi)^6}
     \sqrt{\frac{\omega_{k'}}{\omega_k}}\,
     \Big[
     \big(a^{\dagger}_{k}a_k - a^{\dagger}_{-k}b^{\dagger}_{k} + b_{-k}a_k -
     b_{k}b^{\dagger}_k \big)
\\[-0.5em]
  &\quad\quad\;\;
       + \big(a^{\dagger}_{k}a_k + a^{\dagger}_{-k}b^{\dagger}_{k} - b_{-k}a_k
       - b_{k}b^{\dagger}_k  \big)
     \Big] \\
&= \frac{1}{4}\!\int\!\frac{d^3k}{(2\pi)^3}
\Big[
    2a_{k}^{\dagger} a_k  - 2b_kb^{\dagger}_k
\Big] \\
&= \frac{1}{2}\!\int\!\frac{d^3k}{(2\pi)^3}
\Big(
    a_{k}^{\dagger} a_k  - b_kb^{\dagger}_k
\Big).
\end{align*}
Using $b_k b_k^{\dagger} = b_k^{\dagger}b_k + (2\pi)^3\delta^{(3)}(0)$, we
normal order and drop the infinite constant.
\[
:\hat{Q}: = \frac{1}{2} \int\!\frac{d^3k}{(2\pi)^3}\,
\big(a_k^{\dagger}a_k - b_k^{\dagger}b_k\big).
\]
Thus the conserved charge counts particles minus antiparticles.

Now, if the rotational angle is space dependent $\theta = \theta(x)$, we see
that the Lagrangian is no longer invariant under $\phi \to e^{-i\theta(x)}\phi$
because the first term
\begin{align*}
    \delmu\phi\delMu\phi^{\dagger} &\to
    \delmu(e^{-i\theta(x)}\phi)\delMu(e^{i\theta(x)}\phi) \\ 
                                   &=
    \big(-i\delmu\theta(x)e^{-i\theta(x)}\phi + e^{-i\theta(x)}\delmu\phi \big)
    \big(i\delmu\theta(x)e^{i\theta(x)}\phi^{\dagger} +
    e^{i\theta(x)}\delmu\phi^{\dagger} \big) \\
                                   &=
    \big(
        (\delmu\theta)^2\phi\phi^{\+} -i\delmu\theta\phi\delmu\phi^{\+}
        + i\delmu\phi\delmu\theta\phi^{\+} + \delmu\phi\delmu\phi^{\+}
    \big)
\end{align*}
For a small angle change, we can ignore the $(\delmu\theta)^2$ term, leaving us
with our new Lagrangian:
\begin{equation*}
    \mathcal{L'} = 
    \frac{1}{2}\delmu\phi\delmu\phi^{\+} -
    \frac{i}{2}\delmu\theta(\phi\delmu\phi^{\+} - \delmu\phi\phi^{\+})
    - \frac{1}{2}m^2\phi\phi^{\+}
\end{equation*}
Which is obviously different than prior to the transformation. Recalling that 
the conserved current for the global gauge transformation was $J^{\mu} =
\frac{i}{2}\big(\phi^{\+}\delmu\phi - \delmu\phi^{\+}\phi\big)$, we can 
read off that the variation in the Lagrangian is
\begin{align*}
    \delta\mathcal{L} &= \mathcal{L'} - \mathcal{L} = 
    - \frac{i}{2}\delmu\theta(\phi\delmu\phi^{\+} - \delmu\phi\phi^{\+}) \\
                     &= \delmu\theta J^{\mu}
\end{align*}
as desired.

To show how the current is not invariant, take:
\begin{align*}
    J'^{\mu} &\to \frac{i}{2}\big(e^{-i\theta(x)}\phi \delmu
    (e^{i\theta(x)}\phi^{\+}) - \delmu (e^{-i\theta(x)}\phi)e^{i\theta(x)}\phi^{\+} \big) \\
            &= \frac{i}{2} \big(
                e^{-i\theta(x)}\phi(i\delmu\theta(x)e^{i\theta(x)}\phi^{\+} + 
                e^{i\theta(x)}\delmu\phi^{\+}) -
                (-i\delmu\theta(x)e^{-i\theta(x)}\phi +
                e^{-i\theta(x)}\delmu\phi)e^{i\theta(x)}\phi^{\+}
            \big) \\
            &= \frac{i}{2} \big(
                i\delmu\theta\phi\phi^{\+} + \phi\delmu\phi^{\+} +
                i\delmu\theta\phi\phi^{\+} - \delmu\phi\phi^{\+}
            \big) \\
            &= \frac{i}{2}\big(\phi\delmu\phi^{\+} - \delmu\phi\phi^{\+}\big) 
            - |\phi|^2\delmu\theta \\
\end{align*}
Such that
\begin{align*}
    \delta J^{\mu} &= |\phi|^2\delmu\theta
\end{align*}
Now, let us consider the same transformation $\phi \to e^{-i\theta(x)}\phi$ but with our 
modified Lagrangian
\begin{align*}
    \mathcal{L} &= \frac{1}{2}\delmu\phi^{\+}\delMu\phi - eJ^{\mu}A_{\mu} + \frac{e^2}{2}
    |\phi|^2A_{\mu}A^{\mu} - \frac{1}{2}m^2\phi^{\+}\phi  \\
\end{align*}
Then the variation in the Lagrangian is
\begin{align*}
    \delta\mathcal{L} &= \delta \mathcal{L}_{\text{old}} + \delta \mathcal{L}_{\text{new}} \\
    &= \delmu\theta  J^{\mu} - e \delta J^{\mu} A_{\mu} - e J^{\mu} \delta A_{\mu} + 
    \frac{e^2}{2} |\phi|^2 \delta A_{\mu}A^{\mu} + \frac{e^2}{2} |\phi|^2 A_{\mu} \delta A^{\mu} \\
    &= \delmu\theta  J^{\mu} - e (|\phi|^2 \delmu\theta) A_{\mu} - e J^{\mu} \delta A_{\mu} + 
    e^2 |\phi|^2 \delta A_{\mu}A^{\mu} \\
    &= \delmu\theta \big(J^{\mu} -e|\phi|^2 A^{\mu} \big) - \big(eJ^{\mu} - e^2
    |\phi|^2 A^{\mu} \big) \delta A_{\mu} \\
    &= \delmu\theta \big(J^{\mu} -e|\phi|^2 A^{\mu} \big) - \big(eJ^{\mu} - e^2
    |\phi|^2 A^{\mu} \big) \Big(\frac{1}{e} \delmu\theta  \Big) \\
    &= \delmu\theta \big(J^{\mu} -e|\phi|^2 A^{\mu} \big) - \delmu\theta\big(J^{\mu} - e
    |\phi|^2 A^{\mu} \big)\\
    &= 0
\end{align*}
and so we see that if we choose $\delta A_{\mu} = \frac{1}{e}\delmu\theta$, then
the variation in the Lagrangian is zero, and thus the Lagrangian is invariant
under the local gauge transformation.

To verify the kinetic term for the gauge field is also invariant, we note that,  using
the result above, $A_{\mu}$ transforms as $A_{\mu} \to A_{\mu}' = A_{\mu} +
\frac{1}{e}\delmu\theta$. Therefore:
\begin{align*}
    \delmu A_{\nu}  -  \delnu A_{\mu} &\to  \delmu (A_{\nu} + \frac{1}{e}\delnu\theta) -
    \delnu (A_{\mu} + \frac{1}{e}\delmu\theta) \\
    &= \delmu A_{\nu} - \delnu A_{\mu} + \frac{1}{e}(\delmu\delnu\theta - \delnu\delmu\theta) \\
    &= \delmu A_{\nu} - \delnu A_{\mu} \\
\end{align*}
since partial derivatives commute. Thus the kinetic term for the gauge field is invariant. Specifically,
if $\left(\delmu A_{\nu}  -  \delnu A_{\mu}\right)$ is invariant, then so is 
$\left(\delmu A_{\nu}  -  \delnu A_{\mu}\right)^2$.

Meanwhile, if we have some mass term such as $\frac{1}{2}m^2 A_{\mu}A^{\mu}$,
then under the gauge transformation
\begin{align*}
    A_{\mu}A^{\mu} &\to (A_{\mu} + \frac{1}{e}\delmu\theta)(A^{\mu} +
    \frac{1}{e}\delmu\theta) \\
                   &= A_{\mu}A^{\mu} + \frac{2}{e}A^{\mu}\delmu\theta +
                   \frac{1}{e^2}(\delmu\theta)^2 \\
\end{align*}
which is not equal to the original term, and so the mass term is not gauge invariant.

Now, let us introduce the covariant derivative $D_{\mu} = \delmu - ieA_{\mu}$. Then,
under the gauge transformation, we have
\begin{align*}
    D_{\mu}\phi &= (\delmu - ieA_{\mu})\phi \\
                &\to (\delmu - ie(A_{\mu} + e^{-1}\delmu\theta))e^{-i\theta}\phi \\
                &= \delmu(e^{-i\theta}\phi) - ieA_{\mu}e^{-i\theta}\phi - i\delmu\theta e^{-i\theta}\phi \\
                &= i\delmu\theta e^{-i\theta}\phi + e^{-i\theta}\delmu\phi -
                ieA_{\mu}e^{-i\theta}\phi - i\delmu\theta e^{-i\theta}\phi \\
                &=  e^{-i\theta}\delmu\phi - ieA_{\mu}e^{-i\theta}\phi \\
                &=  \big(\delmu\phi - ieA_{\mu}\phi \big)e^{-i\theta}\phi \\
                &=  e^{-i\theta}D_{\mu}\phi \\
\end{align*}
Thus, the covariant derivative of $\phi$ transforms in the same way as $\phi$ and not
as its derivative.

Finally, to write the QED Lagrangian using the covariant derivative, we start with
a well known definition for the electromagnetic field strength tensor:
\begin{equation*}
    F_{\mu\nu} = \delmu A_{\nu} - \delnu A_{\mu}
\end{equation*}
Then, the QED Lagrangian can be written as
\begin{equation*}
    \mathcal{L}_{\text{QED}} = \frac{1}{2}(D_{\mu}\phi)(D^{\mu}\phi^{\dagger}) -
    \frac{1}{2}m^2\phi\phi^{\dagger} - \frac{1}{4}F_{\mu\nu}F^{\mu\nu}
\end{equation*}
\end{document}
