\documentclass[12pt]{article}

% Packages
\usepackage[margin=1in]{geometry}
\usepackage{fancyhdr}
\usepackage{titlesec}
\usepackage{amsmath, amssymb, bm}
\usepackage{cancel}

\usepackage{mathtools}
\newcommand{\w}{\omega}
\newcommand{\phip}{\phi^{'}}
\newcommand{\phipp}{\phi^{''}}
\newcommand{\del}{\partial}
\newcommand{\delmu}{\partial_{\mu}}
\newcommand{\delMu}{\partial^{\mu}}
\newcommand{\delnu}{\partial_{\nu}}
\newcommand{\delNu}{\partial^{\nu}}
\newcommand{\wk}{\omega_{\mathbf{k}}}
\newcommand{\wkp}{\omega_{\mathbf{k'}}}
\newcommand{\+}{\dagger}
\DeclarePairedDelimiter\bra{\langle}{\rvert}
\DeclarePairedDelimiter\ket{\lvert}{\rangle}
\DeclarePairedDelimiterX\braket[2]{\langle}{\rangle}{#1\,\delimsize\vert\,\mathopen{}#2}
\usepackage{braket}

% Header & Footer
\pagestyle{fancy}
\fancyhf{}
\fancyhead[L]{Name:  \underline{Simon Lavoie}}
\fancyhead[C]{Class: \underline{PHYS 610}}
\fancyhead[R]{Student ID: \underline{261 051 325}}
\fancyfoot[C]{\thepage}

% Title format for neatness
\titleformat{\section}{\large\bfseries}{\thesection.}{0.5em}{}
\titleformat{\subsection}{\normalsize\bfseries}{\thesubsection)}{0.5em}{}

% Document
\begin{document}

\section*{Question 1: Pole structures of the two-point functions}
The two-point functions in momentum space are given by:
\begin{equation*}
    D_F(p) = \frac{i}{p^2 - m^2 + i\epsilon}
\end{equation*}
For a free massless scalar field theory, we set $m=0$:
\begin{align*}
    D_F(p) &= \frac{i}{(p^0)^2 - \mathbf{p}^2 + i\epsilon} \\
    &= \frac{i}{(p^0 - |\mathbf{p}| + i\epsilon)(p^0 + |\mathbf{p}| - i\epsilon)}
\end{align*}
thus we have simple poles at $p^0 = \pm |\mathbf{p}| \mp i\epsilon$. We can compute the
residues via the formula:
\begin{equation*}
    \text{Res}(f, z_0) = \lim_{z \to z_0} (z - z_0) f(z)
\end{equation*}
where $z_0$ are our two poles and $f(z)$ is our two-point function $D_F(p)$. We find:
\begin{align*}
    \text{Res}(D_F, |\mathbf{p}| - i\epsilon) &= \lim_{p^0 \to |\mathbf{p}| -
    i\epsilon} (p^0 - |\mathbf{p}| + i\epsilon) \frac{i}{(p^0 - |\mathbf{p}| + i\epsilon)(p^0 + |\mathbf{p}| - i\epsilon)}
 \\
    &= \lim_{p^0 \to |\mathbf{p}| - i\epsilon} \frac{i}{p^0 + |\mathbf{p}| -
    i\epsilon} = \frac{i}{2(|\mathbf{p}|  - i\epsilon)} \\
    %comment
    \text{Res}(D_F, -|\mathbf{p}| + i\epsilon) &= \lim_{p^0 \to -|\mathbf{p}| +
    i\epsilon} (p^0 + |\mathbf{p}| - i\epsilon) D_F(p) \\
    &= \lim_{p^0 \to -|\mathbf{p}| + i\epsilon} \frac{i}{p^0 - |\mathbf{p}| +
    i\epsilon} = -\frac{i}{2(|\mathbf{p}| + i\epsilon)}
\end{align*}
In the non-relativistic limit, we instead have:
\begin{align*}
    D_F(p) &= \frac{i}{(p^0)^2 - m^2 + i\epsilon} \\
    &= \frac{i}{(p^0 - m + i\epsilon)(p^0 + m - i\epsilon)}
\end{align*}
and so in an identical manner, we find simple poles at $p^0 = \pm m \mp i\epsilon$ with
residues:
\begin{align*}
    \text{Res}(D_F, m - i\epsilon) &= \lim_{p^0 \to m - i\epsilon} (p^0 - m + i\epsilon)
    D_F(p) \\
    &= \lim_{p^0 \to m - i\epsilon} \frac{i}{p^0 + m - i\epsilon} = \frac{i}{2(m - i\epsilon)} \\
    %comment
    \text{Res}(D_F, -m + i\epsilon) &= \lim_{p^0 \to -m + i\epsilon} (p^0 + m - i\epsilon)
    D_F(p) \\
    &= \lim_{p^0 \to -m + i\epsilon} \frac{i}{p^0 - m + i\epsilon} = -\frac{i}{2(m + i\epsilon)} \\
\end{align*}
Thus, the Feynmann propagator
\begin{align*}
    D_F(x-y) &= \int \frac{d^4p}{(2\pi)^4} \frac{i e^{-ip\cdot(x-y)}}{(p^0)^2 - m^2
    + i\epsilon} \\
    &= \int \frac{d^3\mathbf{p}}{(2\pi)^3} \int \frac{dp^0}{2\pi} \frac{i
    e^{-ip^0(x^0 - y^0) + i\mathbf{p}\cdot(\mathbf{x}-\mathbf{y})}}{(p^0)^2 - m^2
    + i\epsilon} \\
    &= \int \frac{d^3\mathbf{p}}{(2\pi)^3}e^{i\mathbf{p}\cdot (\mathbf{x} - \mathbf{y})} \int \frac{dp^0}{2\pi} \frac{i
    e^{-ip^0(x^0 - y^0)}}{(p^0)^2 - m^2 + i\epsilon} \\
    &= \delta^3(x - y) \int \frac{dp^0}{2\pi} \frac{i
    e^{-ip^0(x^0 - y^0)}}{(p^0  - m + i\epsilon)(p^0 + m - i\epsilon)}
\end{align*}
Now, let us define 
\begin{align*}
    I(t) &= \int \frac{dp^0}{2\pi}e^{-ip^0 t} \frac{i}{(p^0  - m +
    i\epsilon)(p^0 + m - i\epsilon)}  \\
    &= \int \frac{dp^0}{2\pi} e^{-ip^0 t} D_F(p)
\end{align*}
where $t = x^0 - y^0$. We can evaluate this integral using the residue theorem.
Notice that the  inclusion of the $e^{-ip^0t}$ has  the same poles as
$D_F(p)$, because it is analytic. We consider two cases: $t>0$ and $t<0$. For
$t>0$, we close the contour in the lower half-plane, which encloses the pole at
$p^0 = m - i\epsilon$. Thus, we find:
\begin{align*}
    I(t) &= \frac{-2\pi i}{2\pi} \cdot \text{Res}(D_F, m - i\epsilon) e^{-i(m - i\epsilon)t} \\
    &= -i \cdot \frac{i}{2(m - i\epsilon)} e^{-i(m - i\epsilon)t} \\
    &= \frac{1}{2(m - i\epsilon)} e^{-i(m - i\epsilon)t}
\end{align*}
Such that the lower half-plane result is
\begin{equation*}
   D_{F_l}(x-y) = \delta^3(x-y) \frac{1}{2(m - i\epsilon)} e^{-i(m - i\epsilon)(x^0 - y^0)}
\end{equation*}
For $t<0$, we close the contour in the upper half-plane, and we obtain  an equivalent 
result:
\begin{equation*}
    D_{F_u}(x-y) = \delta^3(x-y) \frac{1}{2(m + i\epsilon)} e^{i(m + i\epsilon)(x^0 - y^0)}
\end{equation*}
We can combine these two results into one expression using the Heaviside step function
$\theta(t)$:
\begin{equation*}
    D_F(x-y) = \delta^3(x-y) \left[ \theta(t) \frac{e^{-i(m  - i\epsilon)t}}{2(m - i\epsilon)} 
    +  \theta(-t) \frac{e^{i(m + i\epsilon)t}}{2(m + i\epsilon)} \right] \\
\end{equation*}
Taking the limit $\epsilon \to 0$, we find:
\begin{align*}
    D_F(x-y) &= \delta^3(x-y) \left[ \theta(t) \frac{e^{-imt}}{2m} +  \theta(-t)
    \frac{e^{imt}}{2m} \right] \\
    &= \delta^3(x-y) \frac{1}{2m} \left[ \theta(t) e^{-imt} +  \theta(-t) e^{imt}
    \right] \\
    &= \delta^3(x-y) \frac{1}{2m} e^{-im|t|} \\
\end{align*}
\begin{equation}
    \boxed{D_F(x-y) = \delta^3(x-y) \frac{1}{2m} e^{-im|x^0 - y^0|}}
\end{equation}

Now, let us consider the scalar field theory with action 
\begin{equation}
    S =  \int d^{D+1}x \left( \frac{1}{2}\delmu\phi\delMu\phi + |\lambda_2|\phi^2 + \lambda_3\phi^4 \right)
\end{equation}
We always begin by computing the minima of the potential, which  in this case is $|\lambda_2|\phi^2 + \lambda_3\phi^4$. We find:
\begin{align*}
    \frac{d}{d\phi} (|\lambda_2|\phi^2 + \lambda_3\phi^4) &= 2|\lambda_2|\phi + 4\lambda_3\phi^3 = 0 \\
    \phi(2|\lambda_2| + 4\lambda_3\phi^2) &= 0
\end{align*}
Thus, we have three stationary points: $\phi_c = 0$ and $\phi = \pm
\sqrt{\frac{-|\lambda_2|}{2\lambda_3}}$. The following plot shows that $\lambda_3 < 0$ is 
the only case where we have local minima at $\phi = \pm \sqrt{\frac{-|\lambda_2|}{2\lambda_3}}$:
\begin{center}
    %\includegraphics[width=0.8\textwidth]{fig.png}
\end{center}
In the case where the potential has $\lambda_2\phi^2$ instead of
$|\lambda_2|\phi^2$, then when $\lambda_2 < 0$ we have a single minimum at $\phi = 0$,
and when $\lambda_2 > 0$ we have local maxima at $\phi = 0$ and local minima at $\phi = \pm
\sqrt{\frac{-|\lambda_2|}{2\lambda_3}}$.

Now, we can expand the field around one of the minima $\phi_c = \pm
\sqrt{\frac{-|\lambda_2|}{2\lambda_3}}$ by defining a fluctuation field
$\phi = \varphi - \phi_c$. The potential becomes
\begin{equation}
    V(\varphi) = -|\lambda_2|\varphi^2 + \lambda_3 \varphi^4,
\end{equation}
and expanding around $\phi_c$ gives
\begin{align}
    V(\phi_c + \phi)
    &= V(\phi_c) + V'(\phi_c)\phi + \frac{1}{2}V''(\phi_c)\phi^2 + \mathcal{O}(\phi^3) \\
    &= \frac{|\lambda_2|^2}{4\lambda_3} + 2|\lambda_2|\phi^2 + \mathcal{O}(\phi^3),
\end{align}
where we used $V'(\phi_c) = 0$ and $V''(\phi_c) = 4|\lambda_2|$.

Thus, the potential near the minimum is approximately quadratic:
\begin{equation}
    V(\phi) \simeq \frac{|\lambda_2|^2}{4\lambda_3} + 2|\lambda_2|\phi^2.
\end{equation}

The constant term does not affect the dynamics, so the quadratic (free) part of the action becomes
\begin{equation}
    S[\phi] = \frac{1}{2}\int d^{D+1}x\,\left( \partial_\mu\phi\,\partial^\mu\phi - 4|\lambda_2|\,\phi^2 \right).
\end{equation}
By comparison with the canonical Klein–Gordon form
\[
    S = \frac{1}{2}\int d^{D+1}x\,\left( \partial_\mu\phi\,\partial^\mu\phi - m^2\phi^2 \right),
\]
we identify the new mass as
\begin{equation}
    m = 2\sqrt{|\lambda_2|}.
\end{equation}
This way, we can write the Feynmann propagator as 
\begin{equation*}
    D_F(p) = \frac{i}{p^2 - 4|\lambda_2| + i\epsilon}.
\end{equation*}
With simple poles at $p^0 = \pm \sqrt{\mathbf{p}^2 + 4|\lambda_2|} \mp i\epsilon$ and residues:
\begin{align*}
    \text{Res}(D_F, \sqrt{\mathbf{p}^2 + 4|\lambda_2|} - i\epsilon) &= \frac{i}{2(\sqrt{\mathbf{p}^2 + 4|\lambda_2|} - i\epsilon)} \\
    \text{Res}(D_F, -\sqrt{\mathbf{p}^2 + 4|\lambda_2|} + i\epsilon) &= -\frac{i}{2(\sqrt{\mathbf{p}^2 + 4|\lambda_2|} + i\epsilon)}
\end{align*}

To obtain the propagator in coordinate space, we perform the Fourier transform:
\begin{align*}
    D_F(x - y) &= \int \frac{d^{D+1}p}{(2\pi)^{D+1}} \,
    \frac{i\,e^{-ip\cdot(x - y)}}{p^2 - 4|\lambda_2| + i\epsilon}. \\
    D_F(x - y) &=
    \int \frac{d^D\mathbf{p}}{(2\pi)^D} e^{i\mathbf{p}\cdot(\mathbf{x} - \mathbf{y})}
    \int \frac{dp^0}{2\pi} \,
    \frac{i\,e^{-ip^0(x^0 - y^0)}}{(p^0)^2 - (\mathbf{p}^2 + 4|\lambda_2|) + i\epsilon}.
\end{align*}

The integrand has simple poles at
\[
    p^0 = \pm E_{\mathbf{p}} \mp i\epsilon, \qquad
    E_{\mathbf{p}} = \sqrt{\mathbf{p}^2 + 4|\lambda_2|}.
\]
Closing the contouts and using the residue theorem in the same manner as before, we find:
\begin{equation}
    \boxed{D_F(x - y)
    = \int \frac{d^D\mathbf{p}}{(2\pi)^D}
    \frac{e^{i\mathbf{p}\cdot(\mathbf{x} - \mathbf{y})}}{2E_{\mathbf{p}}}
    \left[
        \theta(x^0 - y^0) e^{-iE_{\mathbf{p}}(x^0 - y^0)}
        + \theta(y^0 - x^0) e^{+iE_{\mathbf{p}}(x^0 - y^0)}
    \right]}
\end{equation}

\section*{Question 2: Energy-momentum tensor in field theory}
For a free scalar field theory, we have the Lagrangian density
\begin{align*}
    \mathcal{L} &= \frac{1}{2}\delmu\phi\delMu\phi - \frac{1}{2}m^2\phi^2.\\
    &= \frac{1}{2}\eta^{\mu\nu}\delmu\phi\delnu\phi - \frac{1}{2}m^2\phi^2.\\
\end{align*}
Or, on some arbitrary curved spacetime with metric $g_{\mu\nu}$, we have
\begin{equation*}
    \mathcal{L} = \frac{1}{2}g^{\mu\nu}\delmu\phi\delnu\phi - \frac{1}{2}m^2\phi^2.
\end{equation*}
Running with this, we can compute
\begin{align*}
    -\frac{2}{\sqrt{-g}}\frac{\partial(\sqrt{-g}\mathcal{L})}{\partial g^{\mu\nu}}
    &= -\frac{2}{\sqrt{-g}} \left( \frac{\partial\sqrt{-g}}{\partial g^{\mu\nu}}\mathcal{L} + \sqrt{-g}\frac{\partial\mathcal{L}}{\partial g^{\mu\nu}} \right)\\
\end{align*}
Using the matrix identity $\ln \det A = \text{tr} \ln A$, and varying both sides, we find
\begin{align*}
    \delta(\ln \det A) &= \delta(\text{tr} \ln A) \\
    &= \text{tr}(\delta \ln A) \\
    &= \text{tr} (A^{-1}\delta A) \\
\end{align*}
Therefore, we have
\begin{equation*}
    \frac{\delta(\det A)}{\det A} = \text{tr}(A^{-1}\delta A).
\end{equation*}
Putting this in the language of metrics, we have
\begin{align*}
    \frac{\delta g}{g} &= g^{\mu\nu}\delta g_{\mu\nu} \\
    \delta g &= g g^{\mu\nu}\delta g_{\mu\nu} \\
\end{align*}
Therefore, we can compute
\begin{align*}
    \delta \sqrt{-g} &= -\frac{1}{2\sqrt{-g}}\delta g \\
    &= -\frac{1}{2\sqrt{-g}} g g^{\mu\nu}\delta g_{\mu\nu} \\
    &= -\frac{1}{2}\sqrt{-g} g^{\mu\nu}\delta g_{\mu\nu} \\
    \frac{\partial \sqrt{-g}}{\partial g^{\mu\nu}} &= -\frac{1}{2}\sqrt{-g} \; g_{\mu\nu} \\
\end{align*}
Putting this back into our expression, we find
\begin{align*}
    \frac{2}{\sqrt{-g}}\frac{\partial(\sqrt{-g}\mathcal{L})}{\partial g^{\mu\nu}}
    &= \frac{2}{\sqrt{-g}} \left( -\frac{1}{2}\sqrt{-g} \;
    g_{\mu\nu}\left(\frac{1}{2}\delmu\phi\delnu\phi - \frac{1}{2}m^2\phi^2\right)+
    \sqrt{-g}\left(\frac{1}{2}\delmu\phi\delnu\phi \right)\right)\\
    &= -g_{\mu\nu}\left(\frac{1}{2}\delmu\phi\delnu\phi - \frac{1}{2}m^2\phi^2\right) + \delmu\phi\delnu\phi \\
    &= \delmu\phi\delnu\phi - g_{\mu\nu}\mathcal{L}\\
\end{align*}
Or, swapping back to Minkowski space, we have
\begin{equation*}
    \boxed{T_{\mu\nu} = \delmu\phi\delnu\phi - \eta_{\mu\nu}\mathcal{L}.}
\end{equation*}
Just as was obtained using Noether's theorem.

To couple the free field theory Lagrangian to gravity, we replace the Minkowski
metric with a general metric $g_{\mu\nu}$, and we include a factor of
$\sqrt{-g}$ to ensure that the action is a scalar under general coordinate
transformations. Lastly, we would also have to make the derivatives covariant
instead of partial, but since $\phi$ is a scalar field, the covariant derivative
reduces to the partial derivative.

\section*{Question 3: Complex scalar field and scalar electro-dynamics}
We can imagine $\phi$ and $\phi^{\dagger}$ as a sum of two real fields $\phi_1$ 
and $\phi_2$, such that $\phi = \phi_1 + i\phi_2$ and $\phi^{\dagger} = \phi_1 -
i\phi_2$. Plugging this presecription into our Lagrangian
\begin{align*}
   \mathcal{L} &= \frac{1}{2}\delmu\phi\delMu\phi^{\dagger} \\ 
               &= \frac{1}{2}\delmu(\phi_1 + i \phi_2)\delMu(\phi_1 - i\phi_2) \\
               &= \frac{1}{2}\delmu\phi_1\delMu\phi_1
               + \frac{1}{2}\delmu\phi_2\delMu\phi_2
               + \frac{i}{2}\delmu\phi_2\delMu\phi_1
               - \frac{i}{2}\delmu\phi_1\delMu\phi_2 \\
               &= \frac{1}{2}\delmu\phi_1\delMu\phi_1
               + \frac{1}{2}\delmu\phi_2\delMu\phi_2
               + \frac{i}{2}\delmu\delMu\phi_2\phi_1
               - \frac{i}{2}\delmu\delMu\phi_2\delMu\phi_1 \\
               &= \frac{1}{2}\delmu\phi_1\delMu\phi_1
               + \frac{1}{2}\delmu\phi_2\delMu\phi_2
\end{align*}
because two independent scalar fields $\phi_1$, $\phi_2$ commute. Thus, we can
see this Lagrangian describes two independent scalar field theories. Now,
let us quantize both $\phi_1$ and $\phi_2$ independently, giving them each
their own creation and annihilation operators, which we shall label $\alpha$ 
and $\beta$.
\begin{align*}
   \phi_1(x) &= \int \frac{d^3k}{(2\pi)^3} \frac{1}{\sqrt{2\omega_k}} (\alpha_k
   e^{-ik\cdot x} + \alpha_k^{\dagger}e^{ik\cdot x}) \\
   \phi_2(x) &= \int \frac{d^3k}{(2\pi)^3} \frac{1}{\sqrt{2\omega_k}} (\beta
   e^{-ik\cdot x} + \beta^{\dagger}e^{ik\cdot x}) \\
\end{align*}
Using these definitions, we write $\phi$ and $\phi^{\dagger}$
\begin{align*}
   \phi(x) &= \int \frac{d^3k}{(2\pi)^3} \frac{1}{\sqrt{2\omega_k}} \left(
   (\alpha_k + i\beta_k)e^{-ik\cdot x} + (\alpha^{\dagger}_k +
   i\beta^{\dagger}_k)e^{ik\cdot x} \right) \\
   \phi^{\dagger}(x) &= \int \frac{d^3k}{(2\pi)^3} \frac{1}{\sqrt{2\omega_k}}
   \left( (\alpha_k - i\beta_k)e^{ik\cdot x} + (\alpha^{\dagger}_k -
   i\beta^{\dagger}_k)e^{-ik\cdot x} \right) \\
\end{align*}
let us now define $a_k = \alpha_k + i\beta_k$ and $b_k = \alpha_k - i\beta_k$. Then,
immediately we see $a^{\dagger}_k = \alpha_k^{\dagger} - i\beta_k^{\dagger}$ and 
$b^{\dagger}_k = \alpha_k^{\dagger} + i \beta_k^{\dagger}$ such that

\begin{align*}
   \phi(x) &= \int \frac{d^3k}{(2\pi)^3} \frac{1}{\sqrt{2\omega_k}} \left(
   a_k e^{-ik\cdot x} + b_k^{\dagger}e^{ik\cdot x}
   \right) \\
   \phi^{\dagger}(x) &= \int \frac{d^3k}{(2\pi)^3} \frac{1}{\sqrt{2\omega_k}}
   \left( a^{\dagger}_k e^{ik\cdot x} + b_k e^{-ik\cdot x} \right) \\
\end{align*}
Now, let's compute $[\phi(x), \phi^{\dagger}(y)]$

\begin{align*}
   [\phi(x),\phi^\dagger(y)]
   &=\int \frac{d^3\mathbf{k}}{(2\pi)^3}
          \frac{d^3\mathbf{k}'}{(2\pi)^3}\,
          \frac{1}{\sqrt{2\omega_{\mathbf{k}}}
                   \sqrt{2\omega_{\mathbf{k}'}}}\,
   \Big[
     \big(a_{\mathbf{k}} e^{-ik\cdot x}
      + b_{\mathbf{k}}^\dagger e^{+ik\cdot x}\big),
     \big(a_{\mathbf{k}'}^\dagger e^{+ik'\cdot y}
      + b_{\mathbf{k}'} e^{-ik'\cdot y}\big)
   \Big] \\[6pt]
   &=\int \frac{d^3\mathbf{k}}{(2\pi)^3}
           \frac{d^3\mathbf{k}'}{(2\pi)^3}\,
           \frac{1}{\sqrt{2\omega_{\mathbf{k}}}
                    \sqrt{2\omega_{\mathbf{k}'}}}\,
   \Big(
     a_{\mathbf{k}} a_{\mathbf{k}'}^\dagger e^{-ik\cdot x} e^{+ik'\cdot y}
    +a_{\mathbf{k}} b_{\mathbf{k}'} e^{-ik\cdot x} e^{-ik'\cdot y} \\
   &\qquad\qquad
    +b_{\mathbf{k}}^\dagger a_{\mathbf{k}'}^\dagger e^{+ik\cdot x} e^{+ik'\cdot y}
    +b_{\mathbf{k}}^\dagger b_{\mathbf{k}'} e^{+ik\cdot x} e^{-ik'\cdot y} \\
   &\qquad\qquad
    -a_{\mathbf{k}'}^\dagger a_{\mathbf{k}} e^{+ik'\cdot y} e^{-ik\cdot x}
    -b_{\mathbf{k}'} a_{\mathbf{k}} e^{-ik'\cdot y} e^{-ik\cdot x} \\
   &\qquad\qquad
    -a_{\mathbf{k}'}^\dagger b_{\mathbf{k}}^\dagger e^{+ik'\cdot y} e^{+ik\cdot x}
    -b_{\mathbf{k}'} b_{\mathbf{k}}^\dagger e^{-ik'\cdot y} e^{+ik\cdot x}
   \Big) \\[6pt]
   &=\int \frac{d^3\mathbf{k}}{(2\pi)^3}
           \frac{d^3\mathbf{k}'}{(2\pi)^3}\,
           \frac{1}{\sqrt{2\omega_{\mathbf{k}}}
                    \sqrt{2\omega_{\mathbf{k}'}}}\,
   \Big(
    [a_{\mathbf{k}},a_{\mathbf{k}'}^\dagger]\,e^{-ik\cdot x}e^{+ik'\cdot y}
    +[a_{\mathbf{k}},b_{\mathbf{k}'}]\,e^{-ik\cdot x}e^{-ik'\cdot y} \\
   &\qquad\qquad
    +[b_{\mathbf{k}}^\dagger,a_{\mathbf{k}'}^\dagger]\,e^{+ik\cdot x}e^{+ik'\cdot y}
    +[b_{\mathbf{k}}^\dagger,b_{\mathbf{k}'}]\,e^{+ik\cdot x}e^{-ik'\cdot y}
   \Big) \\[6pt]
   &=\int \frac{d^3\mathbf{k}}{(2\pi)^3}
           \frac{d^3\mathbf{k}'}{(2\pi)^3}\,
           \frac{1}{\sqrt{2\omega_{\mathbf{k}}}
                    \sqrt{2\omega_{\mathbf{k}'}}}\,
   \Big(
    (2\pi)^3\delta^{(3)}(\mathbf{k}-\mathbf{k}')\,e^{-ik\cdot x}e^{+ik'\cdot y}
    -(2\pi)^3\delta^{(3)}(\mathbf{k}-\mathbf{k}')\,e^{+ik\cdot x}e^{-ik'\cdot y}
   \Big) \\[6pt]
   &=\int \frac{d^3\mathbf{k}}{(2\pi)^3}\,
         \frac{1}{2\omega_{\mathbf{k}}}\,
         \Big(e^{-ik\cdot(x-y)} - e^{+ik\cdot(x-y)}\Big). \\
   &= \int \frac{d^3k}{(2\pi)^3} e^{-i\mathbf{k}(\mathbf{x} - \mathbf{y})}
   \left(\frac{1}{2\omega_k} e^{-ik\cdot(x-y)}\bigg|_{k^0 = \omega_k}
   + \frac{1}{-2\omega_k}e^{-ik\cdot(x-y)}\bigg|_{k^0 = -\omega_k}\right) \\
   &= \int \frac{d^3k}{(2\pi)^3} \int \frac{dp^0}{2\pi i} \frac{-1}{p^2 - m^2}
   e^{-ik\cdot (x-y)} \\
   &= \int \frac{d^4k}{(2\pi)^4} \frac{i}{p^2 - m^2} e^{-ik\cdot(x-y)}
\end{align*}
which we immediately recognize as $D_R(x-y)$, the retarded propagator. It makes
sense that the commutator would yield the retarded propagator because the
commutator measures how much the quantum field $\phi$ at one point in spacetime
$x$ interferes with $\phi^{\dagger}$ at a later point $y$, and $D_R(x-y)$
describes what this causal effect is.

Next, to find the Hamiltonian, we will need to compute the conjugate momenta
\begin{align*}
   \pi(x) &= \frac{\partial \mathcal{L}}{\partial(\del_0 \phi)} = \tfrac12\del^0
   \phi^{\dagger} \\
   \pi^{\dagger}(x) &= \frac{\partial \mathcal{L}}{\partial(\del_0
   \phi^{\dagger})} = \tfrac12\del^0 \phi \\
\end{align*}
then, the Hamiltonian density is given by
\begin{align*}
    \mathcal{H} &= \pi \del_0 \phi + \pi^{\dagger} \del_0 \phi^{\dagger} -
    \mathcal{L} \\
    &= \del^0 \phi^{\dagger} \del_0 \phi + \del^0 \phi \del_0
    \phi^{\dagger} - \left( \delmu\phi^{\dagger}\delMu\phi \right) \\
    &= \del^0 \phi^{\dagger} \del_0 \phi + \del^0 \phi \del_0
    \phi^{\dagger} - \left( \del^0\phi^{\dagger}\del_0\phi -
    \nabla\phi^{\dagger}\cdot\nabla\phi \right) \\
    &= \nabla\phi^{\dagger}\cdot\nabla\phi + \del^0 \phi^{\dagger} \del_0 \phi \\
\end{align*}

Now, considering the massive case, we have the Lagrangian
\begin{equation*}
   \mathcal{L} = \delmu\phi^{\dagger}\delMu\phi - \tfrac12 m^2 \phi^{\dagger}\phi
\end{equation*}
If we consider a transformation 
\begin{equation*}
   \phi \;\to\; \phi' \,=\, e^{-i\theta}\phi, \qquad
   \phi^\dagger \;\to\; \phi'^{\dagger} \,=\, e^{+i\theta}\phi^\dagger .
\end{equation*}
then we can tell that if we were to plug this into the Lagrangian, we would
find no change since in both terms, the exponential factors cancel. Further,
expanding the rotated fields in terms of real components, we find
\begin{align*}
   \phi' \,=\, \phi_1'+i\phi_2'
   \;=\; e^{-i\theta}\,(\phi_1+i\phi_2)
   \;=\; (\cos\theta\,\phi_1+\sin\theta\,\phi_2)
   \;+\; i(\cos\theta\,\phi_2-\sin\theta\,\phi_1)
\end{align*}
so
\begin{align*}
   \begin{pmatrix}\phi_1'\\ \phi_2'\end{pmatrix}
   \;=\;
   \begin{pmatrix}
      \cos\theta & \ \ \sin\theta \\
      -\sin\theta & \ \ \cos\theta
   \end{pmatrix}
   \begin{pmatrix}\phi_1\\ \phi_2\end{pmatrix}.
\end{align*}
Thus the global phase $\phi\to e^{-i\theta}\phi$ acts as a rotation by angle $\theta$ in the $(\phi_1,\phi_2)$ plane.
Now, considering a small $\theta$, we can expand the exponentials to first order in $\theta$:
\begin{align*}
   e^{-i\theta} &\simeq 1 - i\theta, \\
   e^{+i\theta} &\simeq 1 + i\theta,
\end{align*}
so
\begin{align*}
    \phi' &= (1 - i\theta)\phi = \phi - i\theta\phi, \\
   \therefore \delta\phi &= -i\theta\,\phi \\
   \phi'^{\dagger} &= (1 + i\theta)\phi^{\dagger} = \phi^{\dagger} + i\theta\phi^{\dagger}, \\
   \therefore \delta\phi^\dagger &= +i\theta\,\phi^\dagger
\end{align*}
Now, we can find the conserved current using Noether's theorem:
\begin{align*}
   J^\mu
   &= \frac{\partial \mathcal{L}}{\partial(\del_\mu \phi)}\,\delta\phi
     \;+\; \frac{\partial \mathcal{L}}{\partial(\del_\mu \phi^\dagger)}\,\delta\phi^\dagger \\
   &= \big(\del^\mu\phi^\dagger\big)\,(-i\theta\,\phi)
     \;+\; \big(\del^\mu\phi\big)\,(+i\theta\,\phi^\dagger) \\
   &= \frac{i\theta}{2}\Big(\phi^\dagger \del^\mu \phi \;-\; (\del^\mu\phi^\dagger)\,\phi\Big).
\end{align*}
Then, the conserved charge $Q$ can be found by integrating the charge density
over space
\begin{align*}
    Q &= \int d^3x J^{0} = \int d^3x \; \frac{i\theta}{2}\Big(\phi^\dagger \del^0 \phi -
    \del^0\phi^\dagger\phi\Big). \\
      &= \int d^3x \frac{i\theta}{2} \Big(\phi^{\dagger}\pi^{\dagger} - \pi\phi \Big) \\
\end{align*}
Knowing that all of $\phi$, $\pi$ (and their daggered counterparts) are quantized,
then we can express the charge in terms of their mode expansions. For this we
can compute $\del^0\phi$ and $\del^0\phi^{\dagger}$. Using 
\begin{align*}
    \del^0 c_k e^{-ik\cdot x} &= c_k \del^0 e^{-ik^0x^0 + i\mathbf{k} \cdot \mathbf{x}} \\
                              &= -i k^0 c_k e^{-ik\cdot x} \\
                              &= -i \omega_k c_k e^{-ik\cdot x}
\end{align*}
Therefore
\begin{align*}
    \tfrac12\del^0\phi &= \pi^{\dagger}(x) = \tfrac12 \int \frac{d^3k}{(2\pi)^3}
    \frac{1}{\sqrt{2\omega_k}} \Big((-i\omega_k) a_k e^{-ik\cdot x} +
    (i\omega_k) b^{\dagger}_k e^{ik\cdot x} \Big) \\
                       &= \frac{-i}{2} \int \frac{d^3k}{(2\pi)^3}
                       \sqrt{\frac{\omega_k}{2}}\Big(a_ke^{-ik\cdot x}
                           - b^{\dagger}_ke^{ik\cdot x} \Big)
\end{align*}
and similarly
\begin{equation*}
    \pi(x) = \frac{i}{2} \int \frac{d^3k}{(2\pi)^3} \sqrt{\frac{\omega_k}{2}}
    \Big(a^{\dagger}_ke^{ik\cdot x} - b_ke^{-ik\cdot x} \Big)
\end{equation*}
And so the conserved charge is given by
\begin{align*}
\hat{Q}
  &= \frac{i}{4} \!\int\! \frac{d^3x\, d^3k\, d^3k'}{(2\pi)^6}
     \sqrt{\frac{\omega_{k'}}{\omega_k}}\,
     \Big[
       \big(a_{k'}^{\dagger} e^{ik'\cdot x} + b_{k'} e^{-ik'\cdot x}\big)
       (-i)\big(a_k e^{-ik\cdot x} - b_k^{\dagger} e^{ik\cdot x}\big)
\\[-0.5em]
  &\quad\quad\;\;
       -\big(a_{k'}^{\dagger} e^{ik'\cdot x} - b_{k'} e^{-ik'\cdot x}\big)
       (i)\big(a_k e^{-ik\cdot x} + b_k^{\dagger} e^{ik\cdot x}\big)
     \Big] \\
  &= \frac{1}{4} \!\int\! \frac{d^3x\, d^3k\, d^3k'}{(2\pi)^6}
     \sqrt{\frac{\omega_{k'}}{\omega_k}}\,
     \Big[
     \big(a^{\dagger}_{k'}a_ke^{i(k' - k)\cdot x} -
         a^{\dagger}_{k'}b^{\dagger}_{k} e^{i(k' + k)\cdot x} + b_{k'}a_k
         e^{i(k + k')\cdot x} - b_{k'}b^{\dagger}_k e^{-i(k - k')\cdot x} \big)
\\[-0.5em]
  &\quad\quad\;\;
       + \big(a^{\dagger}_{k'}a_ke^{-i(k' - k)\cdot x} +
         a^{\dagger}_{k'}b^{\dagger}_{k} e^{-i(k' + k)\cdot x} - b_{k'}a_k
         e^{i(k' + k)\cdot x} - b_{k'}b^{\dagger}_k e^{-i(k - k')\cdot x} \big)
     \Big]
\end{align*}

We now carry out the spatial integral. Each exponential gives either 
$e^{i(\mathbf{k}'-\mathbf{k})\cdot\mathbf{x}}$ or 
$e^{i(\mathbf{k}'+\mathbf{k})\cdot\mathbf{x}}$, so

\[
\int d^3x\, e^{i(\mathbf{k}'-\mathbf{k})\cdot\mathbf{x}} = (2\pi)^3\delta^{(3)}(\mathbf{k}'-\mathbf{k}),
\qquad
\int d^3x\, e^{i(\mathbf{k}'+\mathbf{k})\cdot\mathbf{x}} = (2\pi)^3\delta^{(3)}(\mathbf{k}'+\mathbf{k}).
\]
\begin{align*}
\hat{Q}
  &= \frac{1}{4} \!\int\! \frac{d^3x\, d^3k\, d^3k'}{(2\pi)^6}
     \sqrt{\frac{\omega_{k'}}{\omega_k}}\,
     \Big[
     \big(a^{\dagger}_{k}a_k - a^{\dagger}_{-k}b^{\dagger}_{k} + b_{-k}a_k -
     b_{k}b^{\dagger}_k \big)
\\[-0.5em]
  &\quad\quad\;\;
       + \big(a^{\dagger}_{k}a_k + a^{\dagger}_{-k}b^{\dagger}_{k} - b_{-k}a_k
       - b_{k}b^{\dagger}_k  \big)
     \Big] \\
&= \frac{1}{4}\!\int\!\frac{d^3k}{(2\pi)^3}
\Big[
    2a_{k}^{\dagger} a_k  - 2b_kb^{\dagger}_k
\Big] \\
&= \frac{1}{2}\!\int\!\frac{d^3k}{(2\pi)^3}
\Big(
    a_{k}^{\dagger} a_k  - b_kb^{\dagger}_k
\Big).
\end{align*}
Using $b_k b_k^{\dagger} = b_k^{\dagger}b_k + (2\pi)^3\delta^{(3)}(0)$, we
normal order and drop the infinite constant.
\[
:\hat{Q}: = \frac{1}{2} \int\!\frac{d^3k}{(2\pi)^3}\,
\big(a_k^{\dagger}a_k - b_k^{\dagger}b_k\big).
\]
Thus the conserved charge counts particles minus antiparticles.

Now, if the rotational angle is space dependent $\theta = \theta(x)$, we see
that the Lagrangian is no longer invariant under $\phi \to e^{-i\theta(x)}\phi$
because the first term
\begin{align*}
    \delmu\phi\delMu\phi^{\dagger} &\to
    \delmu(e^{-i\theta(x)}\phi)\delMu(e^{i\theta(x)}\phi) \\ 
                                   &=
    \big(-i\delmu\theta(x)e^{-i\theta(x)}\phi + e^{-i\theta(x)}\delmu\phi \big)
    \big(i\delmu\theta(x)e^{i\theta(x)}\phi^{\dagger} +
    e^{i\theta(x)}\delmu\phi^{\dagger} \big) \\
                                   &=
    \big(
        (\delmu\theta)^2\phi\phi^{\+} -i\delmu\theta\phi\delmu\phi^{\+}
        + i\delmu\phi\delmu\theta\phi^{\+} + \delmu\phi\delmu\phi^{\+}
    \big)
\end{align*}
For a small angle change, we can ignore the $(\delmu\theta)^2$ term, leaving us
with our new Lagrangian:
\begin{equation*}
    \mathcal{L'} = 
    \frac{1}{2}\delmu\phi\delmu\phi^{\+} -
    \frac{i}{2}\delmu\theta(\phi\delmu\phi^{\+} - \delmu\phi\phi^{\+})
    - \frac{1}{2}m^2\phi\phi^{\+}
\end{equation*}
Which is obviously different than prior to the transformation. Recalling that 
the conserved current for the global gauge transformation was $J^{\mu} =
\frac{i}{2}\big(\phi^{\+}\delmu\phi - \delmu\phi^{\+}\phi\big)$, we can 
read off that the variation in the Lagrangian is
\begin{align*}
    \delta\mathcal{L} &= \mathcal{L'} - \mathcal{L} = 
    - \frac{i}{2}\delmu\theta(\phi\delmu\phi^{\+} - \delmu\phi\phi^{\+}) \\
                     &= \delmu\theta J^{\mu}
\end{align*}
as desired.

To show how the current is not invariant, take:
\begin{align*}
    J'^{\mu} &\to \frac{i}{2}\big(e^{-i\theta(x)}\phi \delmu
    (e^{i\theta(x)}\phi^{\+}) - \delmu (e^{-i\theta(x)}\phi)e^{i\theta(x)}\phi^{\+} \big) \\
            &= \frac{i}{2} \big(
                e^{-i\theta(x)}\phi(i\delmu\theta(x)e^{i\theta(x)}\phi^{\+} + 
                e^{i\theta(x)}\delmu\phi^{\+}) -
                (-i\delmu\theta(x)e^{-i\theta(x)}\phi +
                e^{-i\theta(x)}\delmu\phi)e^{i\theta(x)}\phi^{\+}
            \big) \\
            &= \frac{i}{2} \big(
                i\delmu\theta\phi\phi^{\+} + \phi\delmu\phi^{\+} +
                i\delmu\theta\phi\phi^{\+} - \delmu\phi\phi^{\+}
            \big) \\
            &= \frac{i}{2}\big(\phi\delmu\phi^{\+} - \delmu\phi\phi^{\+}\big) 
            + |\phi|^2\delmu\theta \\
\end{align*}
Such that
\begin{align*}
    \delta J^{\mu} &= |\phi|^2\delmu\theta
\end{align*}
Now, let us consider the same transformation $\phi \to e^{-i\theta(x)}\phi$ but with our 
modified Lagrangian
\begin{align*}
    \mathcal{L} &= \frac{1}{2}\delmu\phi^{\+}\delMu\phi - eJ^{\mu}A_{\mu} + \frac{e^2}{2}
    |\phi|^2A_{\mu}A^{\mu} - \frac{1}{2}m^2\phi^{\+}\phi  \\
\end{align*}
Then the variation in the Lagrangian is
\begin{align*}
    \delta\mathcal{L} &= \delta \mathcal{L}_{\text{old}} + \delta \mathcal{L}_{\text{new}} \\
    &= \delmu\theta  J^{\mu} - e \delta J^{\mu} A_{\mu} - e J^{\mu} \delta A_{\mu} + 
    \frac{e^2}{2} |\phi|^2 \delta A_{\mu}A^{\mu} + \frac{e^2}{2} |\phi|^2 A_{\mu} \delta A^{\mu} \\
    &= \delmu\theta  J^{\mu} - e (|\phi|^2 \delmu\theta) A_{\mu} - e J^{\mu} \delta A_{\mu} + 
    e^2 |\phi|^2 \delta A_{\mu}A^{\mu} \\
    &= \delmu\theta \big(J^{\mu} -e|\phi|^2 A^{\mu} \big) - \big(eJ^{\mu} - e^2
    |\phi|^2 A^{\mu} \big) \delta A_{\mu} \\
    &= \delmu\theta \big(J^{\mu} -e|\phi|^2 A^{\mu} \big) - \big(eJ^{\mu} - e^2
    |\phi|^2 A^{\mu} \big) \Big(\frac{1}{e} \delmu\theta  \Big) \\
    &= \delmu\theta \big(J^{\mu} -e|\phi|^2 A^{\mu} \big) - \delmu\theta\big(J^{\mu} - e
    |\phi|^2 A^{\mu} \big)\\
    &= 0
\end{align*}
and so we see that if we choose $\delta A_{\mu} = \frac{1}{e}\delmu\theta$, then
the variation in the Lagrangian is zero, and thus the Lagrangian is invariant
under the local gauge transformation.

To verify the kinetic term for the gauge field is also invariant, we note that,  using
the result above, $A_{\mu}$ transforms as $A_{\mu} \to A_{\mu}' = A_{\mu} +
\frac{1}{e}\delmu\theta$. Therefore:
\begin{align*}
    \delmu A_{\nu}  -  \delnu A_{\mu} &\to  \delmu (A_{\nu} + \frac{1}{e}\delnu\theta) -
    \delnu (A_{\mu} + \frac{1}{e}\delmu\theta) \\
    &= \delmu A_{\nu} - \delnu A_{\mu} + \frac{1}{e}(\delmu\delnu\theta - \delnu\delmu\theta) \\
    &= \delmu A_{\nu} - \delnu A_{\mu} \\
\end{align*}
since partial derivatives commute. Thus the kinetic term for the gauge field is invariant. Specifically,
if $\left(\delmu A_{\nu}  -  \delnu A_{\mu}\right)$ is invariant, then so is 
$\left(\delmu A_{\nu}  -  \delnu A_{\mu}\right)^2$.

Meanwhile, if we have some mass term such as $\frac{1}{2}m^2 A_{\mu}A^{\mu}$,
then under the gauge transformation
\begin{align*}
    A_{\mu}A^{\mu} &\to (A_{\mu} + \frac{1}{e}\delmu\theta)(A^{\mu} +
    \frac{1}{e}\delmu\theta) \\
                   &= A_{\mu}A^{\mu} + \frac{2}{e}A^{\mu}\delmu\theta +
                   \frac{1}{e^2}(\delmu\theta)^2 \\
\end{align*}
which is not equal to the original term, and so the mass term is not gauge invariant.

Now, let us introduce the covariant derivative $D_{\mu} = \delmu - ieA_{\mu}$. Then,
under the gauge transformation, we have
\begin{align*}
    D_{\mu}\phi &= (\delmu - ieA_{\mu})\phi \\
                &\to (\delmu - ie(A_{\mu} + e^{-1}\delmu\theta))e^{-i\theta}\phi \\
                &= \delmu(e^{-i\theta}\phi) - ieA_{\mu}e^{-i\theta}\phi - i\delmu\theta e^{-i\theta}\phi \\
                &= i\delmu\theta e^{-i\theta}\phi + e^{-i\theta}\delmu\phi -
                ieA_{\mu}e^{-i\theta}\phi - i\delmu\theta e^{-i\theta}\phi \\
                &=  e^{-i\theta}\delmu\phi - ieA_{\mu}e^{-i\theta}\phi \\
                &=  \big(\delmu\phi - ieA_{\mu}\phi \big)e^{-i\theta}\phi \\
                &=  e^{-i\theta}D_{\mu}\phi \\
\end{align*}
Thus, the covariant derivative of $\phi$ transforms in the same way as $\phi$ and not
as its derivative.

Finally, to write the QED Lagrangian using the covariant derivative, we start with
a well known definition for the electromagnetic field strength tensor:
\begin{equation*}
    F_{\mu\nu} = \delmu A_{\nu} - \delnu A_{\mu}
\end{equation*}
Then, the QED Lagrangian can be written as
\begin{equation*}
    \mathcal{L}_{\text{QED}} = \frac{1}{2}(D_{\mu}\phi)(D^{\mu}\phi^{\dagger}) -
    \frac{1}{2}m^2\phi\phi^{\dagger} - \frac{1}{4}F_{\mu\nu}F^{\mu\nu}
\end{equation*}

\section*{Question 4: Quantum field theory in 2 + 1 dimensions}
\subsection*{Computing the mode expansions}
We start from the free real scalar field action in $(2+1)$ dimensions:
\begin{equation}
S[\phi] = \frac{1}{2}\int \! d^3x \, 
    \big( \partial_\mu \phi \, \partial^\mu \phi - m^2 \phi^2 \big).
    \label{eq:action}
\end{equation}
The corresponding Euler--Lagrange equation is the Klein-Gordon equation:
\begin{equation}
    (\partial_t^2 - \nabla^2 + m^2)\,\phi = 0.
    \label{eq:KG}
\end{equation}

\subsubsection*{Plane}
For the two dimensional square sheet (Surface A), which I will define to have
the bottom left corner at $(0, 0)$ and length $L$, we have boundary conditions
$\phi(x = 0) = \phi(x = L) = \phi(y = 0) = \phi(y = L) = 0$.
The Laplacian eigenfunctions satisfying these boundary conditions are separable:
\[
    \phi(t,x,y) = f(t)\, \sin(p_x x)\, \sin(p_y y),
\]
where $p_x$ and $p_y$ are quantized by the boundary conditions. Imposing
$\sin(p_x L)=0$ and $\sin(p_y L)=0$ gives
\begin{equation}
p_x = \frac{m\pi}{L}, 
\qquad
p_y = \frac{n\pi}{L},
\qquad
m,n \in \mathbb{Z}.
\label{eq:pxpy}
\end{equation}
Substituting the separable ansatz
\[
\phi(t,x,y) = f(t)\,\sin(p_x x)\,\sin(p_y y)
\]
into the Klein--Gordon equation \eqref{eq:KG},
\[
(\partial_t^2 - \nabla^2 + m^2)\phi = 0,
\]
we evaluate each term separately.
\[
\partial_t^2 \phi = \ddot{f}(t)\,\sin(p_x x)\,\sin(p_y y).
\]
\[
\nabla^2 \phi 
= \left(\frac{\partial^2}{\partial x^2} + \frac{\partial^2}{\partial y^2}\right)
  \big[f(t)\sin(p_x x)\sin(p_y y)\big]
= -f(t)\,(p_x^2 + p_y^2)\,\sin(p_x x)\,\sin(p_y y),
\]
Putting everything together:
\[
(\partial_t^2 - \nabla^2 + m^2)\phi
= \Big[\ddot{f}(t) + (p_x^2 + p_y^2 + m^2)f(t)\Big]
   \sin(p_x x)\sin(p_y y).
\]

For the equation to hold for all \(x,y\), the prefactor of the sine functions
must vanish, giving the ordinary differential equation
\[
\ddot{f}(t) + \w_p^2 f(t) = 0,
\]
with
\begin{equation}
\w_p^2 = p_x^2 + p_y^2 + m^2.
\label{eq:freq}
\end{equation}
The general solution of this ordinary differential equation is
\begin{equation}
f(t) = a_p e^{-i\w_p t} + b_p e^{+i\w_p t}.
\label{eq:ft}
\end{equation}

Putting the spatial and temporal parts together, the complete field can be
written as a double sum over discrete momentum modes:
\begin{equation}
\phi(t,x,y) = \sum_{m,n} \sin(p_x x)\sin(p_y y) \big( a_p e^{-i\w_p t}
+ b_p e^{+i\w_p t} \big),
\label{eq:modeexp}
\end{equation}
Since we want $\phi = \phi^{\dagger}$, then we find that $b_p =
a_p^{\dagger}$

Next, we require that $[\phi(x), \pi(y)] = i\delta^2(x - y)$ and so we 
must compute $\pi(y)$. This is given by
\begin{align*}
    \pi(y) &= \frac{\mathcal{L}}{\del_0\phi} = \del_0\phi \\
           &= \sum_{m,n} \sin(p_x x)\sin(p_y y) \big((-i\w_p)
               a_p e^{-i\w_p t} +
               (i\w_p)a_p^{\dagger}e^{+i\w_p t}
       \big) \\
           &= -i \sum_{m,n} \w_p \sin(p_x x)\sin(p_y y) \big(
               a_p e^{-i\w_p t} - a_p^{\dagger}e^{+i\w_p t}
       \big)
\end{align*}
Now, we compute the commutator. Let us first define $u_n(x) =
\sin\left(\frac{n\pi x}{L}\right)$. Then:
\begin{align*}
    [\phi(x), \pi(y)] &= -i \sum_{m,n} \sum_{j,k} \w_q u_n(x) u_m(y)
    u_j(x') u_k(y') \Big((a_p e^{-i\w_p t} +
        a_p^{\+} e^{i\w_p t})(a_q e^{-i\w_q t} -
        a_q^{\+} e^{i\w_q t})\Big) \\[1em]
    &\quad
    + i \sum_{m,n} \sum_{j,k} \w_q u_n(x) u_m(y) u_j(x') u_k(y')
    \Big((a_q e^{-i\w_q t} - a_q^{\+} e^{i\w_q
    t})(a_p e^{-i\w_p t} + a_p^{\+} e^{i\w_p t})\Big) \\
    &= 
    i \sum_{m,n} \sum_{j,k} \w_q u_n(x) u_m(y) u_j(x') u_k(y') \times
    \Big(
        (a_q e^{-i\w_q t} - a_q^{\+} e^{i\w_q t})
        (a_p e^{-i\w_p t} + a_p^{\+} e^{i\w_p t}) - \\[1em]
    &\quad
        (a_p e^{-i\w_p t} + a_p^{\+} e^{i\w_p t})
        (a_q e^{-i\w_q t} - a_q^{\+} e^{i\w_q t})
    \Big) \\
    &= i \sum_{m,n} \sum_{j,k} \w_q u_n(x) u_m(y) u_j(x') u_k(y') \Big(
        (a_qa_p e^{-i\w_qt} e^{-i\w_p t} - a_pa_q
        e^{-i\w_pt} e^{-i\w_q t}) +  \\[1em]
    &\quad
        (a_qa_p^{\+} e^{-i\w_q t} e^{i\w_p t} - 
        a_p^{\+}a_q e^{i\w_p t} e^{-i\w_q t}) - 
        (a_pa_q^{\+} e^{-i\w_p t} e^{i\w_q t} -
        a_q^{\+}a_p e^{i\w_q t} e^{-i\w_p t} ) + \\[1em]
    &\quad
        (a_q^{\+}a_p^{\+} e^{i\w_q t} e^{-i\w_p t} -
        a_q^{\+}a_p^{\+} e^{i\w_q t} e^{-i\w_p t})
    \Big) \\
    &= i \sum_{m,n} \sum_{j,k} \w_q u_n(x) u_m(y) u_j(x') u_k(y') \Big(
        [a_q, a_p]e^{-it(\w_q + \w_p)} + [a_q,
        a_p^{\+}]e^{-it(\w_q - \w_p)} - \\[1em]
    &\quad
        [a_p, a_q^{\+}] e^{-it(\w_p - \w_q)} +
        [a_q^{\+}, a_p^{\+}] e^{-it(\w_p - \w_q)}
    \Big)
\end{align*}
And using $[a_q, a_q] = [a_q^{\+}, a_p^{\+}] = 0$ 
\begin{align*}
    [\phi(t,\mathbf{x}),\pi(t,\mathbf{x}')]
    &= i \sum_{m,n}\sum_{j,k} \w_{jk}\,
    u_n(x)u_m(y)u_j(x')u_k(y') \\
    &\qquad\times\Big([a_{jk},a_{mn}^\dagger]\,e^{-i(\w_{jk}-\w_{mn})t}
    -[a_{mn},a_{jk}^\dagger]\,e^{-i(\w_{mn}-\w_{jk})t}\Big) \\
    &= i \sum_{m,n} 2\,\w_{mn}\,C_{mn}\,
    u_n(x)u_n(x')\,u_m(y)u_m(y')
\end{align*}
where we have set
\[
[a_{mn},a_{m'n'}^\dagger]=C_{mn}\,\delta_{mm'}\delta_{nn'}
\]
so that the phases cancel when $m=m',\,n=n'$ and the two mixed terms contribute equally.

Using completeness twice,
\[
\sum_{n=1}^\infty u_n(x)u_n(x')=\frac{L}{2}\delta(x-x'),
\qquad
\sum_{m=1}^\infty u_m(y)u_m(y')=\frac{L}{2}\delta(y-y'),
\]
we obtain
\[
[\phi(t,\mathbf{x}),\pi(t,\mathbf{x}')]
= i\,(L/2)^2\!\left(\sum_{m,n} 2\,\w_{mn}\,C_{mn}\right)\delta(x-x')\delta(y-y').
\]
Imposing the canonical equal-time commutator on the box,
\[
[\phi(t,\mathbf{x}),\pi(t,\mathbf{x}')]=i\,\delta(x-x')\delta(y-y'),
\]
forces the coefficient of $\delta(x-x')\delta(y-y')$ to be $1$ mode by mode. Hence
\[
2\,\w_{mn}\,C_{mn}=\frac{1}{(L/2)^2}
\quad\Longrightarrow\quad
C_{mn}=\frac{1}{(L/2)^2\,2\,\w_{mn}}.
\]
Equivalently, if we define rescaled mode operators
\[
\tilde{a}_{mn}\;\equiv\;(L/2)\,\sqrt{2\,\w_{mn}}\;a_{mn},
\]
then their algebra is
\[
[\tilde{a}_{mn},\tilde{a}_{m'n'}^{\dagger}]
=(L/2)^2\,(2\,\w_{mn})\,\delta_{mm'}\delta_{nn'},
\]
and the normalized mode expansion is
\[
\phi(t,x,y)=\sum_{m,n} \frac{1}{(L/2)\sqrt{2\,\w_{mn}}}\,\sin(p_x x)\sin(p_y y)
\Big(\tilde{a}_{mn}\,e^{-i\w_{mn}t}+\tilde{a}_{mn}^{\dagger}\,e^{+i\w_{mn}t}\Big).
\]

\subsubsection*{Cylinder}
Now, for the cylinder, the same boundary exist for for $\phi(y = 0) = \phi(y =
L) = 0$ but now we impose instead $\phi(x = 0) = \phi(x = 2\pi R)$

A convenient separable basis is
\[
u_{m}(x)=e^{i p_x x},\qquad p_x=\frac{m}{R},\quad m\in\mathbb{Z},\qquad
v_{n}(y)=\sin\!\Big(\frac{n\pi y}{L}\Big),\quad n\in\mathbb{Z}_+.
\]
These obey the orthogonality/completeness relations
\begin{align*}
\int_0^{2\pi R}\!dx\,e^{i(p_x-p_x')x}&=2\pi R\,\delta_{m,m'},\\
\int_0^{L}\!dy\,\sin\!\Big(\frac{n\pi y}{L}\Big)\sin\!\Big(\frac{n'\pi y}{L}\Big)
&=\frac{L}{2}\,\delta_{n,n'},\\
\sum_{m\in\mathbb{Z}}\frac{e^{i m(x-x')/R}}{2\pi R}&=\delta(x-x'),\qquad
\sum_{n=1}^\infty \frac{2}{L}\,\sin\!\Big(\frac{n\pi y}{L}\Big)\sin\!\Big(\frac{n\pi y'}{L}\Big)=\delta(y-y').
\end{align*}

As we saw for the sheet, plug a separable ansatz into the Klein-Gordon equation. With
\[
\phi(t,x,y)=f_{mn}(t)\,u_m(x)\,v_n(y),
\]
one finds the same oscillator equation
\[
\ddot f_{mn}(t)+\omega_{mn}^2 f_{mn}(t)=0,\qquad 
\omega_{mn}^2=p_x^2+p_y^2+m^2=\Big(\frac{m}{R}\Big)^2+\Big(\frac{n\pi}{L}\Big)^2+m^2,
\]
so
\[
f_{mn}(t)=\beta_{mn}\,e^{-i\omega_{mn}t}+\beta_{mn}^{\dagger}\,e^{+i\omega_{mn}t}.
\]
Reality of the field enforces the usual relation between opposite momenta. With
the current basis it is convenient to keep $\beta_{mn}$ and $\beta_{mn}^\dagger$
and let the commutator fix the normalization.

Thus the mode expansion is
\begin{equation}
\phi(t,x,y)=\sum_{m\in\mathbb{Z}}\sum_{n=1}^{\infty} 
v_n(y)\Big(\beta_{mn}\,e^{-i(\omega_{mn}t-p_x x)}+\beta_{mn}^{\dagger}\,e^{+i(\omega_{mn}t-p_x x)}\Big).
\label{eq:cyl-mode-raw}
\end{equation}

The canonical momentum is $\pi=\partial_t\phi$, so
\[
\pi(t,x,y)=\sum_{m,n} v_n(y)\Big(-i\omega_{mn}\,\beta_{mn}\,e^{-i(\omega_{mn}t-p_x x)}+i\omega_{mn}\,\beta_{mn}^{\dagger}\,e^{+i(\omega_{mn}t-p_x x)}\Big).
\]
Again, impose
\[
[\phi(t,\mathbf{x}),\pi(t,\mathbf{x}')]=i\,\delta(x-x')\delta(y-y').
\]
\begin{align*}
[\phi(t,x,y),\pi(t,x',y')]
&=\sum_{m,n}\sum_{r,s}
\sin\!\Big(\frac{n\pi y}{L}\Big)\sin\!\Big(\frac{s\pi y'}{L}\Big)\\
&\quad\times\Big(
\;\;(+i\omega_{rs})\,e^{-i(p_m x-p_r x')}\,[\beta_{mn},\beta_{rs}^{\dagger}]
\;+\;(-i\omega_{rs})\,e^{+i(p_m x-p_r x')}\,[\beta_{mn}^{\dagger},\beta_{rs}]
\Big).
\end{align*}

Now use the standard completeness relations on the cylinder:
\[
\sum_{m\in\mathbb{Z}} e^{+ip_m\Delta x}=2\pi R\,\delta(\Delta x),\qquad
\sum_{n=1}^{\infty}\sin\!\Big(\frac{n\pi y}{L}\Big)\sin\!\Big(\frac{n\pi y'}{L}\Big)=\frac{L}{2}\,\delta(y-y').
\]

thus,
\begin{align*}
[\phi(t,x,y),\pi(t,x',y')]
&=i\,(2\pi R)\,(L/2)\sum_{m,n}\omega_{mn}\,[\beta_{mn},\beta_{mn}^{\dagger}]\,\delta(x-x')\delta(y-y').
\end{align*}

This alone only constrains the sum over modes. To get each mode separately, 
Multiply both sides of the above by
\[
\frac{1}{(2\pi R)}\int_{0}^{2\pi R}\!dx\,e^{-i p_\ell x}\,
\frac{2}{L}\int_{0}^{L}\!dy\,\sin\!\Big(\frac{n'\pi y}{L}\Big)\;
\frac{1}{(2\pi R)}\int_{0}^{2\pi R}\!dx'\,e^{+i p_\ell x'}\,
\frac{2}{L}\int_{0}^{L}\!dy'\,\sin\!\Big(\frac{n'\pi y'}{L}\Big)
\]
And use completeness/orthogonality on the cylinder again.
The left-hand side (from the canonical commutator \(i\,\delta(x-x')\delta(y-y')\))
cancels with the deltas on the RHS and leaves just.
\(i\).  The right-hand side collapses the double sum to the single \((\ell,n')\)
mode and produces the extra  factor of \(2\) coming from \(\partial_t\) (i.e.
the \(\beta\beta^\dagger-\beta^\dagger\beta\) pair):
\[
i \;\;=\;\; i\,(2\pi R)\Big(\frac{L}{2}\Big)\,\big(2\,\omega_{\ell n'}\big)\,
[\beta_{\ell n'},\beta_{\ell n'}^{\dagger}]\,.
\]
Therefore,
\[
[\beta_{mn},\beta_{m'n'}^{\dagger}]
=\frac{1}{(2\pi R)\,(L/2)\,(2\,\omega_{mn})}\;
\delta_{m m'}\delta_{n n'}
=\frac{1}{2\pi R\,L\,\omega_{mn}}\;\delta_{m m'}\delta_{n n'}\,,
\]
and
\[
[\beta_{mn},\beta_{m'n'}]=[\beta_{mn}^{\dagger},\beta_{m'n'}^{\dagger}]=0.
\]
Defining
\[
b_{mn}\;\equiv\;\frac{\beta_{mn}}{\sqrt{2\pi R\,L\,\omega_{mn}}}\,,
\qquad
b_{mn}^\dagger\;\equiv\;\frac{\beta_{mn}^\dagger}{\sqrt{2\pi R\,L\,\omega_{mn}}}\,,
\]
one immediately finds
\[
[b_{mn},b_{m'n'}^\dagger]=\delta_{m m'}\delta_{n n'}\,,
\]
and the standard normalized mode expansion for \(\phi\).
\begin{equation}
\phi(t,x,y)=\frac{1}{\sqrt{\pi R L}}\sum_{m\in\mathbb{Z}}\sum_{n=1}^{\infty}
\frac{\sin\!\big(\tfrac{n\pi y}{L}\big)}{\sqrt{2\omega_{mn}}}
\Big( b_{mn}\,e^{-i(\omega_{mn}t-p_x x)}+b_{mn}^{\dagger}\,e^{+i(\omega_{mn}t-p_x x)}\Big),
\label{eq:cyl-mode-final}
\end{equation}

\subsubsection*{Torus}
Finally, for the torus, we impose periodic boundary conditions in both directions.
\[
\phi(t,x+2\pi R_x,y)=\phi(t,x,y),\qquad
\phi(t,x,y+2\pi R_y)=\phi(t,x,y).
\]
Use the separable plane-wave basis
\[
u_m(x)=e^{i p_x x},\quad p_x=\frac{m}{R_x},\quad m\in\mathbb{Z},\qquad
v_n(y)=e^{i p_y y},\quad p_y=\frac{n}{R_y},\quad n\in\mathbb{Z}.
\]
Orthogonality/completeness:
\begin{align*}
\int_0^{2\pi R_x}\!dx\,e^{i(p_x-p_x')x}&=2\pi R_x\,\delta_{m,m'},&
\sum_{m\in\mathbb{Z}}\frac{e^{i m(x-x')/R_x}}{2\pi R_x}&=\delta(x-x'),\\
\int_0^{2\pi R_y}\!dy\,e^{i(p_y-p_y')y}&=2\pi R_y\,\delta_{n,n'},&
\sum_{n\in\mathbb{Z}}\frac{e^{i n(y-y')/R_y}}{2\pi R_y}&=\delta(y-y').
\end{align*}

As before, with the ansatz $\phi(t,x,y)=f_{mn}(t)\,u_m(x)\,v_n(y)$, the Klein–Gordon equation
\[
(\partial_t^2-\nabla^2+m^2)\phi=0
\]
yields
\[
\ddot f_{mn}(t)+\omega_{mn}^2 f_{mn}(t)=0,\qquad
\omega_{mn}^2=p_x^2+p_y^2+m^2=\Big(\frac{m}{R_x}\Big)^2+\Big(\frac{n}{R_y}\Big)^2+m^2,
\]
so
\[
f_{mn}(t)=\beta_{mn}\,e^{-i\omega_{mn}t}+\beta_{mn}^\dagger\,e^{+i\omega_{mn}t}.
\]
(Reality implies $\beta_{mn}^\dagger=\beta_{-m,-n}$; we keep
$\beta,\beta^\dagger$ and let the commutator fix the normalization.)

Thus the mode expansion is
\begin{equation}
\phi(t,x,y)=\sum_{m,n\in\mathbb{Z}}
\Big(\beta_{mn}\,e^{-i(\omega_{mn}t-p_x x-p_y y)}+\beta_{mn}^{\dagger}\,e^{+i(\omega_{mn}t-p_x x+p_y y)}\Big),
\label{eq:torus-phi-raw}
\end{equation}
and $\pi=\partial_t\phi$,
\[
\pi(t,x,y)=\sum_{m,n}
\Big(-i\omega_{mn}\beta_{mn}\,e^{-i(\omega_{mn}t-p_x x-p_y y)}
+i\omega_{mn}\beta_{mn}^{\dagger}\,e^{+i(\omega_{mn}t-p_x x+p_y y)}\Big).
\]

Again, impose
\[
[\phi(t,\mathbf{x}),\pi(t,\mathbf{x}')]=i\,\delta(x-x')\delta(y-y').
\]
A direct computation gives, using the periodic completeness,
\[
[\phi(t,x,y),\pi(t,x',y')]
=i\,(2\pi R_x)\,(2\pi R_y)\sum_{m,n}\omega_{mn}\,[\beta_{mn},\beta_{mn}^{\dagger}]\,\delta(x-x')\delta(y-y').
\]
To isolate a single mode $(\ell,k)$, multiply both sides by
\[
\frac{1}{2\pi R_x}\!\int_0^{2\pi R_x}\!dx\,e^{-ip_\ell x}\;
\frac{1}{2\pi R_y}\!\int_0^{2\pi R_y}\!dy\,e^{-iq_k y}\;
\frac{1}{2\pi R_x}\!\int_0^{2\pi R_x}\!dx'\,e^{+ip_\ell x'}\;
\frac{1}{2\pi R_y}\!\int_0^{2\pi R_y}\!dy'\,e^{+iq_k y'},
\]
with $q_k=k/R_y$. The LHS evaluates to $i$, while the RHS collapses the double sum to $(\ell,k)$ and picks up the usual extra factor $2$ from the $\beta\beta^\dagger-\beta^\dagger\beta$ pair, yielding
\[
i\;=\;i\,(2\pi R_x)(2\pi R_y)\,(2\,\omega_{\ell k})\,[\beta_{\ell k},\beta_{\ell k}^{\dagger}].
\]
therefore
\begin{equation}
[\beta_{mn},\beta_{m'n'}^{\dagger}]
=\frac{1}{(2\pi R_x)(2\pi R_y)\,(2\,\omega_{mn})}\;\delta_{m m'}\delta_{n n'}
=\frac{1}{4\pi^2 R_x R_y\,\omega_{mn}}\;\delta_{m m'}\delta_{n n'},
\label{eq:torus-raw-algebra}
\end{equation}
and $[\beta,\beta]=[\beta^\dagger,\beta^\dagger]=0$.

Define
\[
b_{mn}\equiv \frac{\beta_{mn}}{\sqrt{(2\pi R_x)(2\pi R_y)\,\omega_{mn}}},\qquad
[b_{mn},b_{m'n'}^\dagger]=\delta_{mm'}\delta_{nn'}.
\]
Then
\begin{equation}
\phi(t,x,y)=\frac{1}{\sqrt{(2\pi R_x)(2\pi R_y)}}
\sum_{m,n\in\mathbb{Z}}\frac{1}{\sqrt{2\omega_{mn}}}
\Big(b_{mn}\,e^{-i(\omega_{mn}t-p_x x-p_y y)}+b_{mn}^\dagger\,e^{+i(\omega_{mn}t-p_x x+p_y y)}\Big),
\end{equation}

\subsection*{Propagators (From mode expansions)}
\subsubsection*{Plane}
Using the mode expansion for the plane, we have: 
\begin{align*}
    \bra{0} \phi(t,x,y)\phi(t',x',y') \ket{0} &= \frac{4}{L^2} \bra{0} \Big[
        \sum_{m,n} \frac{\sin(p_x x)\sin(p_y y)}{\sqrt{2\omega_p}} \big(
            a_p e^{-i\omega_p t} + a_p^{\dagger} e^{i\omega_p t} 
        \big)
    \Big] \\[1em]
    &\quad \Big[
        \sum_{m',n'} \frac{\sin(p_{x'} x')\sin(p_{y'} y')}{\sqrt{2\omega_{p'}}} \big(
            a_{p'} e^{-i\omega_{p'} t'} + a_{p'}^{\dagger} e^{i\omega_{p'} t'} 
        \big)
    \Big]   
    \ket{0}
\end{align*}
where we assume $t > t'$. Expanding the products, we see that only the term
with $a_p a_{p'}^{\dagger}$ will contribute, since $a_p \ket{0} = 0$ and
$\bra{0} a_{p'}^{\dagger} = 0$. Thus:
\begin{align*}
    \bra{0} \phi(t,x,y)\phi(t',x',y') \ket{0} &= \frac{4}{L^2} \sum_{m,n}
    \sum_{m',n'} \frac{\sin(p_x x)\sin(p_y y)\sin(p_{x'} x')\sin(p_{y'} y')}{\sqrt{2\omega_p}
    \sqrt{2\omega_{p'}}} \\[1em]
    &\quad \bra{0} a_p a_{p'}^{\dagger} \ket{0} e^{-i\omega_p t} e^{i\omega_{p'} t'} \\
    &= \frac{4}{L^2} \sum_{m,n} \sum_{m',n'} \frac{\sin(p_x x)\sin(p_y y)\sin(p_{x'} x')\sin(p_{y'} y')}{\sqrt{2\omega_p}
    \sqrt{2\omega_{p'}}} \\[1em]
    &\quad \delta_{mm'}\delta_{nn'} e^{-i\omega_p (t - t')} \\
    &= \frac{4}{L^2} \sum_{m,n} \frac{\sin(p_x x)\sin(p_y y)\sin(p_{x} x')\sin(p_{y} y')}{2\omega_p}
    e^{-i\omega_p (t - t')}
\end{align*}
\subsubsection*{Cylinder}
Using the mode expansion for the cylinder, we have:
\begin{align*}
    \bra{0} \phi(t,x,y)\phi(t',x',y') \ket{0} &= \frac{1}{\pi R L} \bra{0} \Big[
        \sum_{m,n} \frac{\sin\big(\frac{n\pi y}{L}\big)}{\sqrt{2\omega_{mn}}}
        \big( b_{mn} e^{-i(\omega_{mn} t - p_x x)} + b_{mn}^{\dagger} e^{i(\omega_{mn} t - p_x x)} 
        \big)
    \Big] \\[1em]
    &\quad \Big[
        \sum_{m',n'} \frac{\sin\big(\frac{n'\pi y'}{L}\big)}{\sqrt{2\omega_{m'n'}}}
        \big( b_{m'n'} e^{-i(\omega_{m'n'} t' - p_{x'} x')} + b_{m'n'}^{\dagger} e^{i(\omega_{m'n'} t' - p_{x'} x')} 
        \big)
    \Big]   
    \ket{0}
\end{align*}
where we assume $t > t'$. Expanding the products, we see that only the term
with $b_{mn} b_{m'n'}^{\dagger}$ will contribute, since $b_{mn} \ket{0} = 0$ and
$\bra{0} b_{m'n'}^{\dagger} = 0$. Thus:
\begin{align*}
    \bra{0} \phi(t,x,y)\phi(t',x',y') \ket{0} &= \frac{1}{\pi R L} \sum_{m,n}
    \sum_{m',n'} \frac{\sin\big(\frac{n\pi y}{L}\big)\sin\big(\frac{n'\pi y'}{L}\big)}{\sqrt{2\omega_{mn}}
    \sqrt{2\omega_{m'n'}}} \\[1em]
    &\quad \bra{0} b_{mn} b_{m'n'}^{\dagger} \ket{0} e^{-i(\omega_{mn} t - p_x x)} e^{i(\omega_{m'n'} t' - p_{x'} x')} \\
    &= \frac{1}{\pi R L} \sum_{m,n} \sum_{m',n'} \frac{\sin\big(\frac{n\pi y}{L}\big)\sin\big(\frac{n'\pi y'}{L}\big)}{\sqrt{2\omega_{mn}}
    \sqrt{2\omega_{m'n'}}} \\[1em]
    &\quad \delta_{mm'}\delta_{nn'} e^{-i(\omega_{mn} (t - t') - p_x (x - x'))} \\
    &= \frac{1}{\pi R L} \sum_{m,n} \frac{\sin\big(\frac{n\pi y}{L}\big)\sin\big(\frac{n\pi y'}{L}\big)}{2\omega_{mn}}
    e^{-i(\omega_{mn} (t - t') - p_x (x - x'))}
\end{align*}

\subsubsection*{Torus}
Using the mode expansion for the torus, we have:
\begin{align*}
    \bra{0} \phi(t,x,y)\phi(t',x',y') \ket{0} &= \frac{1}{(2\pi)^2 R_x R_y} \bra{0} \Big[
        \sum_{m,n} \frac{1}{\sqrt{2\omega_{mn}}}
        \big( b_{mn} e^{-i(\omega_{mn} t - p_x x - p_y y)} + b_{mn}^{\dagger} e^{i(\omega_{mn} t - p_x x + p_y y)} 
        \big)
    \Big] \\[1em]
    &\quad \Big[
        \sum_{m',n'} \frac{1}{\sqrt{2\omega_{m'n'}}}
        \big( b_{m'n'} e^{-i(\omega_{m'n'} t' - p_{x'} x' - p_{y'} y')} + b_{m'n'}^{\dagger} e^{i(\omega_{m'n'} t' - p_{x'} x' + p_{y'} y')} 
        \big)
    \Big]   
    \ket{0}
\end{align*}
where we assume $t > t'$. Expanding the products, we see that only the term
with $b_{mn} b_{m'n'}^{\dagger}$ will contribute, since $b_{mn} \ket{0} = 0$ and
$\bra{0} b_{m'n'}^{\dagger} = 0$. Thus:
\begin{align*}
    \bra{0} \phi(t,x,y)\phi(t',x',y') \ket{0} &= \frac{1}{(2\pi)^2 R_x R_y} \sum_{m,n}
    \sum_{m',n'} \frac{1}{\sqrt{2\omega_{mn}}
    \sqrt{2\omega_{m'n'}}} \\[1em]
    &\quad \bra{0} b_{mn} b_{m'n'}^{\dagger} \ket{0} e^{-i(\omega_{mn} t - p_x x - p_y y)} e^{i(\omega_{m'n'} t' - p_{x'} x' + p_{y'} y')} \\
    &= \frac{1}{(2\pi)^2 R_x R_y} \sum_{m,n} \sum_{m',n'} \frac{1}{\sqrt{2\omega_{mn}}
    \sqrt{2\omega_{m'n'}}} \\[1em]
    &\quad \delta_{mm'}\delta_{nn'} e^{-i(\omega_{mn} (t - t') - p_x (x - x') - p_y (y - y'))} \\
    &= \frac{1}{(2\pi)^2 R_x R_y} \sum_{m,n} \frac{1}{2\omega_{mn}}
    e^{-i(\omega_{mn} (t - t') - p_x (x - x') - p_y (y - y'))}
\end{align*}

\section*{Propagator from Path Integral}
Let's express the field values by a discrete Fourier series
\begin{align*}
\phi(t,x,y) &= \frac{1}{V} \sum_k e^{-i\omega_k t} w_{mn}(x,y) \phi_{mn}^a,\\
\end{align*}
Where \(V\) is the spatial volume, and \(w_k(x,y)\) are orthonormal basis functions. Since $\phi(x)$ is real, we have 
the condition $(\phi^a_{mn})^{\+}= \phi^{-a}_{mn}$. The integral measure becomes
\begin{align*}
    \mathcal{D}\phi &= \prod_k d\text{Re}\,\phi^a_{mn}\,d\text{Im}\,\phi^a_{mn} \\
\end{align*}
and later we will take the continuum limit \(V \to \infty\) to convert sums to integrals 
(up to factors of \(2\pi\)).
Looking at the action
\begin{align*}
    S[\phi] &= \frac{1}{2} \int dt \int d^2\mathbf{x} \Big[
        (\partial_t \phi)^2 - (\nabla \phi)^2 - m^2 \phi^2
    \Big] \\
\end{align*}
We can write the kinetic term as 
\begin{align*}
    \int dt \int d^2\mathbf x \; (\partial_t \phi)^2 &= \int dt \int d^2\mathbf x \Bigg(
        \left(\frac{1}{\sqrt{V}} \sum_p \dot{\phi}_p(t) w_p(x,y) \right)
        \left(\frac{1}{\sqrt{V}} \sum_q \dot{\phi}_q(t) w_q(x,y) \right)
    \Bigg) \\
    &= \int dt \; \frac{1}{V} \sum_{p,q} \dot{\phi}_p(t) \dot{\phi}_q(t) \int d^2\mathbf x \; w_p(x,y) w_q(x,y) \\
    &= \int dt \; \sum_{p,q} \dot{\phi}_p(t) \dot{\phi}_q(t) \delta_{pq} \\
    &= \int dt \; \sum_{p} |\dot{\phi}_p(t)|^2 \\
\end{align*}
Now Fourier-expand
\[
\phi_{\mathbf p}(t)=\frac{1}{V}\sum_{a}e^{-i\omega_a t}\,\phi_{\mathbf p}^{\,a}.
\]

\[
\dot\phi_{\mathbf p}(t)=\frac{1}{V}\sum_{a}(-i\omega_a)e^{-i\omega_a t}\,\phi_{\mathbf p}^{\,a}.
\]

\[
\int dt\,|\dot\phi_{\mathbf p}(t)|^2
= \int dt\,\frac{1}{V^2}\sum_{a,b}(-i\omega_a)(+i\omega_b)
  e^{-i(\omega_a-\omega_b)t}\,\phi_{\mathbf p}^{\,a}(\phi_{\mathbf p}^{\,b})^\ast.
\]

\[
\int dt\,e^{-i(\omega_a-\omega_b)t}=V\,\delta_{ab}.
\]

\[
\int dt\,|\dot\phi_{\mathbf p}(t)|^2
= \frac{1}{V}\sum_{a}\omega_a^2\,|\phi_{\mathbf p}^{\,a}|^2.
\]
Therefore
\[
\int dt \int d^2\mathbf x \; (\partial_t \phi)^2 =\frac{1}{V}\sum_{\mathbf p}\sum_{a}\omega_a^2\,|\phi_{\mathbf p}^{\,a}|^2.
\]

Similarly, the gradient term is
\begin{align*}
\int dt \int d^2\mathbf x\,(\nabla\phi)^2
&=\frac{1}{V}\sum_{\mathbf p,\mathbf q}\phi_{\mathbf p}\phi_{\mathbf q}
\int d^2\mathbf x\,\nabla w_{\mathbf p}\!\cdot\!\nabla w_{\mathbf q} \\
&=\frac{1}{V}\sum_{\mathbf p,\mathbf q}\phi_{\mathbf p}\phi_{\mathbf q}
\int d^2\mathbf x\,w_{\mathbf p}(-\nabla^2) w_{\mathbf q} \\
\end{align*}
From here, we can integrate by parts and notice that for both our relvant
boundary conditions (Dirichlet and periodic), the boundary term vanishes.
Meanwhile, for each geometry, $(-\nabla^2)w_p = p_x^2 + p_y^2 \equiv \lambda_p$.
Thus:
\begin{align*}
    \int dt \int d^2\mathbf x (\nabla \phi)^2 
    &= \int dt \frac{1}{V}\sum_{\mathbf p,\mathbf q}\lambda_q \phi_{\mathbf p}\phi_{\mathbf q}
    \int d^2\mathbf x\,w_{\mathbf p} w_{\mathbf q} \\ 
    &= \int dt \frac{1}{V}\sum_{\mathbf p,\mathbf q}\lambda_q \phi_{\mathbf p}\phi_{\mathbf q}
    \delta_{pq}\\
    &= \int dt \sum_{\mathbf p}\lambda_{\mathbf p}\,\phi_{\mathbf p}^2.
\end{align*}
Similarly, Fourier expanding
\begin{align*}
    \int dt \int d^2\mathbf x (\nabla\phi)^2 
    &= \frac{1}{V} \int dt \sum_{\mathbf p}\sum_{\mathbf a}\lambda_{\mathbf p}\,\phi_{\mathbf p}^a \phi_{\mathbf p}^b e^{-it(\omega_a + \omega_b)} .
\end{align*}
Integrating over time, we obtain $\delta(\omega_a + \omega_b)$ which collapses
$\phi_p^b \to \phi_p^{-a}$. Using $\phi_p^{-a} = \phi_p^{a\+}$, we obtain:
\begin{align*}
    \int dt \int d^2\mathbf x (\nabla\phi)^2 
    &= \frac{1}{V} \sum_{\mathbf p}\sum_{\mathbf a}\lambda_{\mathbf p}\,|\phi_{\mathbf p}^a|^2 
\end{align*}
Lastly, using identical arguments, the mass term in the action can be written as:
\begin{align*}
    \int dt \int d^2\mathbf x \; m^2\phi^2
    &= \frac{1}{V} \sum_p \sum_a m^2 |\phi_{\mathbf p}^a|^2
\end{align*}
Therefore, our full expansion of the action in Fourier space is
\begin{align*}
    S &= \frac{1}{2V} \sum_{p, a} \omega_a^2|\phi_{p}^a|^2 - 
    \lambda_p |\phi_p^a|^2 - m^2|\phi_p^a|^2 \\
      &= \frac{1}{2V} \sum_{p,a} |\phi_p^a|^2(\w_a^2 - \w_p^2) \\
      &= \frac{1}{2V} \sum_{p,a} (\text{Re}(\phi_p^a)^2 +
      \text{Im}(\phi_p^a)^2)(\w_a^2 - \w_p^2) \\
\end{align*}
Where we defined $\omega_p^2 = \lambda_p + m^2$. Now, the denominator of quantity
we wish to compute is
\begin{equation*}
    \int \mathcal{D}\phi e^{-i S[\phi]} =
\end{equation*}
\begin{align*}
    &= \int \prod_a d\text{Re}\,\phi^a_{p}\,d\text{Im}\,\phi^a_{p} \exp\left[
    \frac{-i}{2V} \sum_{p,a} (\text{Re}(\phi_p^a)^2 + \text{Im}(\phi_p^a)^2)(\w_a^2
    - \w_p^2) \right] \\
    &= \prod_a \left(\int d\text{Re}\,\phi^a_{p}\, \exp\left[
    \frac{-i}{2V} \sum_{p,a} \text{Re}(\phi_p^a)^2 (\w_a^2 - \w_p^2)  \right]
    \right) \left( \int
        d\text{Im}\,\phi^a_{p}\exp \Bigg[ \frac{-i}{2V} \sum_{p,a} 
    \text{Im}(\phi_p^a)^2 (\w_a^2 - \w_p^2) \Bigg] \right) \\
    &= \prod_{p, a} \bigg( \frac{-2i\pi V}{\w_a^2 - \w_p^2} \bigg)
\end{align*}
Where we used the standard Gaussian integral result. Now, we must compute the 
quantity
\begin{equation*}
    \int \mathcal{D}\phi e^{-i S[\phi]} \phi(x_1)\phi(x_2) =
\end{equation*}
\begin{align*}
    &= \int \prod_a d\text{Re}\,\phi^a_{p}\,d\text{Im}\,\phi^a_{p} \exp\left[
    \frac{-i}{2V} \sum_{p,a} (\text{Re}(\phi_p^a)^2 + \text{Im}(\phi_p^a)^2)(\w_a^2
    - \w_p^2) \right] \\[1em]
    &\quad \times \Bigg(\frac{1}{V} \sum_{p, a} e^{-i\omega_a
    t_1}w_p(x_1)(\text{Re}(\phi_p^a) + i\text{Im}(\phi_p^a))\Bigg) 
    \Bigg(\frac{1}{V} \sum_{q, b} e^{-i\omega_b t_2}w_q(x_2)(\text{Re}(\phi_q^b) +
    i\text{Im}(\phi_q^b))\Bigg) \\
    &= \int \prod_a d\text{Re}\,\phi^a_{p}\,d\text{Im}\,\phi^a_{p} \exp\left[
    \frac{-i}{2V} \sum_{p,a} (\text{Re}(\phi_p^a)^2 + \text{Im}(\phi_p^a)^2)(\w_a^2
    - \w_p^2) \right] \\[1em]
    &\quad \times \Bigg(\frac{1}{V^2} \sum_{p, a}\sum_{q, b} e^{-i(\omega_at_1
    + \omega_bt_2)}w_p(x_1)w_q(x_2)(\text{Re}(\phi_p^a) +
    i\text{Im}(\phi_p^a))(\text{Re}(\phi_q^b) + i\text{Im}(\phi_q^b))\Bigg) 
\end{align*}

Let $\alpha = \frac{i}{2V} \sum_{p,a} (\w_a^2 - \w_p^2)$ such that the exponential
term can be written as
\begin{align*}
    \prod_{p, a} e^{-\alpha \text{Re}(\phi_p^a)^2} e^{-\alpha \text{Im}(\phi_p^a)^2}
\end{align*}
notice now that if $a = b$ and $p = q$, these gaussian terms multiply the real and
imaginary decompositions of $\phi(x_1)$ and $\phi(x_2)$ which explicitly takes
the form
\begin{align*}
    \text{Re}(\phi^a_p)^2 + 2i\text{Im}(\phi_p^a)\text{Re}(\phi_p^a) - \text{Im}(\phi_p^a)^2
\end{align*}
where the $2i\text{Im}(\phi_p^a)\text{Re}(\phi_p^a)$ term makes for an integral
of the form
\begin{align}
    \int xe^{-\alpha x^2} = 0
    \label{eq:gausszero}
\end{align}
and thus only the $\text{Re}(\phi_p^a)^2$ and $\text{Im}(\phi_p^a)^2$ survive, 
but then we are left with 
\begin{align*}
        \int \prod_{p, a} d\text{Re}\phi_p^a d\text{Im}\phi_p^a e^{-\alpha
        \text{Re}(\phi_p^a)^2} e^{-\alpha \text{Im}(\phi_p^a)^2}\left(
            \frac{1}{V^2} \sum_{p, a} e^{-i\omega_a(t_1 + t_2)}w_p(x_1)w_p(x_2)
            (\text{Re}(\phi_p^a)^2 - \text{Im}(\phi_p^a)^2)
        \right)
\end{align*}
which we notice is a difference of identical integrals and thus this entire 
path integral vanishes. 
If $a \neq b$ and $p \neq q$, the real and imaginary decomposition looks like
\begin{align*}
    \text{Re}(\phi^a_p)\text{Re}(\phi^b_q) + i\text{Im}(\phi_p^a)\text{Re}(\phi_p^a) + i\text{Re}(\phi_p^a)\text{Im}(\phi_p^a) - \text{Im}(\phi_p^a) \text{Im}(\phi_q^b)
\end{align*}
and so every term looks like equation \ref{eq:gausszero} and the entire integral
vanishes.

So, we find that this integral only yeilds a nonzero result when $a = -b$ and
and $p = q$. Thus the real and imaginary expansion is
\begin{align}
    (\text{Re}(\phi_p^a) + i\text{Im}(\phi_p^a))(\text{Re}(\phi_p^{-a}) +
i\text{Im}(\phi_p^{-a}))
\end{align}
and using our reality condition $\phi_p^{-a} = (\phi_p^a)^{\+}$, we have
\begin{align*}
    (\text{Re}(\phi_p^a) + i\text{Im}(\phi_p^a))(\text{Re}(\phi_p^a) -
    i\text{Im}(\phi_p^a)) 
    = \text{Re}(\phi^a_p)^2 + 2i\text{Im}(\phi_p^a)\text{Re}(\phi_p^a) +
    \text{Im}(\phi_p^a)^2
\end{align*}
Where once again the cross terms cancel due to equation \ref{eq:gausszero}
but instead of the real and imaginary squared terms cancelling, they add.
This leaves us to solve
\begin{align*}
    \int \prod_{p, a} d\text{Re}\phi_p^a d\text{Im}\phi_p^a e^{-\alpha
    \text{Re}(\phi_p^a)^2} e^{-\alpha \text{Im}(\phi_p^a)^2}\left(
        \frac{1}{V^2} \sum_{p, a} e^{-i\omega_a(t_1 - t_2)}w_p(x_1)w_p(x_2)
        (\text{Re}(\phi_p^a)^2 + \text{Im}(\phi_p^a)^2)
    \right)
\end{align*}
which is composed of products of integrals of the form
\begin{align*}
    \int d\text{Re}\phi_p^ae^{-\alpha \text{Re}(\phi_p^a)^2}
    \text{Re}(\phi_p^a)^2 \int d\text{Im}\phi_p^a e^{-\alpha
    \text{Im}(\phi_p^a)^2} \quad \text{and} \quad 
    \int d\text{Re}\phi_p^a e^{-\alpha \text{Re}(\phi_p^a)^2} \int
    d\text{Im}\phi_p^a e^{-\alpha \text{Im}(\phi_p^a)^2} \text{Im}(\phi_p^a)^2
\end{align*}
and thus we can use the results
\begin{align*}
    \int e^{-\alpha x^2} = \sqrt{\frac{\pi}{\alpha}} \quad \text{and} \quad 
    \int x^2 e^{-\alpha x^2} = \frac{1}{2\alpha}\sqrt{\frac{\pi}{\alpha}}
\end{align*}
to obtain
\begin{align*}
    \frac{1}{V^2} \sum_{p, a, l} \Bigg(
        \prod_k \frac{1}{2\alpha_l}\sqrt{\frac{\pi}{\alpha_k}}
        \prod_{k'} \sqrt{\frac{\pi}{\alpha_{k'}}} + 
        \prod_k \frac{1}{2\alpha_l}\sqrt{\frac{\pi}{\alpha_k}}
        \prod_{k'} \sqrt{\frac{\pi}{\alpha_{k'}}}
    \Bigg) w(x_1)w(x_2)e^{-i\omega_a(t_1 - t_2)}
\end{align*}
Where we only obtain a single one of the $\frac{1}{2\alpha}\sqrt{\frac{\pi}{\alpha}}$
terms, which occurs when the $a$ in the summation matches that of the exponential
term, which we label as $\alpha_l$. Since $k$ and $k'$ go over all possibilities,
this simplifies to
\begin{align*}
    \frac{1}{V^2} \sum_{p, a, l} \Bigg(
    \prod_k \frac{\pi}{\alpha_k} \Bigg)
    \frac{1}{\alpha_l}w(x_1)w(x_2)e^{-i\omega_a(t_1 - t_2)}
\end{align*}
And plugging back in our expressions for $\alpha$
\begin{align*}
    \frac{1}{V} \sum_{p, a, l} \Bigg(
    \prod_k \frac{-2i\pi V}{\w_a^2 - \w_p^2} \Bigg)
    \frac{-2i\pi w(x_1)w(x_2)e^{-i\omega_a(t_1 - t_2)}}{\w_a^2 - \w_p^2 + i\epsilon}
\end{align*}
where we recognize the product in brackets as being exactly the denominator and
the $i\epsilon$ is added to obtain the Feynmann propagator. We can write the sum
over $l$ as an integral and obtain
\begin{align*}
\frac{1}{V} \sum_{p, a} \int \frac{d\w_a}{2\pi} 
    \frac{-2i\pi w(x_1)w(x_2)e^{-i\omega_a(t_1 - t_2)}}{\w_a^2 - \w_p^2 + i\epsilon}
\end{align*}
and performing the integral by closing the contour from underneath we get
\begin{equation}
    \frac{\int \mathcal{D}\phi e^{-i S[\phi]} \phi(x_1)\phi(x_2)}{\int
    \mathcal{D}\phi e^{-i S[\phi]}} = 
    \frac{1}{V} \sum_p \frac{w(x_1)w(x_2)e^{-i\omega_p(t_1 - t_2)}}{2\w_p}
    \label{eq:FINALLY}
\end{equation}
For the plane, our normalized basis functions are $u_n(x) = \sqrt{2 / L}\sin(n\pi x / L)$
and we have $w_{nm}(x_1) = u_n(x_1) u_m(x_1)$ and so we obtain
\begin{align*}
    \frac{\int \mathcal{D}\phi e^{-i S[\phi]} \phi(x_1)\phi(x_2)}{\int
    \mathcal{D}\phi e^{-i S[\phi]}} &= \frac{4}{L^2} \sum_{m, n}
    \frac{e^{-i\w_p(t_1 - t_2)}}{2\w_p}\sin\left(\frac{n\pi
    x_1}{L}\right)\sin\left(\frac{m\pi x_1}{L}\right) \sin\left(\frac{n\pi
x_2}{L}\right)\sin\left(\frac{m\pi x_2}{L}\right)
\end{align*}
where we let the normalization account for the volume factor.

\subsubsection*{Cylinder}
For the cylinder, all we have to do is change the basis functions to
\begin{align*}
u_m(x)=\frac{e^{i p_x x}}{\sqrt{2\pi R}},\quad p_x=\frac{m}{R},\quad m\in\mathbb{Z},\qquad
v_n(y)=\sqrt{\frac{2}{L}}\sin\!\frac{n\pi y}{L},\quad n=1,2,3,\ldots
\end{align*}
so that when substituted into \ref{eq:FINALLY}, we obtain
\begin{align}
\langle 0|\phi(t,x,y)\phi(t',x',y')|0\rangle
= \frac{1}{\pi R L} \sum_{m\in\mathbb{Z}}\sum_{n=1}^\infty
\frac{e^{i p_x (x - x')}\,v_n(y)v_n(y')}{2\,\omega_{mn}}\;
e^{-i\omega_{mn}(t-t')}.
\end{align}

\subsubsection*{Torus}
For the torus, we change the basis functions to 
\begin{align*}
u_m(x)=\frac{e^{i p_x x}}{\sqrt{2\pi R_x}},\quad p_x=\frac{m}{R_x},\quad m\in\mathbb{Z},\qquad
v_n(y)=\frac{e^{i p_y y}}{\sqrt{2\pi R_y}},\quad p_y=\frac{n}{R_y},\quad n\in\mathbb{Z},
\end{align*}
so that when substituted into \eqref{eq:FINALLY}, we get
\begin{align}
\langle 0|\phi(t,x,y)\phi(t',x',y')|
0\rangle
= \frac{1}{(2\pi)^2 R_x R_y} \sum_{m,n\in\mathbb{Z}}
\frac{e^{i p_x (x - x')}\,e^{i p_y (y - y')}}{2\,\omega_{mn}}\;
e^{-i\omega_{mn}(t-t')}.
\end{align}
therefore both methods are equivalent.


\section*{Question 5: Interaction picture and the Heisenberg picture}
Fix a Cauchy slice $\Sigma_{t_0}$ (say $t_0=0$). The fields are
$\phi(\mathbf{x},0)$ and $\pi(\mathbf{x},0)$, and the full Hamiltonian
\[
H \;=\; H_0 + H_{\text{int}}
\]
is the spatial integral of the Hamiltonian density on $\Sigma_{t}$, evaluated at $t=t_0$.
Here $H_0$ is quadratic (free theory), while $H_{\text{int}}$ contains interactions such as $\lambda \phi^4$.

\subsection*{Schr\"odinger picture (S).}
Operators are time independent, states carry all time dependence:
\[
i\frac{d}{dt}\,|\Psi_S(t)\rangle = H\,|\Psi_S(t)\rangle,
\qquad
\phi_S(\mathbf{x},t)=\phi_S(\mathbf{x}).
\]
The Schr\"odinger evolution operator is
\[
U_S(t,t_0)=e^{-iH(t-t_0)},\qquad |\Psi_S(t)\rangle = U_S(t,t_0)|\Psi_S(t_0)\rangle.
\]

\subsection*{Heisenberg picture (H).}
States are fixed, operators evolve with the full Hamiltonian:
\[
|\Psi_H\rangle = |\Psi_S(t_0)\rangle,\qquad
\phi_H(\mathbf{x},t) = e^{+iHt}\,\phi_S(\mathbf{x})\,e^{-iHt}.
\]
Hence
\[
i\frac{d}{dt}\phi_H(\mathbf{x},t) = [\phi_H(\mathbf{x},t),H],
\]
so $\phi_H$ satisfies the full interacting equation of motion.

\subsection*{Interaction picture (I).}
Split $H=H_0+H_{\text{int}}$. Define
\[
\phi_I(\mathbf{x},t) \equiv e^{+iH_0 t}\,\phi_S(\mathbf{x})\,e^{-iH_0 t},
\qquad
|\Psi_I(t)\rangle \equiv e^{+iH_0 t}\,|\Psi_S(t)\rangle.
\]
Then the fields evolve freely:
\[
i\frac{d}{dt}\phi_I(\mathbf{x},t) = [\phi_I(\mathbf{x},t),H_0],
\]
while the states evolve with the interaction-picture Hamiltonian
\[
H_I(t) \equiv e^{+iH_0 t}\,H_{\text{int}}\,e^{-iH_0 t}.
\]
Indeed,
\begin{align}
i\frac{d}{dt}|\Psi_I(t)\rangle
&= i\frac{d}{dt}\!\big(e^{+iH_0 t}|\Psi_S(t)\rangle\big)
= H_0 e^{+iH_0 t}|\Psi_S(t)\rangle + e^{+iH_0 t}\big(-iH|\Psi_S(t)\rangle\big)\nonumber\\
&= -\,e^{+iH_0 t}H_{\text{int}}e^{-iH_0 t}\,|\Psi_I(t)\rangle
= H_I(t)\,|\Psi_I(t)\rangle.
\label{eq:IP-Schro}
\end{align}

In the interaction picture, the time-evolution operator $U_I(t,t_0)$ is defined as the solution of
\[
i\,\frac{d}{dt}U_I(t,t_0) = H_I(t)\,U_I(t,t_0),
\qquad U_I(t_0,t_0)=\mathbf{1},
\]
where $H_I(t)$ is the interaction Hamiltonian in the interaction picture.

To check that the time-ordered exponential
\[
U_I(t,t_0) = \mathcal{T}\exp\!\left[-i\!\int_{t_0}^{t}\!H_I(t')\,dt'\right]
\]
actually satisfies this, we can expand the exponential definition of the time-ordered operator as
\[
U_I(t,t_0)
= \mathbf{1} - i\!\int_{t_0}^{t}\!dt_1\,H_I(t_1)
+ (-i)^2\!\int_{t_0}^{t}\!dt_1\!\!\int_{t_0}^{t_1}\!dt_2\,H_I(t_1)H_I(t_2) + \cdots.
\]

Now differentiate both sides with respect to $t$:
\begin{align*}
\frac{d}{dt}U_I(t,t_0)
&= -i\,H_I(t)
+ (-i)^2\!\int_{t_0}^{t}\!dt_2\,H_I(t)H_I(t_2)
+ (-i)^3\!\int_{t_0}^{t}\!dt_2\!\!\int_{t_0}^{t_2}\!dt_3\,H_I(t)H_I(t_2)H_I(t_3)+\cdots. \\
&= -i\,H_I(t)\,\Big[\mathbf{1}
- i\!\int_{t_0}^{t}\!dt_2\,H_I(t_2)
+ (-i)^2\!\int_{t_0}^{t}\!dt_2\!\!\int_{t_0}^{t_2}\!dt_3\,H_I(t_2)H_I(t_3)+\cdots\Big] \\
&= -i\,H_I(t)\,U_I(t,t_0).
\end{align*}
Therefore
\[
i\,\frac{d}{dt}U_I(t,t_0) = H_I(t)\,U_I(t,t_0),
\]
as required. We also have $U_I(t_0,t_0)=\mathbf{1}$ since all integrals vanish when $t=t_0$.

If all $H_I(t)$ at different times commuted with each other, then we could simply write
\[
U_I(t,t_0) = \exp\!\left[-i\!\int_{t_0}^{t}\!H_I(t')\,dt'\right],
\]
and the ordinary exponential would work fine.

However, in quantum field theory the operators $H_I(t_1)$ and $H_I(t_2)$ do not generally commute:
\[
[H_I(t_1),\,H_I(t_2)] \neq 0.
\]
That means the order in which we multiply them matters. An ordinary exponential assumes all terms commute,
so it would mix up the order and give the wrong result.

To fix this, we need to order the operators in terms of increasing time.
\[
\mathcal{T}\big[H_I(t_1)H_I(t_2)\big] =
\begin{cases}
H_I(t_1)H_I(t_2), & \text{if } t_1 > t_2,\\[0.3em]
H_I(t_2)H_I(t_1), & \text{if } t_2 > t_1.
\end{cases}
\]
In the interaction picture, we explicitly separate the free evolution $e^{-iH_0(t-t_0)}$ from the
interaction effects, so that fields evolve as in the free theory while the interaction acts on the states.
This separation introduces the time-dependent $H_I(t)$, which cannot be pulled out of the integral as a constant.

Therefore, one cannot replace $H_I(t)$ by the full $H$ in the exponential:
the equality
\[
U_I(t,t_0) = e^{-iH(t-t_0)}
\]
does not hold because $H_I(t)$ is not constant in time and does not commute with itself at different times.
Instead, the exact relation between the full and interaction-picture evolution operators is
\[
e^{-iH(t-t_0)} \;=\; e^{-iH_0(t-t_0)}\,U_I(t,t_0),
\]
which shows how the full dynamics factorizes into free evolution and an additional
time-ordered contribution from the interaction Hamiltonian.

\section*{Question 6: An interesting path-integral transformation}
So the local operator $\mathcal{O}(\phi(x))$ allows us to write the interaction piece
\[
S_1[\phi] = \frac{1}{2} \sum_{A, B} \int d^dx \, d^dy \, \mathcal{O}_A(\phi(x)) \, \Delta_{AB}(x,y) \, \mathcal{O}_B(\phi(y)).
\]
and we are allowed to assume that 
\[
\exp\big\{- S_1[\phi]\big\} \propto \int \mathcal{D}\alpha \exp\bigg\{-S_2[\phi, \alpha]\bigg\},
\]
and we want to show
\[
    S_2[\phi, \alpha] = \frac{1}{2} \sum_{A, B} \int d^4x d^4y \alpha_A(x)
    (\Delta^{-1})_{AB}(x,y)\alpha_B(y) - \sum_A\int d^4z \alpha_A(z) \mathcal{O}_a(x)
\]
Let us complete the square by introducing a shifted fields
\[
    \alpha_A(x) = \alpha'_A(x) + \sum_C \int d^4 z \Delta_{CA}(x, z) \mathcal{O}_C(z) \\
\]
\[
    \alpha_B(y) = \alpha'_B(y) + \sum_D \int d^4 z \Delta_{DB}(y, z) \mathcal{O}_D(z)
\]
such that $S_2[\phi, \alpha]$ becomes
\begin{align*}
    &= \frac{1}{2}\sum_{A,B} \int d^4x \, d^4 y \left( \alpha'_A(x) + \sum_C
    \int d^4z \, \Delta_{CA}(x,z) \mathcal{O}_C(z) \right) (\Delta^{-1})_{AB}(x,y)
    \\[1em]
    &\quad \times \left(
            \alpha'_B(y) + \sum_D \int d^4 z \, \Delta_{DB}(y, z) \mathcal{O}_D(z)
        \right) \\[1em]
    &\quad - \sum_A \int d^4 z \left( \alpha'_A(z) + \sum_C
    \int d^4x \, \Delta_{CA}(z,x) \mathcal{O}_C(x) \right)\mathcal{O}_A(x) \\
    &= \frac{1}{2}\sum_{A,B} \int d^4x \, d^4 y \;
    \alpha'_A(x)(\Delta^{-1})_{AB}(x,y)\alpha'_B(y) \\[1em]
    &\quad + \frac{1}{2} \sum_{A,B} \int d^4x \, d^4y \sum_C \int d^4z \Delta_{CA}(x, z)
    \mathcal{O}_C(z)(\Delta^{-1})_{AB}(x,y) \alpha_B'(y) \\[1em]
    &\quad + \frac{1}{2} \sum_{A,B} \int d^4x \, d^4y \; \alpha'_A(x)(\Delta^{-1})_{AB}(x,y)
    \sum_D \int d^4z \Delta_{DB}(y,z)\mathcal{O}_D(z) \\[1em]
    &\quad + \frac{1}{2} \sum_{A,B} \int d^4x \, d^4y \; \sum_C \int d^4z \;
    \Delta_{CA}(x, z) \mathcal{O}_C(z) (\Delta^{-1})_{AB}(x,y) \sum_D \int d^4w \;
    \Delta_{DB}(y, w) \mathcal{O}_D(w) \\[1em]
    &\quad - \sum_A \int d^4z \; \alpha'_A(z)\mathcal{O}_A(x) - \sum_A \int d^4z
    \sum_C \int d^4x \; \Delta_{CA}(z, x)\mathcal{O}_C(x)\mathcal{O}_A(x)
\end{align*}
From here, we can perform contractions of the form $\Delta_{AB}\Delta^{-1}_{AC}
= \delta_{BC}$ where it appears as if doing this would require commuting a 
$\Delta$ with an $\mathcal{O}$, but it just appears as such because we can factor
such $\mathcal{O}$s who spectate in the contraction. This gives
\begin{align*}
    &= \frac{1}{2}\sum_{A,B} \int d^4x \, d^4 y \;
    \alpha'_A(x)(\Delta^{-1})_{AB}(x,y)\alpha'_B(y) \\[1em]
    &\quad + \frac{1}{2} \sum_{B, C} \int d^4x \, d^4y \int d^4z \;
    \mathcal{O}_C(z)\delta_{CB}(z,y) \alpha_B'(y) \\[1em]
    &\quad + \frac{1}{2} \sum_{A,D} \int d^4x \, d^4y \; \alpha'_A(x)
    \int d^4z \; \delta_{AD}(x,z) \mathcal{O}_D(z) \\[1em]
    &\quad + \frac{1}{2} \sum_{A,B} \int d^4x \, d^4y \; \sum_C \int d^4z \;
     \mathcal{O}_C(z) \sum_D \int d^4w \;
    \Delta_{DB}(y, w) \mathcal{O}_D(w) \\[1em]
    &\quad - \sum_A \int d^4z \; \alpha'_A(z)\mathcal{O}_A(x) - \sum_A \int d^4z
    \sum_C \int d^4x \; \Delta_{CA}(z, x)\mathcal{O}_C(x)\mathcal{O}_A(x)
\end{align*}
performing the integrals and collapsing the $\delta$s, we obtain
\begin{align*}
    \frac{1}{2} \sum_{AB} \int d^4x \, d^4y \alpha'_A(x) (\Delta^{-1})_{AB}(x,y)
    \alpha'_B(y) - \frac{1}{2}\sum_{AB}\int d^4 x d^4 z \mathcal{O}_A(x) 
    \Delta_{AB}(x,y)\mathcal{O}_B(y)
\end{align*}
If we exponentiate this result and integrate over field configurations $\alpha$
\begin{align*}
    &\int \mathcal{D}\alpha \exp\left[
    \frac{1}{2} \sum_{AB} \int d^4x \, d^4y \alpha'_A(x) (\Delta^{-1})_{AB}(x,y)
    \alpha'_B(y) - \frac{1}{2}\sum_{AB}\int d^4 x d^4 z \mathcal{O}_A(x) 
    \Delta_{AB}(x,y)\mathcal{O}_B(y)
    \right] \\
    &= \int \mathcal{D}\alpha \exp\left[
    \frac{1}{2} \sum_{AB} \int d^4x \, d^4y \alpha'_A(x) (\Delta^{-1})_{AB}(x,y)
    \alpha'_B(y)\right] \exp \left[ \frac{1}{2}\sum_{AB}\int d^4 x d^4 z
    \mathcal{O}_A(x) \Delta_{AB}(x,y)\mathcal{O}_B(y)
    \right]
\end{align*}
where the second exponential term is independent of $\alpha$. Thus the first
integral yields a proportionality factor of the form
\[
    \propto 2\pi\det(\Delta)^{\frac{1}{2}}
\]
and clearly the second exponential factor has $S_1[\phi]$ as its exponent.
\end{document}
