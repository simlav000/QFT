\documentclass[12pt]{article}

% Packages
\usepackage[margin=1in]{geometry}
\usepackage{fancyhdr}
\usepackage{titlesec}
\usepackage{amsmath, amssymb, bm}
\usepackage{cancel}

\usepackage{mathtools}
\newcommand{\w}{\omega}
\newcommand{\phip}{\phi^{'}}
\newcommand{\phipp}{\phi^{''}}
\newcommand{\delmu}{\partial_{\mu}}
\newcommand{\delMu}{\partial^{\mu}}
\newcommand{\wk}{\omega_{\mathbf{k}}}
\newcommand{\wkp}{\omega_{\mathbf{k'}}}
\DeclarePairedDelimiter\bra{\langle}{\rvert}
\DeclarePairedDelimiter\ket{\lvert}{\rangle}
\DeclarePairedDelimiterX\braket[2]{\langle}{\rangle}{#1\,\delimsize\vert\,\mathopen{}#2}
\usepackage{braket}

% Header & Footer
\pagestyle{fancy}
\fancyhf{}
\fancyhead[L]{Name:  \underline{Simon Lavoie}}
\fancyhead[C]{Class: \underline{PHYS 610}}
\fancyhead[R]{Student ID: \underline{261 051 325}}
\fancyfoot[C]{\thepage}

% Title format for neatness
\titleformat{\section}{\large\bfseries}{\thesection.}{0.5em}{}
\titleformat{\subsection}{\normalsize\bfseries}{\thesubsection)}{0.5em}{}

% Document
\begin{document}

\section*{Question 3}
We can imagine $\phi$ and $\phi^{\dagger}$ as a sum of two real fields $\phi_1$ 
and $\phi_2$, such that $\phi = \phi_1 + i\phi_2$ and $\phi^{\dagger} = \phi_1 -
i\phi_2$. Plugging this presecription into our Lagrangian
\begin{align*}
   \mathcal{L} &= \frac{1}{2}\delmu\phi\delMu\phi^{\dagger} \\ 
               &= \frac{1}{2}\delmu(\phi_1 + i \phi_2)\delMu(\phi_1 - i\phi_2) \\
               &= \frac{1}{2}\delmu\phi_1\delMu\phi_1
               + \frac{1}{2}\delmu\phi_2\delMu\phi_2
               + \frac{i}{2}\delmu\phi_2\delMu\phi_1
               - \frac{i}{2}\delmu\phi_1\delMu\phi_2 \\
               &= \frac{1}{2}\delmu\phi_1\delMu\phi_1
               + \frac{1}{2}\delmu\phi_2\delMu\phi_2
               + \frac{i}{2}\delmu\delMu\phi_2\phi_1
               - \frac{i}{2}\delmu\delMu\phi_2\delMu\phi_1 \\
               &= \frac{1}{2}\delmu\phi_1\delMu\phi_1
               + \frac{1}{2}\delmu\phi_2\delMu\phi_2
\end{align*}
because two independent scalar fields $\phi_1$, $\phi_2$ commute. Thus, we can
see this Lagrangian describes two independent scalar field theories. Now,
let us quantize both $\phi_1$ and $\phi_2$ independently, giving them each
their own creation and annihilation operators, which we shall label $\alpha$ 
and $\beta$.
\begin{align*}
   \phi_1(x) &= \int \frac{d^3k}{(2\pi)^3} \frac{1}{\sqrt{2\omega_k}} (\alpha_k
   e^{-ik\cdot x} + \alpha_k^{\dagger}e^{ik\cdot x}) \\
   \phi_2(x) &= \int \frac{d^3k}{(2\pi)^3} \frac{1}{\sqrt{2\omega_k}} (\beta
   e^{-ik\cdot x} + \beta^{\dagger}e^{ik\cdot x}) \\
\end{align*}
Using these definitions, we write $\phi$ and $\phi^{\dagger}$
\begin{align*}
   \phi(x) &= \int \frac{d^3k}{(2\pi)^3} \frac{1}{\sqrt{2\omega_k}} \left(
   (\alpha_k + i\beta_k)e^{-ik\cdot x} + (\alpha^{\dagger}_k +
   i\beta^{\dagger}_k)e^{ik\cdot x} \right) \\
   \phi^{\dagger}(x) &= \int \frac{d^3k}{(2\pi)^3} \frac{1}{\sqrt{2\omega_k}}
   \left( (\alpha_k - i\beta_k)e^{ik\cdot x} + (\alpha^{\dagger}_k -
   i\beta^{\dagger}_k)e^{-ik\cdot x} \right) \\
\end{align*}
let us now define $a = \alpha_k + i\beta_k$ and $b = \alpha_k - i\beta_k$. Then,
immediately we see $a^{\dagger} = \alpha_k^{\dagger} - i\beta_k^{\dagger}$ and 
$b^{\dagger} = \alpha_k^{\dagger} + i \beta_k^{\dagger}$ such that

\begin{align*}
   \phi(x) &= \int \frac{d^3k}{(2\pi)^3} \frac{1}{\sqrt{2\omega_k}} \left(
   a_k e^{-ik\cdot x} + b_k^{\dagger}e^{ik\cdot x}
   \right) \\
   \phi^{\dagger}(x) &= \int \frac{d^3k}{(2\pi)^3} \frac{1}{\sqrt{2\omega_k}}
   \left( a^{\dagger}_k e^{ik\cdot x} + b_k e^{-ik\cdot x} \right) \\
\end{align*}
Now, let's compute $[\phi(x), \phi^{\dagger}(y)]$

\begin{align*}
   [\phi(x),\phi^\dagger(y)]
   &=\int \frac{d^3\mathbf{k}}{(2\pi)^3}
          \frac{d^3\mathbf{k}'}{(2\pi)^3}\,
          \frac{1}{\sqrt{2\omega_{\mathbf{k}}}
                   \sqrt{2\omega_{\mathbf{k}'}}}\,
   \Big[
     \big(a_{\mathbf{k}} e^{-ik\cdot x}
      + b_{\mathbf{k}}^\dagger e^{+ik\cdot x}\big),
     \big(a_{\mathbf{k}'}^\dagger e^{+ik'\cdot y}
      + b_{\mathbf{k}'} e^{-ik'\cdot y}\big)
   \Big] \\[6pt]
   &=\int \frac{d^3\mathbf{k}}{(2\pi)^3}
           \frac{d^3\mathbf{k}'}{(2\pi)^3}\,
           \frac{1}{\sqrt{2\omega_{\mathbf{k}}}
                    \sqrt{2\omega_{\mathbf{k}'}}}\,
   \Big(
     a_{\mathbf{k}} a_{\mathbf{k}'}^\dagger e^{-ik\cdot x} e^{+ik'\cdot y}
    +a_{\mathbf{k}} b_{\mathbf{k}'} e^{-ik\cdot x} e^{-ik'\cdot y} \\
   &\qquad\qquad
    +b_{\mathbf{k}}^\dagger a_{\mathbf{k}'}^\dagger e^{+ik\cdot x} e^{+ik'\cdot y}
    +b_{\mathbf{k}}^\dagger b_{\mathbf{k}'} e^{+ik\cdot x} e^{-ik'\cdot y} \\
   &\qquad\qquad
    -a_{\mathbf{k}'}^\dagger a_{\mathbf{k}} e^{+ik'\cdot y} e^{-ik\cdot x}
    -b_{\mathbf{k}'} a_{\mathbf{k}} e^{-ik'\cdot y} e^{-ik\cdot x} \\
   &\qquad\qquad
    -a_{\mathbf{k}'}^\dagger b_{\mathbf{k}}^\dagger e^{+ik'\cdot y} e^{+ik\cdot x}
    -b_{\mathbf{k}'} b_{\mathbf{k}}^\dagger e^{-ik'\cdot y} e^{+ik\cdot x}
   \Big) \\[6pt]
   &=\int \frac{d^3\mathbf{k}}{(2\pi)^3}
           \frac{d^3\mathbf{k}'}{(2\pi)^3}\,
           \frac{1}{\sqrt{2\omega_{\mathbf{k}}}
                    \sqrt{2\omega_{\mathbf{k}'}}}\,
   \Big(
    [a_{\mathbf{k}},a_{\mathbf{k}'}^\dagger]\,e^{-ik\cdot x}e^{+ik'\cdot y}
    +[a_{\mathbf{k}},b_{\mathbf{k}'}]\,e^{-ik\cdot x}e^{-ik'\cdot y} \\
   &\qquad\qquad
    +[b_{\mathbf{k}}^\dagger,a_{\mathbf{k}'}^\dagger]\,e^{+ik\cdot x}e^{+ik'\cdot y}
    +[b_{\mathbf{k}}^\dagger,b_{\mathbf{k}'}]\,e^{+ik\cdot x}e^{-ik'\cdot y}
   \Big) \\[6pt]
   &=\int \frac{d^3\mathbf{k}}{(2\pi)^3}
           \frac{d^3\mathbf{k}'}{(2\pi)^3}\,
           \frac{1}{\sqrt{2\omega_{\mathbf{k}}}
                    \sqrt{2\omega_{\mathbf{k}'}}}\,
   \Big(
    (2\pi)^3\delta^{(3)}(\mathbf{k}-\mathbf{k}')\,e^{-ik\cdot x}e^{+ik'\cdot y}
    -(2\pi)^3\delta^{(3)}(\mathbf{k}-\mathbf{k}')\,e^{+ik\cdot x}e^{-ik'\cdot y}
   \Big) \\[6pt]
   &=\int \frac{d^3\mathbf{k}}{(2\pi)^3}\,
         \frac{1}{2\omega_{\mathbf{k}}}\,
         \Big(e^{-ik\cdot(x-y)} - e^{+ik\cdot(x-y)}\Big). \\
   &= \int \frac{d^3k}{(2\pi)^3} e^{-i\mathbf{k}(\mathbf{x} - \mathbf{y})}
   \left(\frac{1}{2\omega_k} e^{-ik\cdot(x-y)}\bigg|_{k^0 = \omega_k}
   + \frac{1}{-2\omega_k}e^{-ik\cdot(x-y)}\bigg|_{k^0 = -\omega_k}\right) \\
   &= \int \frac{d^3k}{(2\pi)^3} \int \frac{dp^0}{2\pi i} \frac{-1}{p^2 - m^2}
   e^{-ik\cdot (x-y)} \\
   &= \int \frac{d^4k}{(2\pi)^4} \frac{i}{p^2 - m^2} e^{-ik\cdot(x-y)}
\end{align*}
which we immediately recognize as $D_R(x-y)$, the retarded propagator. It makes
sense that the commutator would yield the retarded propagator because the
commutator measures how much the quantum field $\phi$ at one point in spacetime
$x$ interferes with $\phi^{\dagger}$ at a later point $y$, and $D_R(x-y)$
describes what this causal effect is.

Next, to find the Hamiltonian, we will use the usual formula
\begin{equation*}
   H = \int d^3x \mathcal{H} = \frac{1}{2} \int d^3 x (\pi^2(x) + 
\end{equation*}
\end{document}

